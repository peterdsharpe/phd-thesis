\chapter{Extended Code References}
\label{chap:code}


\section{AeroSandbox Benchmark on Beam Static Structural Analysis}

This Python code is used for the AeroSandbox performance benchmark used in Section \ref{sec:benchmark_gpkit}, with associated runtimes shown as the line labeled ``AeroSandbox'' in Figure \ref{fig:benchmark_gp_beam}. In this analysis problem, a cantilever beam is subject to a distributed load, and the goal is to compute the state of the deflected beam. The code is shown in Listing \ref{lst:gpkit_beam}.

\begin{listing}[h]
    \begin{minted}{python}
import aerosandbox as asb
import aerosandbox.numpy as np

N = 50  # Number of discretization nodes
L = 6  # Overall length of the beam [m]
EI = 1.1e4  # Bending stiffness [N*m^2]
q = 110 * np.ones(N)  # Distributed load [N/m]

x = np.linspace(0, L, N)  # Node locations along beam length [m]

opti = asb.Opti()  # Initialize an optimization environment

w = opti.variable(init_guess=np.zeros(N))  # Displacement [m]

th = opti.derivative_of(  # Slope [rad]
    w, with_respect_to=x,
    derivative_init_guess=np.zeros(N),
)

M = opti.derivative_of(  # Moment [N*m]
    th * EI, with_respect_to=x,
    derivative_init_guess=np.zeros(N),
)

V = opti.derivative_of(  # Shear force [N]
    M, with_respect_to=x,
    derivative_init_guess=np.zeros(N),
)

opti.constrain_derivative(  # Shear integration
    variable=V, with_respect_to=x,
    derivative=q,
)

opti.subject_to([  # Boundary conditions
    w[0] == 0,
    th[0] == 0,
    M[-1] == 0,
    V[-1] == 0,
])

sol = opti.solve()

print(sol(w[-1]))  # Prints the tip deflection; should be 1.62 m.
    \end{minted}
    \caption{AeroSandbox code for a static structural analysis of a beam. Written in Python.}
    \label{lst:gpkit_beam}
\end{listing}

\newpage
\section{Simple Aircraft (SimpleAC)}
\label{sect:simpleac-code}

The Simple Aircraft (SimpleAC) problem described in Section \ref{sect:simpleac} can be solved using the following AeroSandbox code, reproduced from Sharpe \cite{sharpe_aerosandbox_2021}:

%\begin{listing}
    \begin{minted}{python}

import aerosandbox as asb
import aerosandbox.numpy as np

opti = asb.Opti()

### Env. constants
g = 9.81  # gravitational acceleration, m/s^2
mu = 1.775e-5  # viscosity of air, kg/m/s
rho = 1.23  # density of air, kg/m^3
rho_f = 817  # density of fuel, kg/m^3

### Non-dimensional constants
C_Lmax = 1.6  # stall CL
e = 0.92  # Oswald's efficiency factor
k = 1.17  # form factor
N_ult = 3.3  # ultimate load factor
S_wetratio = 2.075  # wetted area ratio
tau = 0.12  # airfoil thickness to chord ratio
W_W_coeff1 = 2e-5  # wing weight coefficient 1
W_W_coeff2 = 60  # wing weight coefficient 2

### Dimensional constants
Range = 1000e3  # aircraft range, m
TSFC = 0.6 / 3600  # thrust specific fuel consumption, 1/sec
V_min = 25  # takeoff speed, m/s
W_0 = 6250  # aircraft weight excluding wing, N

### Free variables (same as SimPleAC, with extraneous variables removed)
AR = opti.variable(init_guess=10, log_transform=True)  # aspect ratio
S = opti.variable(init_guess=10, log_transform=True)  # total wing area, m^2
V = opti.variable(init_guess=100, log_transform=True)  # cruise speed, m/s
W = opti.variable(init_guess=10000, log_transform=True)  # total aircraft weight, N
C_L = opti.variable(init_guess=1, log_transform=True)  # lift coefficient
W_f = opti.variable(init_guess=3000, log_transform=True)  # fuel weight, N
V_f_fuse = opti.variable(init_guess=1, log_transform=True)  # fuel volume in the fuselage, m^3

### Wing weight
W_w_surf = W_W_coeff2 * S
W_w_strc = W_W_coeff1 / tau * N_ult * AR ** 1.5 * np.sqrt(
    (W_0 + V_f_fuse * g * rho_f) * W * S
)
W_w = W_w_surf + W_w_strc

### Entire weight
opti.subject_to(W >= W_0 + W_w + W_f)

### Lift equals weight constraint
opti.subject_to([
    W_0 + W_w + 0.5 * W_f <= 0.5 * rho * S * C_L * V ** 2,
    W <= 0.5 * rho * S * C_Lmax * V_min ** 2,
])

### Flight duration
T_flight = Range / V

### Drag
Re = (rho / mu) * V * (S / AR) ** 0.5
C_f = 0.074 / Re ** 0.2

CDA0 = V_f_fuse / 10

C_D_fuse = CDA0 / S
C_D_wpar = k * C_f * S_wetratio
C_D_ind = C_L ** 2 / (np.pi * AR * e)
C_D = C_D_fuse + C_D_wpar + C_D_ind
D = 0.5 * rho * S * C_D * V ** 2

opti.subject_to(W_f >= TSFC * T_flight * D)

V_f = W_f / g / rho_f
V_f_wing = 0.03 * S ** 1.5 / AR ** 0.5 * tau

V_f_avail = V_f_wing + V_f_fuse

opti.subject_to(V_f_avail >= V_f)

opti.minimize(W_f)

sol = opti.solve()
    \end{minted}
%    \caption{AeroSandbox code for the Simple Aircraft (SimpleAC) problem. Written in Python.}
%    \label{lst:simpleac}
%\end{listing}

\newpage
\section{Efficient Computational Implementation of an Arbitrary-Order Noise Estimator}
\label{sec:estimator_code_example}

While the given equation for an arbitrary-order data-driven noise estimator (Equation \ref{eq:arbitrary_order_noise_estimator}) is mathematically correct, it is nontrivial to computationally implement. This is because an implementation of the math as-written involves combinatorial coefficients that grow exponentially with the order of the estimator, quickly exceeding standard double-precision floating-point overflow. A naive implementation also has a runtime complexity that scales linearly with both with the number of samples in the dataset and the order of the estimator, making it computationally unattractive for large datasets or high-order estimators.

An efficient computational implementation of the arbitrary-order noise estimator is given in Listing \ref{lst:efficient_arbitrary_order_noise_estimator}. This example is based on syntax for NumPy 1.24.3 and SciPy 1.11.1 within Python 3. This implementation uses the log-gamma function, which is much faster for high-order estimators and delays numerical overflow until much higher orders. The implementation also uses a convolutional kernel to vectorize the summation, enabling further speedups.

One possible way to verify the correctness of the code in Listing \ref{lst:efficient_arbitrary_order_noise_estimator} is to generate a synthetic signal (e.g., a sampled sine wave), add some normally-distributed, uncorrelated, homoscedastic noise to it, and then use this code to reconstruct the standard deviation of this noise from the dirty signal.

\begin{listing}[H]
    \begin{minted}[mathescape=true]{python}
import numpy as np
from scipy.special import gammaln

ln_factorial = lambda x: gammaln(x + 1)  # Shorthand for the function $f(x) = \ln(x!)$

def estimate_noise_standard_deviation(
        data: np.ndarray,
        estimator_order: int = 10
    ) -> float:
    """
    Estimates the standard deviation of noise in a time-series dataset, where the dataset
    is the additive summation of some meaningful signal and some noise. Assumes that this
    noise is both stationary and white (in other words, i.i.d. across samples).

    Args:

        data: A 1D NumPy array of time-series data, assumed to consist of signal + noise.

        estimator_order: The order of the estimator to use. A positive integer. Higher
            values are better at spectrally-separating signal from noise, especially
            when the signal has relatively high-frequency components compared to the
            sample rate.

    Returns: The estimated standard deviation of the noise in the data.
    """
    # For speed, cache $\ln(x!)$ for $x \in [$0, estimator_order$]$
    ln_f = ln_factorial(np.arange(estimator_order + 1))

    # Create a convolutional kernel to vectorize the summation
    coefficients = np.exp(
        2 * ln_f[estimator_order] - ln_f - ln_f[::-1] - 0.5 * ln_factorial(2 * estimator_order)
    ) * (-1) ** np.arange(estimator_order + 1)

    # Remove any bias introduced by floating-point error
    coefficients -= np.mean(coefficients)

    # Convolve the data with the kernel
    sample_stdev = np.convolve(data, coefficients[::-1], 'valid')

    # Return the standard deviation of the result
    return np.mean(sample_stdev ** 2) ** 0.5
    \end{minted}
    \caption{Example efficient implementation of the arbitrary-order noise estimator using NumPy/SciPy in Python 3.}
    \label{lst:efficient_arbitrary_order_noise_estimator}
\end{listing}
