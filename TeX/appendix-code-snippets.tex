\chapter{AeroSandbox Code Examples for Benchmarks}
\label{chap:code}


\section{Benchmark on Beam Static Structural Analysis}

This Python code is used for the AeroSandbox performance benchmark used in Section \ref{sec:benchmark_gpkit}, with associated runtimes shown as the line labeled ``AeroSandbox'' in Figure \ref{fig:benchmark_gp_beam}. In this analysis problem, a cantilever beam is subject to a distributed load, and the goal is to compute the state of the deflected beam. The code is shown in Listing \ref{lst:gpkit_beam}.

\begin{listing}[h]
    \begin{minted}{python}
import aerosandbox as asb
import aerosandbox.numpy as np

N = 50  # Number of discretization nodes
L = 6  # Overall length of the beam [m]
EI = 1.1e4  # Bending stiffness [N*m^2]
q = 110 * np.ones(N)  # Distributed load [N/m]

x = np.linspace(0, L, N)  # Node locations along beam length [m]

opti = asb.Opti()  # Initialize an optimization environment

w = opti.variable(init_guess=np.zeros(N))  # Displacement [m]

th = opti.derivative_of(  # Slope [rad]
    w, with_respect_to=x,
    derivative_init_guess=np.zeros(N),
)

M = opti.derivative_of(  # Moment [N*m]
    th * EI, with_respect_to=x,
    derivative_init_guess=np.zeros(N),
)

V = opti.derivative_of(  # Shear force [N]
    M, with_respect_to=x,
    derivative_init_guess=np.zeros(N),
)

opti.constrain_derivative(  # Shear integration
    variable=V, with_respect_to=x,
    derivative=q,
)

opti.subject_to([  # Boundary conditions
    w[0] == 0,
    th[0] == 0,
    M[-1] == 0,
    V[-1] == 0,
])

sol = opti.solve()

print(sol(w[-1]))  # Prints the tip deflection; should be 1.62 m.
    \end{minted}
    \caption{AeroSandbox code for a static structural analysis of a beam. Written in Python.}
    \label{lst:gpkit_beam}
\end{listing}
