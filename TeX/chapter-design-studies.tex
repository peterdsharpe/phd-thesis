\chapter{Aircraft Design Case Studies}

As indicated in Table \ref{tab:paradigm_comparison}, the main benefit of the proposed code transformations paradigm is to achieve runtime performance comparable to specialized methods, but without sacrificing ease-of-use and mathematical flexibility. Because this value proposition hinges on this usability, it becomes especially important to demonstrate the effectiveness of the proposed paradigm in real-world engineering applications.

In order to support this goal, Chapter \ref{chapter-code-transformations} introduced an example framework, AeroSandbox, which implements the proposed code transformations paradigm. AeroSandbox was developed using an iterative ``spiral development process'', which is depicted in Figure \ref{fig:spiral}. In short, at every step of the development process, a symbiotic bidirectional relationship between the framework and the case studies was maintained; the goal of this was to simultaneously ask and answer two questions:

\begin{enumerate}[noitemsep]
    \item Using the capabilities of this design framework, how can we improve this airplane and gain practical insight into the specific design space at hand?
    \item Based on this airplane case study, what broadly-applicable user needs and workflows can we identify, and how can we improve our design framework accordingly?
\end{enumerate}

\begin{figure}[H]
    \centering
    \includesvg[width=0.8\textwidth]{../figures/spiral.svg}
    \caption{The iterative spiral development process used to develop AeroSandbox, where applied use and framework development are intertwined. The aircraft in the figure is a visualization output using the AeroSandbox geometry stack, and depicts the hydrogen-fueled aircraft from the case study of Section \ref{sec:hydrogen}.}
    \label{fig:spiral}
\end{figure}

Over the development history of AeroSandbox, this spiral process was iterated upon multiple times, with each iteration resulting in a more refined and capable framework. This chapter presents a few vignettes of the various case studies that were used to develop AeroSandbox, which provides a fruitful opportunity to discuss the practical capabilities enabled by a code transformations MDO paradigm. It is important to emphasize that, although the aircraft design case studies presented in this chapter are fascinating real-world problems, the main intended intellectual contribution of this portion of the thesis is not the design of these specific aircraft (or even the example framework itself\footnote{MDO frameworks naturally come and go over the years as new opportunities in scientific computing and programming languages emerge. It is not the author's intent or expectation that AeroSandbox be the ``end-all'' framework for aircraft design. Rather, it is a proof-of-concept implementation that aims to demonstrate the potential utility of one possible framework approach for readers interested in building a future framework.}). Instead, the deeper goal is to introduce a principled paradigm for building engineering design optimization frameworks. Because of this, complete detailed enumeration of all modeling assumptions for specific problems is omitted in this chapter for brevity; however, both references to literature discussing these details and links to the raw source code itself are provided for most case studies, for the interested reader.


\section{Simple Aircraft Design Problem}
\label{sec:simpleac}

\begin{quote}
    \emph{This section includes content adapted from the author's prior work} \cite{sharpe_aerosandbox_2021}.
\end{quote}

A first simple aircraft design problem, called \emph{SimpleAC}, is included here to show what design code might look like in code transformations framework at the early stages of conceptual aircraft sizing. At this ``napkin-math'' stage, the main analyses (e.g., aerodynamics, structures, propulsion) can be cleanly expressed in simple analytical expressions, and the optimizer fulfills the role of closing the sizing loop. Because of this, this problem is simple enough to include both the complete problem statement and the complete solution code directly here, which is useful for readers interested in observing the mapping between these. This design problem itself was proposed by Hoburg \cite{hoburg} and is reproduced (with slight modifications) by both Ozturk \cite{Ozturk2018} and Kirschen \cite{kirschen}. It is restated here in full form:

\begin{example}
    \textbf{Simple Aircraft (SimpleAC)}
    \begin{mini}
        |l|
            {\AR, S, V, W, C_L, W_f, V_\text{f, fuse}}{W_f}
            {}{}
        \addConstraint{W \geq W_0 + W_w + W_f}
        \addConstraint{W_0 + W_w + \frac{1}{2} W_f \leq L_\text{cruise}}
        \addConstraint{W \leq L_\text{takeoff}}
        \addConstraint{W_f \geq \text{TSFC} \cdot t_\text{flight} \cdot D}
        \addConstraint{V_\text{f, wing} + V_\text{f, fuse} \geq V_f}
        \label{eq:simpleac}
    \end{mini}
    \begin{eqexpl}
        \item{$D$} Cruise drag
        \item{$\AR$} Wing aspect ratio (here, the wing is assumed to be rectangular)
        \item{$S$} Wing area
        \item{$V$} Cruise airspeed
        \item{$W$} Total weight
        \item{$C_L$} Cruise lift coefficient
        \item{$W_f$} Fuel weight
        \item{$V_\text{f, fuse}$} Volume of fuel in fuselage
    \end{eqexpl}

    \noindent
    We are also given the following physics models:

    \begin{itemize}[noitemsep]
        \item The chord $c = \sqrt{S / \AR}$, from geometric relations.
        \item The drag $D = \frac{1}{2} \rho V^2 C_D S$
        \item The drag coefficient $C_D = \frac{\mathrm{CDA}_0}{S} + k C_f \frac{S_\text{wet}}{S} + \frac{C_L^2}{\pi \AR e}$
        \begin{itemize}[noitemsep]
            \item $C_f = 0.074 \cdot \text{Re}^{-0.2}$, the Schlichting turbulent flat plate boundary layer model
            \item $\text{Re} = \frac{\rho V c}{\mu}$
        \end{itemize}
        \item The cruise lift $L_\text{cruise}=\frac{1}{2}\rho V^2 C_L S$
        \item The takeoff lift $L_\text{takeoff}=\frac{1}{2}\rho V_\text{min}^2 C_{L, \text{max}} S$
        \item The wing weight $W_{\text{wing}}= W_\text{w, structural} + W_\text{w, surface}$
        \begin{itemize}[noitemsep]
            \item $W_\mathrm{w,\ structural} = W_\text{w, c1} \cdot \frac{N \AR^{1.5} \sqrt{W_0 W S}}{\tau}$
            \item $W_\mathrm{w,\ surface} = W_\text{w, c2} \cdot S$
        \end{itemize}
        \item The aircraft's endurance $t_\text{flight} = \text{Range} / V$
        \item The fuselage drag area scales with fuel volume as $\text{CDA}_0 = V_\text{f, fuse} / (10\ \si{\meter})$
        \item The total fuel volume $V_f = \frac{W_f}{g\rho_f}$
        \item The fuel volume in the wing $V_\text{f, wing} = 0.03 S^{1.5} \AR^{-0.5} \tau$
    \end{itemize}

    \begin{eqexpl}
        \item {$g$} $9.81\ \si{\meter/\second\squared}$, Earth gravity.
        \item {$\rho_f$} $817\ \si{\kg/\meter\cubed}$, the density of fuel.
        \item {$\text{Range}$} $1000\ \si{\kilo\meter}$, the aircraft mission range.
        \item {$\text{TSFC}$} $0.6\ \si{\per\hour} = 1.67 \times 10^{-4}\ \si{\per\second}$, the thrust-specific fuel consumption.
        \item {$k$} $1.17$, the form factor.
        \item {$e$} $0.92$, the Oswald efficiency factor.
        \item {$\mu$} $1.775 \times 10^{-5}$ \si{\kg\per\meter\per\second}, the sea-level dynamic viscosity of air.
        \item {$\rho$} $1.23\ \si{\kg/\meter\cubed}$, the sea-level density of air.
        \item {$\tau$} $0.12$, the airfoil thickness-to-chord ratio.
        \item {$N$} $3.3$, the ultimate load factor.
        \item {$V_\text{min}$} $25\ \si{\meter/\second}$, the takeoff airspeed.
        \item {$C_{L, \text{max}}$} $1.6$, the takeoff lift coefficient.
        \item {$S_\text{wet}/S$} $2.075$, the wetted area ratio.
        \item {$W_0$} $6250 \si{\newton}$, the aircraft weight excluding the wing and fuel.
        \item {$W_\text{w, c1}$} $2 \times 10^{-5}\ \si{\per\meter}$, a wing weight coefficient.
        \item {$W_\text{w, c2}$} $60\ \si{\Pa}$, another wing weight coefficient.
    \end{eqexpl}

\end{example}

This design problem can be translated from the formulation above into the solution code, which is given in Appendix \ref{sec:simpleac-code}. Notably, the code of this problem is essentially a 1:1 translation of the problem statement into Python code, with almost zero additional boilerplate code required to set up this problem beyond the raw physics models and constants definitions. In fact, the solution code corresponding to this problem is shorter than the natural-language problem statement itself. This conciseness is a good approximate measure of the framework's usability, since it allows the user to focus on the problem physics and mathematical formulation, rather than the numerical mechanics of the optimization process.

The code given in Appendix \ref{sec:simpleac-code} converges to a solution in 14 iterations, with a median runtime\footnote{Measured on a laptop a Ryzen 7 5800H CPU. Measures end-to-end wall-clock runtime, including problem formulation (i.e., tracing, in Python), optimization (via IPOPT, in C++), and function evaluation (i.e., on the CasADi VM, in C++).} of 18 milliseconds. The results of this AeroSandbox solve are shown in Table \ref{tab:simpleac}.

\begin{table}[H]
    \centering
    \caption{Solution of SimpleAC (Eq. \ref{eq:simpleac}), found with AeroSandbox.}
    \label{tab:simpleac}
    \begin{tabular}[t]{ll}
        \toprule
        Figure of Merit                            & Optimal Value             \\
        \midrule
        Fuel weight $W_f$                          & 937.8 \si{\newton}        \\
        Aspect ratio $\AR$                         & 12.10                     \\
        Wing area $S$                              & 14.15 \si{\meter\squared} \\
        Cruise airspeed $V$                        & 57.11 \si{\meter/\second} \\
        All-up weight $W$                          & 8705 \si{\newton}         \\
        Cruise lift coefficient $C_L$              & 0.2901                    \\
        Fuel volume in fuselage $V_\text{f, fuse}$ & 0.0619 \si{\meter\cubed}  \\
        \bottomrule
    \end{tabular}
\end{table}



\section{Firefly} % TODO name
\label{sec:firefly}


\section{Dawn} % TODO name
\label{sec:dawn}


\section{Hydrogen} % TODO name
\label{sec:hydrogen}


\section{Solar Seaplane} % TODO name
\label{sec:solar-seaplane}


%\section{Other Case Studies} % TODO name
