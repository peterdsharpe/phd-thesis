\chapter{Introduction}
% TODO review entire chapter
\label{chap:intro}

The promise of multidisciplinary design optimization (MDO) to revolutionize engineering design has motivated pioneering research into new frameworks and methods for decades. MDO saw rapid progress beginning in the late 1980s, with early works outlining visions for integrated computational design tools that couple analyses across engineering disciplines \cite{ashley_making_1982, vanderplaats_automated_1976, haftka_multidisciplinary_1997}. However, contemporaneous warnings also highlighted the limited adoption of these methods by industry practitioners up to that point \cite{kroo_multidisciplinary_1997, drela_pros_1998}.

Surveys assessing the state of aircraft MDO years later reveal a persisting gap between MDO’s promise and practice. Documented examples of complete aircraft designed with MDO remain relatively uncommon in industry, even as exponential growth in computational power has mitigated other technical barriers. Expert commentary affirms this observation, noting that the use of advanced optimization techniques often stops at the academic proof-of-concept stage \cite{mcmasters_airplane_2002, kroo_multidisciplinary_1997, agte_mdo_2010, ashley_making_1982, haftka_multidisciplinary_1997,gazaix_industrialization_2017}.

The core remaining barriers to widespread adoption of conceptual-level MDO relate strongly to practical limitations at the human-optimizer interface, rather than purely on traditional metrics of computational speed \cite{gpkit}. The complexity intrinsic to coordinating many coupled models across disciplines poses inherent difficulties for design interpretation, method credibility, and managing tradeoffs \cite{salas_framework_1998}. Modern scientific computing capabilities enable unprecedented problem scale, but push human limitations in complexity management. The central difficulty is leveraging advanced techniques in scientific computing -- which offer enormous performance improvements -- without requiring end-users to be joint experts in applied mathematics and computer science on top of their engineering domain expertise \cite{ma_modelingtoolkit_2021}.

The path forward lies in a deliberate focus on the practical human interface with optimization tools. Technical advances must synthesize with pragmatic methods (e.g., syntax) to address open challenges around complexity and usability. The overarching motivation is to address this industry gap, allowing the benefits of advanced conceptual design optimization to be more readily accessed by industry. This grounds the proposed research directions on fundamentally human-centered principles in addition to technical ones.

%    Before defining specific contributions, we first survey key developments in MDO algorithms, architectures, and strategies over recent decades. Reviewing progress to date allows us to contextualize remaining barriers to real-world impact. We ultimately find that formidable capability now exists, but practical open problems persist nearly unchanged since the genesis of MDO research.


\section{Project Definition and Thesis Overview}
\label{sec:definition}

To address the above challenges, this thesis proposes five novel contributions:

\begin{enumerate}
    \item \textbf{Code Transformation Paradigm} (Chapters \ref{chap:code_transformations} and \ref{chap:design-studies}): Conceptually introduces a computational paradigm for MDO frameworks that offers most of the benefits of existing state-of-the-art paradigms with fewer user frictions. Provides a proof-of-concept implementation of this paradigm in the open-source \textit{AeroSandbox} \cite{sharpe_aerosandbox_2021} framework, and compares this implementation with existing frameworks on a set of practical aircraft design problems.
    \item \textbf{Traceable Physics Models} (Chapters \ref{chap:design-studies} and \ref{chap:physics}): Provides the first implementations of several key aerospace physics models that are purpose-built to take advantage of code transformations. These aim to both stress-test the code transformation paradigm on common analysis code patterns and jump-start future applied research with a set of modular analyses.
    \item \textbf{Sparsity Tracing via NaN-Propagation} (Chapter \ref{chap:nan_propagation}): Conceptually introduces the novel idea of ``NaN-propagation'' as a method to trace sparsity through black-box numerical analyses by exploiting floating-point math handling. This opens a path to including such models in a code transformations framework while still retaining some (but not all) of the speed advantages over current black-box optimization methods.
    \item \textbf{Physics-Informed Machine Learning Surrogates for Black-Box Models} (Chapter \ref{chap:physics-informed-ml}): Explores strategies to incorporate black-box models into a code transformation framework, expanding the practicality of the paradigm for users who require use of black-box code. As an example, a physics-informed machine learning surrogate model for airfoil aerodynamics analysis will be presented. This demonstrates that accurate surrogates can be constructed to stand in for complex analyses while retaining compatibility with code transformations, and without sacrificing mathematical flexibility.
    \item \textbf{Aircraft System Identification from Minimal Sensor Data} (Chapter \ref{chap:aircraft_sysid}): Demonstrates that the code transformation paradigm enables straightforward formulation of optimization problems beyond just design optimization. As an example, the thesis will show an application to aircraft system identification and performance reconstruction from minimal flight data. Here, physics-based corrections are used alongside statistical inference techniques to accurately estimate aircraft performance characteristics from a single short test flight. This approach is enabled by the flexibility of the code transformation framework.
%        \item Build a computational framework for engineering design optimization based on the paradigm of \textit{code transformations}, which enables the use of modern techniques in computer science and applied math (detailed more in Section). This framework will act as a ``proving ground'' on top of which subsequent research objectives will be implemented and evaluated.
%        \item Demonstrate that this \textit{code transformation} paradigm enables the formulation of large-scale engineering design optimization problems at the conceptual stage, to an extent that is not otherwise possible with existing public aircraft design frameworks. To satisfy this objective, the framework will demonstrate (on at least one applied aircraft design case study) performance equaling-or-exceeding the state-of-the-art across the following metrics:
%        \begin{itemize}
%            \item Runtime speed (i.e., scalability of computational resources, and runtimes compatible with human-in-the-loop interactive design).
%            \item Problem implementation and re-implementation speed (i.e., scalability of engineering time resources).
%            \item Mathematical flexibility: the ability to achieve the above metrics without imposing restrictions on user models that can force undue deviations from physical reality (e.g., log-convexity).
%        \end{itemize}
%        \item Demonstrate the double-edged sword of the large-scale conceptual design optimization capabilities enabled by this framework:
%        \begin{enumerate}
%            \item \textbf{Benefit:} Demonstrate that enabling large-scale engineering design at the conceptual stage offers measurable performance gains in the resulting aircraft designs compared to those produced with existing methods. Specifically, the thesis will present at least one applied aircraft design case study, which is then solved with two separate approaches:
%            \begin{enumerate}
%                \item A first-order sizing study where an aircraft is designed by coupling only ``core'' aircraft design disciplines (aerodynamics, structures, and propulsion) and traditional sizing relations (e.g., $W/S$ and $T/W$ diagrams)
%                \item A higher-order sizing study where more ``non-core'' disciplines (e.g., stability and control, trajectory and mission optimization, cost analysis, field lengths, noise, manufacturability) are added to the aircraft design problem. Geometric flexibility, as measured by the number of design variables describing the aircraft geometry, will also be increased.
%            \end{enumerate}
%            The resulting designs from these two studies will then be compared (post-optimality) to assess their performance on both stated objectives (to assess how much large-scale modeling expands the feasible design space) and secondary metrics (to assess how well core disciplines act as a surrogate for non-core disciplines).
%            \item \textbf{Risk:} Demonstrate the major pitfall of the new large-scale conceptual design optimization: the lack of result \textit{interpretability}. Show that the system complexity enabled by large-scale optimization leads to results are more difficult to communicate, interpret, review, and trust.
%        \end{enumerate}
%        \item
%        \item Identify framework features and characteristics that aid in the \textit{interpretability} and \textit{reviewability} of large-scale engineering design optimization problems. This should be based on a combination of literature review, aircraft design case studies, and (as resources allow) discussions with framework users and aircraft designers. Fundamentally, this is more a communication question than a technical one: setting up a large-scale optimization problem is doable with state-of-the-art aircraft design optimization tools, but (``how do we communicate whether we can trust the results of a design optimization study?'').
%        \item Implement some of these practical This should include, at minimum:
%        \begin{itemize}
%            \item A constraint activity log, which
%        \end{itemize}
\end{enumerate}

