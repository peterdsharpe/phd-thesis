\chapter{Introduction}
% TODO review entire chapter
\label{chap:intro}

%The design of complex engineering systems—aircraft, spacecraft, automobiles, and more -- is a challenging and multi-faceted task.

Our everyday lives are filled with interactions with complex engineering systems, from the cars we drive, to the electricity grid that powers our cities, to the aircraft we fly in. With every passing year, the design requirements that we place on these systems increase in scope. This inexorable progress towards ``doing more with less'' has brought staggering benefits in performance, cost, safety, and environmental impact: as just one example, the fuel economy\footnote{Measured as miles per gallon for cars, and fuel burn per passenger-mile for aircraft} of both consumer cars and commercial aircraft has roughly tripled over the past 60 years \cite{EPA_Automotive_Trends_2023, drela_tedx}.

This progress, however, often comes at the cost of increased complexity in the engineering design process. Satisfying these requirements often involves adding new subsystems or considering new analyses. As the number of these subsystems and analyses grows, the interactions between them also become more numerous, important, and non-intuitive. This presents both challenge and opportunity. The challenge is that, even at the earliest conceptual design stages, the designer must juggle a far wider problem to even begin to meaningfully understand and improve the system's performance. The opportunity is that, by understanding and exploiting these subsystem interactions, the designer can unlock vast new design spaces that were previously inaccessible \cite{drela_design_2011}.

Fortunately, computational techniques have evolved to meet the challenge of managing this ever-increasing complexity and navigating these new design spaces in a human-understandable way. These techniques go by many names in various fields, but in the context of aerospace engineering, they are often referred to as \emph{Multidisciplinary Design Optimization} (MDO) frameworks \cite{martins_multidisciplinary_2013}. Under the umbrella of these \emph{frameworks}, an end-user can implement a problem-specific \emph{design tool} that can be used to explore the design space of a complex engineering system.

The recognized potential of MDO frameworks to unlock the next generation of performance improvements has motivated a great deal of research in the field. MDO saw rapid progress beginning in the late 1980s, with early works building the mathematical and computer science basis for integrated computational design tools that couple and orchestrate analyses across engineering disciplines \cite{ashley_making_1982, vanderplaats_automated_1976, haftka_multidisciplinary_1997}. Even from these early days, it was clear that \emph{industrial} adoption of these tools was not straightforward \cite{kroo_multidisciplinary_1997, drela_pros_1998}. Often, reasons were not purely technical: the complexity of integrating models into MDO tools, the difficulties of quickly formulating and iterating on design problems, and the fragility of resulting designs in the absence of careful modeling all presented significant practical barriers.

Significant progress in the development of MDO frameworks has been made in the decades since. However, surveys assessing the state of aircraft MDO years later still reveal a persisting gap between MDO’s potential and practice \cite{agte_mdo_2010, drela_design_2011}. Despite large advances in computational power, advanced optimization techniques, and MDO architectures, many of these benefits have yet to be fully realized in industry. Documented examples of complete built-and-flown aircraft designed with MDO exist, but remain less common than intuition might suggest, even as many technical barriers have fallen to the exponential rise in computational power in recent years \cite{gazaix_industrialization_2017}.

The core remaining barriers to widespread adoption of conceptual-level MDO relate strongly to practical limitations at the human-optimizer interface, rather than purely on traditional metrics of computational speed \cite{gpkit}. The complexity intrinsic to coordinating many coupled models across disciplines poses inherent difficulties for design interpretation, method credibility, and managing tradeoffs \cite{salas_framework_1998}. Modern scientific computing capabilities enable unprecedented problem scale, but push human limitations in complexity management. As we develop the next generation of MDO frameworks, the central difficulty is leveraging advanced techniques in scientific computing -- which offer enormous performance improvements -- without requiring end-users to be joint experts in applied mathematics and computer science on top of their engineering domain expertise \cite{ma_modelingtoolkit_2021}.

The path forward lies in a deliberate focus on the practical human interface with optimization tools, from the perspective of an industrial end-user. Technical advances must synthesize with pragmatic methods (e.g., code syntax and style) to address open challenges around complexity and usability. This grounds the proposed research directions on fundamentally human-centered principles in addition to technical ones.

\section{Project Definition and Thesis Overview}
\label{sec:definition}

This thesis addresses these challenges by proposing five novel contributions that collectively map out a path towards improved usability and effectiveness of MDO frameworks:

\begin{enumerate}
    \item \textbf{Code Transformation Paradigm} (Chapter \ref{chap:code_transformations}): First, this thesis conceptually introduces a new computational paradigm for MDO frameworks based on \emph{code transformations}, a related collection of modern scientific computing techniques. The value proposition of this paradigm is that it gives end-users most of the benefits of existing state-of-the-art paradigms, but with fewer of the practical user frictions that impede industry use. To demonstrate this, this thesis provides a proof-of-concept implementation of this paradigm in the open-source \textit{AeroSandbox} \cite{asb_github, sharpe_aerosandbox_2021} framework. In this thesis, this framework is benchmarked against existing frameworks on both technical and non-technical metrics to evaluate the aforementioned value proposition.
    \begin{itemize}
    \item \textbf{Aircraft Design Case Studies} (Chapter \ref{chap:design-studies}): The developed framework is then applied to a series of aircraft design problems, many of which supported real-world aircraft development programs. This facilitates more precise discussion about the possible opportunities and challenges of this proposed framework-level MDO approach. These case studies also provide more realistic performance benchmarks, as well as starting points for future applied research.
    \end{itemize}
    \item \textbf{Traceable Physics Models} (Chapter \ref{chap:physics}): This chapter provides implementations of several common aerospace physics models that serve as optional, modular plugins into the example MDO framework of Chapter \ref{chap:code_transformations}. These plugins aim to a) stress-test the code transformation paradigm on common analysis code patterns and b) serve as building blocks to enable end-users to rapidly formulate aircraft design problems.
    \item \textbf{Sparsity Tracing via NaN-Propagation} (Chapter \ref{chap:nan_propagation}): This chapter conceptually introduces the novel idea of ``NaN-propagation'' as a technique to trace sparsity through black-box numerical analyses by exploiting floating-point math handling. This opens a path to including black-box models in a code transformations framework while still retaining some (but not all) of the speed advantages over existing options.
    \item \textbf{Physics-Informed Machine Learning Surrogates for Black-Box Models} (Chapter \ref{chap:physics-informed-ml}): This chapter explores a strategy to incorporate black-box models into a code transformation framework, where a model is replaced with a learned approximator with desirable mathematical properties. As an example, a physics-informed machine learning surrogate model for airfoil aerodynamics analysis will be presented. This demonstrates that accurate surrogates can be constructed to stand in for complex analyses while retaining compatibility with code transformations.
    \item \textbf{Aircraft System Identification from Minimal Sensor Data} (Chapter \ref{chap:aircraft_sysid}): Finally, this chapter demonstrates that a code transformation paradigm enables rapid formulation of optimization problems in applications beyond just design. This can make it attractive to use numerical optimization in contexts where formulation would have otherwise been overly tedious. As an example, the thesis will show an application to aircraft system identification and performance reconstruction from minimal flight data. Here, physics-based corrections are used alongside statistical inference techniques to accurately estimate aircraft performance characteristics from a single short test flight.
\end{enumerate}

