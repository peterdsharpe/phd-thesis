
\title{Title}

\author{Peter Sharpe}
\prevdegrees{B.S., Washington University in St. Louis (2018)}
\prevdegrees{S.M., Massachusetts Institute of Technology (2021)}
% If you wish to list your previous degrees on the cover page, use the 
% previous degrees command:
%       \prevdegrees{A.A., Harvard University (1985)}
% You can use the \\ command to list multiple previous degrees
%       \prevdegrees{B.S., University of California (1978) \\
%                    S.M., Massachusetts Institute of Technology (1981)}
\department{Department of Aeronautics and Astronautics}

% If the thesis is for two degrees simultaneously, list them both
% separated by \and like this:
% \degree{Doctor of Philosophy \and Master of Science}
\degree{Doctor of Philosophy in Aeronautics and Astronautics}

% As of the 2007-08 academic year, valid degree months are September, 
% February, or June.  The default is June.
\degreemonth{MONTH}
\degreeyear{YEAR}
\thesisdate{DATE}

%% By default, the thesis will be copyrighted to MIT.  If you need to copyright
%% the thesis to yourself, just specify the `vi' documentclass option.  If for
%% some reason you want to exactly specify the copyright notice text, you can
%% use the \copyrightnoticetext command.  
%\copyrightnoticetext{\copyright IBM, 1990.  Do not open till Xmas.}

% If there is more than one supervisor, use the \supervisor command
% once for each.
\supervisor{R. John Hansman}{T. Wilson Professor in Aeronautics}

% This is the department committee chairman, not the thesis committee
% chairman.  You should replace this with your Department's Committee
% Chairman.
\chairman{Jonathan How}{Professor, Aeronautics and Astronautics\\Chair, Graduate Program Committee}

% Make the titlepage based on the above information.  If you need
% something special and can't use the standard form, you can specify
% the exact text of the titlepage yourself.  Put it in a titlepage
% environment and leave blank lines where you want vertical space.
% The spaces will be adjusted to fill the entire page.  The dotted
% lines for the signatures are made with the \signature command.
\maketitle

% The abstractpage environment sets up everything on the page except
% the text itself.  The title and other header material are put at the
% top of the page, and the supervisors are listed at the bottom.  A
% new page is begun both before and after.  Of course, an abstract may
% be more than one page itself.  If you need more control over the
% format of the page, you can use the abstract environment, which puts
% the word "Abstract" at the beginning and single spaces its text.

%% You can either \input (*not* \include) your abstract file, or you can put
%% the text of the abstract directly between the \begin{abstractpage} and
%% \end{abstractpage} commands.

% First copy: start a new page, and save the page number.
\cleardoublepage
% Uncomment the next line if you do NOT want a page number on your
% abstract and acknowledgments pages.
% \pagestyle{empty}
\setcounter{savepage}{\thepage}
\begin{abstractpage}
Multidisciplinary design optimization has immense potential to improve conceptual design workflows for large-scale engineered systems, such as aircraft. Despite remarkable theoretical progress in recent decades, however, practical industry adoption of such methods lags far behind. This thesis addresses the root causes of this theory-to-practice gap by introducing a new paradigm for computational design optimization frameworks called \textit{code transformations}. Code transformations encompass a variety of best-practices scientific computing strategies (e.g., automatic differentiation, automatic sparsity detection, problem auto-scaling) that automatically analyze, augment, and accelerate the user's code before passing it to a modern gradient-based optimization algorithm.

This paradigm offers a compelling combination of ease-of-use, computational speed, and modeling flexibility, whereas existing paradigms typically make sacrifices in at least one of these key areas. Consequently, code transformations present a competitive avenue for increasing the adoption of advanced optimization techniques in industry, all without requiring significant expertise in applied mathematics or computer science from end users.

The major contributions of this thesis are fivefold. First, it introduces the concept of code transformations as a possible foundation for an MDO framework and demonstrates their feasibility through aircraft design case studies. Second, it implements several common aircraft analyses in a form compatible with code transformations, providing a practical illustration of the opportunities, challenges, and considerations here. Third, it presents a novel technique to automatically trace sparsity through certain external black-box functions by exploiting IEEE 754 handling of not-a-number (NaN) values. Fourth, it proposes strategies for efficiently incorporating black-box models into a code transformation framework through physics-informed machine learning surrogates, demonstrated with an airfoil aerodynamics analysis case study. Finally, it shows how a code transformations paradigm can simplify the formulation of other optimization-related aerospace tasks outside of design, exemplified by aircraft system identification and performance reconstruction from minimal flight data.

Taken holistically, these contributions aim to improve the accessibility of advanced optimization techniques for industry engineers, making large-scale conceptual multidisciplinary design optimization more practical for real-world systems.
\end{abstractpage}

% Additional copy: start a new page, and reset the page number.  This way,
% the second copy of the abstract is not counted as separate pages.
% Uncomment the next 6 lines if you need two copies of the abstract
% page.
% \setcounter{page}{\thesavepage}
% \begin{abstractpage}
% Multidisciplinary design optimization has immense potential to improve conceptual design workflows for large-scale engineered systems, such as aircraft. Despite remarkable theoretical progress in recent decades, however, practical industry adoption of such methods lags far behind. This thesis addresses the root causes of this theory-to-practice gap by introducing a new paradigm for computational design optimization frameworks called \textit{code transformations}. Code transformations encompass a variety of best-practices scientific computing strategies (e.g., automatic differentiation, automatic sparsity detection, problem auto-scaling) that automatically analyze, augment, and accelerate the user's code before passing it to a modern gradient-based optimization algorithm.

This paradigm offers a compelling combination of ease-of-use, computational speed, and modeling flexibility, whereas existing paradigms typically make sacrifices in at least one of these key areas. Consequently, code transformations present a competitive avenue for increasing the adoption of advanced optimization techniques in industry, all without requiring significant expertise in applied mathematics or computer science from end users.

The major contributions of this thesis are fivefold. First, it introduces the concept of code transformations as a possible foundation for an MDO framework and demonstrates their feasibility through aircraft design case studies. Second, it implements several common aircraft analyses in a form compatible with code transformations, providing a practical illustration of the opportunities, challenges, and considerations here. Third, it presents a novel technique to automatically trace sparsity through certain external black-box functions by exploiting IEEE 754 handling of not-a-number (NaN) values. Fourth, it proposes strategies for efficiently incorporating black-box models into a code transformation framework through physics-informed machine learning surrogates, demonstrated with an airfoil aerodynamics analysis case study. Finally, it shows how a code transformations paradigm can simplify the formulation of other optimization-related aerospace tasks outside of design, exemplified by aircraft system identification and performance reconstruction from minimal flight data.

Taken holistically, these contributions aim to improve the accessibility of advanced optimization techniques for industry engineers, making large-scale conceptual multidisciplinary design optimization more practical for real-world systems.
% \end{abstractpage}

\cleardoublepage

\section*{Acknowledgments}

This thesis and my associated research endeavors would not have been possible without the support of so many people, and for that I am deeply thankful.

In particular, I would like to thank my advisor, Prof. John Hansman, for his unwavering support and guidance throughout my time at MIT. I'm also indebted to Prof. Mark Drela, both for his personal mentorship and teaching, as well as his broader professional contributions -- much of the work in this thesis would not have been possible without his prolific contributions to the field of aerospace engineering. I would also like to thank Prof. Karen Willcox, who graciously agreed to serve on my doctoral committee and has provided wise and valuable insights throughout our meetings. I am also grateful for Dr. Tony Tao and Prof. Joaquim Martins, both for serving as readers on this thesis and for their contributions to the field.

I would like to thank the open-source scientific computing community for their contributions to this work. In particular, I'd like to thank the maintainers of NumPy, SciPy, CasADi, JAX, IPOPT, and the Python programming language for their tireless work on these tools that have enabled so much of the research in this thesis. The modern day open-source software ecosystem epitomizes the spirit of worldwide collaboration that Newton described so long ago: ``If I have seen further, it is by standing on the shoulders of giants.'' It is a privilege to contribute to this community.

Various parts of the research in this thesis were funded by the MIT Portugal Program, BAE Systems, the National Defense Science and Engineering Graduate Fellowship, and I thank them for their support. I would also like to thank the MIT Supercloud team for supporting my free access to supercomputing resources that enabled parts of this research.

At MIT AeroAstro, I would like to thank my labmates in ICAT for their camraderie and support over the years, especially Matt Vernacchia, Kelly Mathesius, Julia Gaubatz, Annick Dewald, Ara Mahseridjian, Estelle Martin, Trevor Long, Juju Wang, Sandro Salgueiro, Marek Travnik, Mina Cezairli, and many others. I've been so fortunate to spend the past few years alongside such a talented and friendly group of people.

Also at AeroAstro, I'd like to thank machine shop staff Todd Billings and Dave Robertson for their friendship and help over the years. When you've been stuck on a hard math problem for days, sometimes there's nothing like cutting some balsa and building a model airplane to clear your head.

Closer to home, I'd like to thank my friends and family for their support and encouragement -- it means the world to me. Most of all, I'd like to thank my partner and best friend, Marta Manzin, for her seemingly infinite love and support throughout this journey. She has been my rock through the ups and downs of graduate school. I cannot imagine having done this without you.

%%%%%%%%%%%%%%%%%%%%%%%%%%%%%%%%%%%%%%%%%%%%%%%%%%%%%%%%%%%%%%%%%%%%%%
% -*-latex-*-
