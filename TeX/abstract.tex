Multidisciplinary design optimization has immense potential to improve conceptual design workflows for large-scale engineered systems, such as aircraft. Despite remarkable theoretical progress in recent decades, however, practical industry adoption of such methods lags far behind. This thesis addresses the root causes of this theory-to-practice gap by introducing a new paradigm for computational design optimization frameworks called \textit{code transformations}. Code transformations encompass a variety of best-practices scientific computing strategies (e.g., automatic differentiation, automatic sparsity detection, problem auto-scaling) that automatically analyze, augment, and accelerate the user's code before passing it to a modern gradient-based optimization algorithm.

This paradigm offers a compelling combination of ease-of-use, computational speed, and modeling flexibility, whereas existing paradigms typically make sacrifices in at least one of these key areas. Consequently, code transformations present a competitive avenue for increasing the adoption of advanced optimization techniques in industry, all without requiring significant expertise in applied mathematics or computer science from end users.

The major contributions of this thesis are fivefold. First, it introduces the concept of code transformations as a possible foundation for an MDO framework and demonstrates their feasibility through aircraft design case studies. Second, it implements several common aircraft analyses in a form compatible with code transformations, providing a practical illustration of the opportunities, challenges, and considerations here. Third, it presents a novel technique to automatically trace sparsity through certain external black-box functions by exploiting IEEE 754 handling of not-a-number (NaN) values. Fourth, it proposes strategies for efficiently incorporating black-box models into a code transformation framework through physics-informed machine learning surrogates, demonstrated with an airfoil aerodynamics analysis case study. Finally, it shows how a code transformations paradigm can simplify the formulation of other optimization-related aerospace tasks outside of design, exemplified by aircraft system identification and performance reconstruction from minimal flight data.

Taken holistically, these contributions aim to improve the accessibility of advanced optimization techniques for industry engineers, making large-scale conceptual multidisciplinary design optimization more practical for real-world systems.