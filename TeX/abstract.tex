Design optimization has immense potential to improve aircraft conceptual design, yet practical industry adoption of advanced methods remains relatively limited despite decades of academic research. This thesis takes steps to address the root causes of this gap by introducing a new paradigm for computational design optimization frameworks called \textit{code transformations}. Code transformations encompass a variety of best-practices scientific computing strategies (e.g., automatic differentiation, automatic sparsity detection, problem auto-scaling, etc.) that automatically analyze, augment, and accelerate the user's code before passing it to a modern gradient-based optimization algorithm.

This paradigm offers a compelling combination of ease-of-use, computational speed, and modeling flexibility; existing paradigms typically make sacrifices in at least one of these key areas. Because of this, code transformations offer a competitive avenue for increasing the adoption of advanced optimization techniques in industry, all without requiring significant expertise in applied mathematics or computer science from end users.

% TODO proofread this paragraph
The major contributions of this thesis are fivefold. First, this thesis introduces code transformations conceptually and demonstrates their feasibility through aircraft design case studies. Second, several common aircraft analyses are implemented in a traceable form compatible with code transformations, which provides a practical illustration of the opportunities, challenges, and considerations. Third is a novel technique to automatically trace sparsity through some kinds of external black-box functions by exploiting IEEE 754 handling of not-a-number (NaN) values. Fourth, strategies for efficiently incorporating black-box models into a code transformation framework through physics-informed machine learning surrogates are proposed; this will be demonstrated with an airfoil aerodynamics analysis case study. Finally, the thesis will show how a code transformations paradigm can simplify the formulation of other optimization-related aerospace tasks outside of design; here, an example of aircraft system identification and performance reconstruction from minimal flight data will be given. Taken holistically, these contributions aim to improve the accessibility of advanced optimization techniques for industry engineers, making large-scale conceptual multidisciplinary design optimization more practical for real-world systems.
