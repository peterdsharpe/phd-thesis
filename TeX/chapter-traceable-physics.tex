\chapter{Traceable Physics Models}
\label{chap:physics}

This thesis also contributes computational implementations of several key physics models for aircraft design that are traceable within the example code transformations framework introduced in Section \ref{sec:aerosandbox}, AeroSandbox. The broad motivation for this contribution stems from the observation that ``ease-of-model-implementation'' has historically proven to be one of the most important factors for determining whether an MDO paradigm can achieve use in industry. More specifically, the goals of this contribution are to:

\begin{enumerate}
    \item Stress-test the feasibility of code transformations in practice—how much added user effort and expertise is required to bring typical engineering analyses into a code-transformations-based MDO tool? To what extent can existing code be used as-is? Finally, the thesis aims to identify any specific computational elements that cause ``pain points'' when attempting to make an analysis traceable.
    \item Jump-start future applied research by providing a set of modular, plug-and-play analyses that can be used to quickly build a variety of aircraft design optimization problems. Since the long-term goal of this research direction is to establish whether the proposed MDO paradigm improves practicality, many practical aircraft design problems must be posed. Creating a set of modular general-purpose building blocks reduces the need to write similar analysis code repeatedly, saving time for designers.
\end{enumerate}

To achieve these goals, this thesis contributes the traceable implementations of the analyses given in Table \ref{tab:models_to_contribute}. This set of analyses was deliberately chosen to be mathematically diverse, spanning a wide range of common code patterns in scientific computing. The types of attributes that each analysis is intended to stress-test are given in the right-most column of Table \ref{tab:models_to_contribute}.

\begin{table}[H]

    \centering
    \caption{A list of aircraft design analyses that the thesis implements within a code transformations framework. The middle column lists the non-traceable tools for each analysis that are commonly used in industry today. The right-most column lists the computational attributes that each analysis is intended to stress-test.}
    \label{tab:models_to_contribute}
    \setstretch{1.25}
    \begin{tblr}{
        colspec={@{} X X X @{}},
        row{1}={font=\bfseries},
        hline{3-6}
    }
        \toprule
        Analysis To Contribute                            & Non-Traceable Analogue                                                        & Tests tracing through\ldots                                                                                                 \\ \midrule
        Vortex-Lattice Method Aerodynamics Analysis       & AVL \cite{avl}                                                                & An aerospace geometry engine and discretization; large, vectorized matrix methods, like linear solves                       \\
        Nonlinear Lifting Line Aerodynamics Analysis      & Phillips \& Snyder \cite{phillips_modern_2000}, Reid \cite{reid_general_2020} & Nonlinear systems of equations (i.e., implicit), which often lead to value-dependent code execution (via convergence loops) \\
        Workbook-style Aerodynamics Buildup               & USAF Digital DATCOM \cite{datcom}                                             & Table lookups, large amounts of conditional logic (yielding a wide, branching graph), and scalar-heavy math                 \\
        Rigid-Body Equations of Motion                    & ASWING (dynamics) \cite{aswing}                                               & Ordinary differential equations, which are often implemented in a loop-heavy way (yielding a deep graph)                    \\
        Linearized Aircraft Stability Modal Decomposition & AVL \cite{avl}                                                                & More advanced matrix methods, such as an eigenvalue decomposition                                                           \\
        \bottomrule
    \end{tblr}

\end{table}

These traceable implementations offer value within the context of the thesis itself (in stress-testing the practicality of code transformations), but they also have value as a standalone contribution to the aircraft design community—even outside of design optimization. Because of the mixed-backend numerics library described in Section \ref{sec:code_syntax}, these analyses can be used independently from the design optimization context of AeroSandbox if desired. In such cases, however, they still can retain certain runtime benefits from the code transformations framework, if desired by the user. For example, a traceable workbook-style aerodynamics buildup would seamlessly enable GPU-accelerated evaluations and automatic vectorization, offering significant speedups; an example application might be real-time performance estimation for model predictive controllers or flight simulation.

\section{Vortex-Lattice Method Aerodynamics Analysis}

\subsection{Method Overview}

The vortex-lattice method (VLM) is a low-fidelity aerodynamics analysis used to model the inviscid 3D flow field around a system of lifting surfaces (e.g., wings). It is one of the most common conceptual-level aerodynamics analyses used in aircraft design, as it is computationally inexpensive and interpretable. Common tools that implement the VLM include AVL \cite{avl} and XFLR5 \cite{xflr5}.

A VLM analysis is based on classical potential-flow theory. Because this flow field model is a linear partial differential equation, it can be quickly solved using a boundary-element method representation by superimposing Green's-function kernels. These kernels model disturbances in the flow field that are induced by the lifting surfaces. In a VLM analysis, these kernels are modeled as a collection of \emph{horseshoe vortices} that are distributed along the wing in a regular lattice pattern. Each horseshoe vortex is a connected polyline of uniform-strength vortex filaments, with three segments: a bound vortex on the wing, and two trailing legs extending downstream to infinity\footnote{These trailing legs extend to the far-field Trefftz plane. In theory, these trailing legs should follow the local flow direction (and hence be ``force-free'' by the Kutta-Joukowski theorem), but often they are simply extended directly backwards which simplifies induced velocity computation and removes the need for an iterative wake relaxation.}. (In practice, a fourth leg consisting of a far-downstream ``starting vortex'' can be imagined to close the horseshoe vortex, which forms a ring vortex and thus satisfies the Helmholtz vortex theorems.) The Kutta condition is naturally satisfied, as the only place where the wing can shed vorticity is at the trailing edge (due to placement of the trailing legs).

Each horseshoe vortex has an initially-unknown strength, and the vorticity associated with each vortex creates an induced velocity that affects the global flowfield. To solve for these $N$ unknown vortex strengths, $N$ constraints are needed. A convenient choice is to impose a \emph{flow-tangency} (also called \emph{no-penetration}) boundary condition associated with each horseshoe vortex, where the flow velocity normal to each horseshoe vortex is zero. It is not immediately obvious at which location this flow-tangency condition (called the \emph{collocation point}) should be imposed, relative to each horseshoe vortex. As it turns out, the best choice is to discretize the wing into quadrilateral panels, then place the bound leg at the quarter-chord point of each panel, and place the collocation point at the three-quarter-chord point. This choice results in higher-order convergence with respect to discretization resolution than any other choice, with derivation for this given by Katz and Plotkin \cite{katz_lowspeed_2004}. (One way to intuitively understand this reasoning is that the VLM is essentially a 3D analogue of 2D thin airfoil theory.)

Because of the linearity of the governing equations, the unknown horseshoe vortex strengths can be solved as a linear system of equations. Due to the global influence of each vortex on the flowfield, this linear system of equations is dense and asymmetric, so it is typically solved using LU factorization. The solution to this linear system gives the vortex strengths, which can then be used to reconstruct the flowfield around the lifting surfaces. The vortex strengths are also conveniently equal to the local difference in pressure coefficient between the top and bottom surfaces of the wing, which can be used to visually interpret the flow field. Lift force computation is usually performed using the Kutta-Joukowski theorem on each bound leg. Drag force calculation (which only includes induced drag, as a VLM is inviscid) can be accurately performed using a Trefftz plane wake integral. Further details on this analysis formulation are available in work by Katz and Plotkin \cite{katz_lowspeed_2004} and Drela \cite{drela_flight_2013}.

\subsection{Implementation}

