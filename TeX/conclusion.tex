\chapter{Conclusions}

This thesis introduces five new contributions to aircraft design, multidisciplinary design optimization, and scientific machine learning. These include:

\begin{enumerate}[noitemsep]
    \item A code transformations paradigm for engineering design optimization (Chapter \ref{chap:code_transformations}), and a reference implementation in the open-source Python library AeroSandbox \cite{asb_github}; also includes a series of aircraft design case studies to demonstrate this framework (Chapter \ref{chap:design-studies})
    \item Traceable implementations of aerospace physics models (Chapter \ref{chap:physics})
    \item Sparsity tracing via NaN-propagation (Chapter \ref{chap:nan_propagation})
    \item Integration of physics-informed machine learning surrogates into a code transformations framework (Chapter \ref{chap:physics-informed-ml})
    \item A novel approach to aircraft system identification from minimal flight data (Chapter \ref{chap:aircraft_sysid})
\end{enumerate}

These contributions are synthesized into a unified framework for engineering design optimization, AeroSandbox, which is demonstrated throughout on a variety of aerospace design problems. This framework is designed to be high-performance, mathematically-flexible, and user-friendly, in order to address various frictions that industry users currently experience with design optimization workflows (Chapter \ref{chap:literature}). If this thesis is to be distilled into a series of key takeaways, they would be as follows:
\begin{enumerate}
    \item Code transformations have enormous potential to improve not only the runtime speed of engineering design optimization problems, but also the ease of model development and implementation. This reduction in the barrier to entry for design optimization can help academic advances translate more readily into industry practice.
    \item When developing a new multidisciplinary design optimization (MDO) framework, the decision to use a code transformations paradigm is best considered from the outset of the framework's development. Code ultimately must be written with traceability in mind, and this burden will inevitably either be put on the end-user (undesirable) or on the framework developer (preferred). If the latter is to be achieved, the framework itself must be built on top of a unified numerics stack that allows computational graphs to be traced\footnote{As a result, an MDO framework that only adds in automatic differentiation support to individual models after the fact will have more user frictions than a framework that supports this end-to-end. The discussion of limitations around code syntax and style in Section \ref{sec:syntax-interface} provides examples of why this occurs.}.
\end{enumerate}

\noindent We live in a time of great bounty in scientific computing, thanks to abundant computational power, fundamental algorithmic advances, and widespread accessibility through the vibrant open-source community. These factors \emph{should} supercharge design engineers, but these theoretical improvements have not been reflected in reality -- largely due to the difficulty of wielding these tools. Thus, the onus upon tool developers to focus on advancing not just traditional performance metrics, but also usability and interoperability. The contributions of this thesis are intended to address—and kick off further research regarding -- the intersection of scientific computing and the practical realities of engineering design, so that as a community we may work toward bridging the math-to-metal gap.