\documentclass[11pt]{article}

    \usepackage[breakable]{tcolorbox}
    \usepackage{parskip} % Stop auto-indenting (to mimic markdown behaviour)
    

    % Basic figure setup, for now with no caption control since it's done
    % automatically by Pandoc (which extracts ![](path) syntax from Markdown).
    \usepackage{graphicx}
    % Maintain compatibility with old templates. Remove in nbconvert 6.0
    \let\Oldincludegraphics\includegraphics
    % Ensure that by default, figures have no caption (until we provide a
    % proper Figure object with a Caption API and a way to capture that
    % in the conversion process - todo).
    \usepackage{caption}
    \DeclareCaptionFormat{nocaption}{}
    \captionsetup{format=nocaption,aboveskip=0pt,belowskip=0pt}

    \usepackage{float}
    \floatplacement{figure}{H} % forces figures to be placed at the correct location
    \usepackage{xcolor} % Allow colors to be defined
    \usepackage{enumerate} % Needed for markdown enumerations to work
    \usepackage{geometry} % Used to adjust the document margins
    \usepackage{amsmath} % Equations
    \usepackage{amssymb} % Equations
    \usepackage{textcomp} % defines textquotesingle
    % Hack from http://tex.stackexchange.com/a/47451/13684:
    \AtBeginDocument{%
        \def\PYZsq{\textquotesingle}% Upright quotes in Pygmentized code
    }
    \usepackage{upquote} % Upright quotes for verbatim code
    \usepackage{eurosym} % defines \euro

    \usepackage{iftex}
    \ifPDFTeX
        \usepackage[T1]{fontenc}
        \IfFileExists{alphabeta.sty}{
              \usepackage{alphabeta}
          }{
              \usepackage[mathletters]{ucs}
              \usepackage[utf8x]{inputenc}
          }
    \else
        \usepackage{fontspec}
        \usepackage{unicode-math}
    \fi

    \usepackage{fancyvrb} % verbatim replacement that allows latex
    \usepackage{grffile} % extends the file name processing of package graphics
                         % to support a larger range
    \makeatletter % fix for old versions of grffile with XeLaTeX
    \@ifpackagelater{grffile}{2019/11/01}
    {
      % Do nothing on new versions
    }
    {
      \def\Gread@@xetex#1{%
        \IfFileExists{"\Gin@base".bb}%
        {\Gread@eps{\Gin@base.bb}}%
        {\Gread@@xetex@aux#1}%
      }
    }
    \makeatother
    \usepackage[Export]{adjustbox} % Used to constrain images to a maximum size
    \adjustboxset{max size={0.9\linewidth}{0.9\paperheight}}

    % The hyperref package gives us a pdf with properly built
    % internal navigation ('pdf bookmarks' for the table of contents,
    % internal cross-reference links, web links for URLs, etc.)
    \usepackage{hyperref}
    % The default LaTeX title has an obnoxious amount of whitespace. By default,
    % titling removes some of it. It also provides customization options.
    \usepackage{titling}
    \usepackage{longtable} % longtable support required by pandoc >1.10
    \usepackage{booktabs}  % table support for pandoc > 1.12.2
    \usepackage{array}     % table support for pandoc >= 2.11.3
    \usepackage{calc}      % table minipage width calculation for pandoc >= 2.11.1
    \usepackage[inline]{enumitem} % IRkernel/repr support (it uses the enumerate* environment)
    \usepackage[normalem]{ulem} % ulem is needed to support strikethroughs (\sout)
                                % normalem makes italics be italics, not underlines
    \usepackage{mathrsfs}
    

    
    % Colors for the hyperref package
    \definecolor{urlcolor}{rgb}{0,.145,.698}
    \definecolor{linkcolor}{rgb}{.71,0.21,0.01}
    \definecolor{citecolor}{rgb}{.12,.54,.11}

    % ANSI colors
    \definecolor{ansi-black}{HTML}{3E424D}
    \definecolor{ansi-black-intense}{HTML}{282C36}
    \definecolor{ansi-red}{HTML}{E75C58}
    \definecolor{ansi-red-intense}{HTML}{B22B31}
    \definecolor{ansi-green}{HTML}{00A250}
    \definecolor{ansi-green-intense}{HTML}{007427}
    \definecolor{ansi-yellow}{HTML}{DDB62B}
    \definecolor{ansi-yellow-intense}{HTML}{B27D12}
    \definecolor{ansi-blue}{HTML}{208FFB}
    \definecolor{ansi-blue-intense}{HTML}{0065CA}
    \definecolor{ansi-magenta}{HTML}{D160C4}
    \definecolor{ansi-magenta-intense}{HTML}{A03196}
    \definecolor{ansi-cyan}{HTML}{60C6C8}
    \definecolor{ansi-cyan-intense}{HTML}{258F8F}
    \definecolor{ansi-white}{HTML}{C5C1B4}
    \definecolor{ansi-white-intense}{HTML}{A1A6B2}
    \definecolor{ansi-default-inverse-fg}{HTML}{FFFFFF}
    \definecolor{ansi-default-inverse-bg}{HTML}{000000}

    % common color for the border for error outputs.
    \definecolor{outerrorbackground}{HTML}{FFDFDF}

    % commands and environments needed by pandoc snippets
    % extracted from the output of `pandoc -s`
    \providecommand{\tightlist}{%
      \setlength{\itemsep}{0pt}\setlength{\parskip}{0pt}}
    \DefineVerbatimEnvironment{Highlighting}{Verbatim}{commandchars=\\\{\}}
    % Add ',fontsize=\small' for more characters per line
    \newenvironment{Shaded}{}{}
    \newcommand{\KeywordTok}[1]{\textcolor[rgb]{0.00,0.44,0.13}{\textbf{{#1}}}}
    \newcommand{\DataTypeTok}[1]{\textcolor[rgb]{0.56,0.13,0.00}{{#1}}}
    \newcommand{\DecValTok}[1]{\textcolor[rgb]{0.25,0.63,0.44}{{#1}}}
    \newcommand{\BaseNTok}[1]{\textcolor[rgb]{0.25,0.63,0.44}{{#1}}}
    \newcommand{\FloatTok}[1]{\textcolor[rgb]{0.25,0.63,0.44}{{#1}}}
    \newcommand{\CharTok}[1]{\textcolor[rgb]{0.25,0.44,0.63}{{#1}}}
    \newcommand{\StringTok}[1]{\textcolor[rgb]{0.25,0.44,0.63}{{#1}}}
    \newcommand{\CommentTok}[1]{\textcolor[rgb]{0.38,0.63,0.69}{\textit{{#1}}}}
    \newcommand{\OtherTok}[1]{\textcolor[rgb]{0.00,0.44,0.13}{{#1}}}
    \newcommand{\AlertTok}[1]{\textcolor[rgb]{1.00,0.00,0.00}{\textbf{{#1}}}}
    \newcommand{\FunctionTok}[1]{\textcolor[rgb]{0.02,0.16,0.49}{{#1}}}
    \newcommand{\RegionMarkerTok}[1]{{#1}}
    \newcommand{\ErrorTok}[1]{\textcolor[rgb]{1.00,0.00,0.00}{\textbf{{#1}}}}
    \newcommand{\NormalTok}[1]{{#1}}

    % Additional commands for more recent versions of Pandoc
    \newcommand{\ConstantTok}[1]{\textcolor[rgb]{0.53,0.00,0.00}{{#1}}}
    \newcommand{\SpecialCharTok}[1]{\textcolor[rgb]{0.25,0.44,0.63}{{#1}}}
    \newcommand{\VerbatimStringTok}[1]{\textcolor[rgb]{0.25,0.44,0.63}{{#1}}}
    \newcommand{\SpecialStringTok}[1]{\textcolor[rgb]{0.73,0.40,0.53}{{#1}}}
    \newcommand{\ImportTok}[1]{{#1}}
    \newcommand{\DocumentationTok}[1]{\textcolor[rgb]{0.73,0.13,0.13}{\textit{{#1}}}}
    \newcommand{\AnnotationTok}[1]{\textcolor[rgb]{0.38,0.63,0.69}{\textbf{\textit{{#1}}}}}
    \newcommand{\CommentVarTok}[1]{\textcolor[rgb]{0.38,0.63,0.69}{\textbf{\textit{{#1}}}}}
    \newcommand{\VariableTok}[1]{\textcolor[rgb]{0.10,0.09,0.49}{{#1}}}
    \newcommand{\ControlFlowTok}[1]{\textcolor[rgb]{0.00,0.44,0.13}{\textbf{{#1}}}}
    \newcommand{\OperatorTok}[1]{\textcolor[rgb]{0.40,0.40,0.40}{{#1}}}
    \newcommand{\BuiltInTok}[1]{{#1}}
    \newcommand{\ExtensionTok}[1]{{#1}}
    \newcommand{\PreprocessorTok}[1]{\textcolor[rgb]{0.74,0.48,0.00}{{#1}}}
    \newcommand{\AttributeTok}[1]{\textcolor[rgb]{0.49,0.56,0.16}{{#1}}}
    \newcommand{\InformationTok}[1]{\textcolor[rgb]{0.38,0.63,0.69}{\textbf{\textit{{#1}}}}}
    \newcommand{\WarningTok}[1]{\textcolor[rgb]{0.38,0.63,0.69}{\textbf{\textit{{#1}}}}}


    % Define a nice break command that doesn't care if a line doesn't already
    % exist.
    \def\br{\hspace*{\fill} \\* }
    % Math Jax compatibility definitions
    \def\gt{>}
    \def\lt{<}
    \let\Oldtex\TeX
    \let\Oldlatex\LaTeX
    \renewcommand{\TeX}{\textrm{\Oldtex}}
    \renewcommand{\LaTeX}{\textrm{\Oldlatex}}
    % Document parameters
    % Document title
    \title{manuscript}
    
    
    
    
    
    
    
% Pygments definitions
\makeatletter
\def\PY@reset{\let\PY@it=\relax \let\PY@bf=\relax%
    \let\PY@ul=\relax \let\PY@tc=\relax%
    \let\PY@bc=\relax \let\PY@ff=\relax}
\def\PY@tok#1{\csname PY@tok@#1\endcsname}
\def\PY@toks#1+{\ifx\relax#1\empty\else%
    \PY@tok{#1}\expandafter\PY@toks\fi}
\def\PY@do#1{\PY@bc{\PY@tc{\PY@ul{%
    \PY@it{\PY@bf{\PY@ff{#1}}}}}}}
\def\PY#1#2{\PY@reset\PY@toks#1+\relax+\PY@do{#2}}

\@namedef{PY@tok@w}{\def\PY@tc##1{\textcolor[rgb]{0.73,0.73,0.73}{##1}}}
\@namedef{PY@tok@c}{\let\PY@it=\textit\def\PY@tc##1{\textcolor[rgb]{0.24,0.48,0.48}{##1}}}
\@namedef{PY@tok@cp}{\def\PY@tc##1{\textcolor[rgb]{0.61,0.40,0.00}{##1}}}
\@namedef{PY@tok@k}{\let\PY@bf=\textbf\def\PY@tc##1{\textcolor[rgb]{0.00,0.50,0.00}{##1}}}
\@namedef{PY@tok@kp}{\def\PY@tc##1{\textcolor[rgb]{0.00,0.50,0.00}{##1}}}
\@namedef{PY@tok@kt}{\def\PY@tc##1{\textcolor[rgb]{0.69,0.00,0.25}{##1}}}
\@namedef{PY@tok@o}{\def\PY@tc##1{\textcolor[rgb]{0.40,0.40,0.40}{##1}}}
\@namedef{PY@tok@ow}{\let\PY@bf=\textbf\def\PY@tc##1{\textcolor[rgb]{0.67,0.13,1.00}{##1}}}
\@namedef{PY@tok@nb}{\def\PY@tc##1{\textcolor[rgb]{0.00,0.50,0.00}{##1}}}
\@namedef{PY@tok@nf}{\def\PY@tc##1{\textcolor[rgb]{0.00,0.00,1.00}{##1}}}
\@namedef{PY@tok@nc}{\let\PY@bf=\textbf\def\PY@tc##1{\textcolor[rgb]{0.00,0.00,1.00}{##1}}}
\@namedef{PY@tok@nn}{\let\PY@bf=\textbf\def\PY@tc##1{\textcolor[rgb]{0.00,0.00,1.00}{##1}}}
\@namedef{PY@tok@ne}{\let\PY@bf=\textbf\def\PY@tc##1{\textcolor[rgb]{0.80,0.25,0.22}{##1}}}
\@namedef{PY@tok@nv}{\def\PY@tc##1{\textcolor[rgb]{0.10,0.09,0.49}{##1}}}
\@namedef{PY@tok@no}{\def\PY@tc##1{\textcolor[rgb]{0.53,0.00,0.00}{##1}}}
\@namedef{PY@tok@nl}{\def\PY@tc##1{\textcolor[rgb]{0.46,0.46,0.00}{##1}}}
\@namedef{PY@tok@ni}{\let\PY@bf=\textbf\def\PY@tc##1{\textcolor[rgb]{0.44,0.44,0.44}{##1}}}
\@namedef{PY@tok@na}{\def\PY@tc##1{\textcolor[rgb]{0.41,0.47,0.13}{##1}}}
\@namedef{PY@tok@nt}{\let\PY@bf=\textbf\def\PY@tc##1{\textcolor[rgb]{0.00,0.50,0.00}{##1}}}
\@namedef{PY@tok@nd}{\def\PY@tc##1{\textcolor[rgb]{0.67,0.13,1.00}{##1}}}
\@namedef{PY@tok@s}{\def\PY@tc##1{\textcolor[rgb]{0.73,0.13,0.13}{##1}}}
\@namedef{PY@tok@sd}{\let\PY@it=\textit\def\PY@tc##1{\textcolor[rgb]{0.73,0.13,0.13}{##1}}}
\@namedef{PY@tok@si}{\let\PY@bf=\textbf\def\PY@tc##1{\textcolor[rgb]{0.64,0.35,0.47}{##1}}}
\@namedef{PY@tok@se}{\let\PY@bf=\textbf\def\PY@tc##1{\textcolor[rgb]{0.67,0.36,0.12}{##1}}}
\@namedef{PY@tok@sr}{\def\PY@tc##1{\textcolor[rgb]{0.64,0.35,0.47}{##1}}}
\@namedef{PY@tok@ss}{\def\PY@tc##1{\textcolor[rgb]{0.10,0.09,0.49}{##1}}}
\@namedef{PY@tok@sx}{\def\PY@tc##1{\textcolor[rgb]{0.00,0.50,0.00}{##1}}}
\@namedef{PY@tok@m}{\def\PY@tc##1{\textcolor[rgb]{0.40,0.40,0.40}{##1}}}
\@namedef{PY@tok@gh}{\let\PY@bf=\textbf\def\PY@tc##1{\textcolor[rgb]{0.00,0.00,0.50}{##1}}}
\@namedef{PY@tok@gu}{\let\PY@bf=\textbf\def\PY@tc##1{\textcolor[rgb]{0.50,0.00,0.50}{##1}}}
\@namedef{PY@tok@gd}{\def\PY@tc##1{\textcolor[rgb]{0.63,0.00,0.00}{##1}}}
\@namedef{PY@tok@gi}{\def\PY@tc##1{\textcolor[rgb]{0.00,0.52,0.00}{##1}}}
\@namedef{PY@tok@gr}{\def\PY@tc##1{\textcolor[rgb]{0.89,0.00,0.00}{##1}}}
\@namedef{PY@tok@ge}{\let\PY@it=\textit}
\@namedef{PY@tok@gs}{\let\PY@bf=\textbf}
\@namedef{PY@tok@gp}{\let\PY@bf=\textbf\def\PY@tc##1{\textcolor[rgb]{0.00,0.00,0.50}{##1}}}
\@namedef{PY@tok@go}{\def\PY@tc##1{\textcolor[rgb]{0.44,0.44,0.44}{##1}}}
\@namedef{PY@tok@gt}{\def\PY@tc##1{\textcolor[rgb]{0.00,0.27,0.87}{##1}}}
\@namedef{PY@tok@err}{\def\PY@bc##1{{\setlength{\fboxsep}{\string -\fboxrule}\fcolorbox[rgb]{1.00,0.00,0.00}{1,1,1}{\strut ##1}}}}
\@namedef{PY@tok@kc}{\let\PY@bf=\textbf\def\PY@tc##1{\textcolor[rgb]{0.00,0.50,0.00}{##1}}}
\@namedef{PY@tok@kd}{\let\PY@bf=\textbf\def\PY@tc##1{\textcolor[rgb]{0.00,0.50,0.00}{##1}}}
\@namedef{PY@tok@kn}{\let\PY@bf=\textbf\def\PY@tc##1{\textcolor[rgb]{0.00,0.50,0.00}{##1}}}
\@namedef{PY@tok@kr}{\let\PY@bf=\textbf\def\PY@tc##1{\textcolor[rgb]{0.00,0.50,0.00}{##1}}}
\@namedef{PY@tok@bp}{\def\PY@tc##1{\textcolor[rgb]{0.00,0.50,0.00}{##1}}}
\@namedef{PY@tok@fm}{\def\PY@tc##1{\textcolor[rgb]{0.00,0.00,1.00}{##1}}}
\@namedef{PY@tok@vc}{\def\PY@tc##1{\textcolor[rgb]{0.10,0.09,0.49}{##1}}}
\@namedef{PY@tok@vg}{\def\PY@tc##1{\textcolor[rgb]{0.10,0.09,0.49}{##1}}}
\@namedef{PY@tok@vi}{\def\PY@tc##1{\textcolor[rgb]{0.10,0.09,0.49}{##1}}}
\@namedef{PY@tok@vm}{\def\PY@tc##1{\textcolor[rgb]{0.10,0.09,0.49}{##1}}}
\@namedef{PY@tok@sa}{\def\PY@tc##1{\textcolor[rgb]{0.73,0.13,0.13}{##1}}}
\@namedef{PY@tok@sb}{\def\PY@tc##1{\textcolor[rgb]{0.73,0.13,0.13}{##1}}}
\@namedef{PY@tok@sc}{\def\PY@tc##1{\textcolor[rgb]{0.73,0.13,0.13}{##1}}}
\@namedef{PY@tok@dl}{\def\PY@tc##1{\textcolor[rgb]{0.73,0.13,0.13}{##1}}}
\@namedef{PY@tok@s2}{\def\PY@tc##1{\textcolor[rgb]{0.73,0.13,0.13}{##1}}}
\@namedef{PY@tok@sh}{\def\PY@tc##1{\textcolor[rgb]{0.73,0.13,0.13}{##1}}}
\@namedef{PY@tok@s1}{\def\PY@tc##1{\textcolor[rgb]{0.73,0.13,0.13}{##1}}}
\@namedef{PY@tok@mb}{\def\PY@tc##1{\textcolor[rgb]{0.40,0.40,0.40}{##1}}}
\@namedef{PY@tok@mf}{\def\PY@tc##1{\textcolor[rgb]{0.40,0.40,0.40}{##1}}}
\@namedef{PY@tok@mh}{\def\PY@tc##1{\textcolor[rgb]{0.40,0.40,0.40}{##1}}}
\@namedef{PY@tok@mi}{\def\PY@tc##1{\textcolor[rgb]{0.40,0.40,0.40}{##1}}}
\@namedef{PY@tok@il}{\def\PY@tc##1{\textcolor[rgb]{0.40,0.40,0.40}{##1}}}
\@namedef{PY@tok@mo}{\def\PY@tc##1{\textcolor[rgb]{0.40,0.40,0.40}{##1}}}
\@namedef{PY@tok@ch}{\let\PY@it=\textit\def\PY@tc##1{\textcolor[rgb]{0.24,0.48,0.48}{##1}}}
\@namedef{PY@tok@cm}{\let\PY@it=\textit\def\PY@tc##1{\textcolor[rgb]{0.24,0.48,0.48}{##1}}}
\@namedef{PY@tok@cpf}{\let\PY@it=\textit\def\PY@tc##1{\textcolor[rgb]{0.24,0.48,0.48}{##1}}}
\@namedef{PY@tok@c1}{\let\PY@it=\textit\def\PY@tc##1{\textcolor[rgb]{0.24,0.48,0.48}{##1}}}
\@namedef{PY@tok@cs}{\let\PY@it=\textit\def\PY@tc##1{\textcolor[rgb]{0.24,0.48,0.48}{##1}}}

\def\PYZbs{\char`\\}
\def\PYZus{\char`\_}
\def\PYZob{\char`\{}
\def\PYZcb{\char`\}}
\def\PYZca{\char`\^}
\def\PYZam{\char`\&}
\def\PYZlt{\char`\<}
\def\PYZgt{\char`\>}
\def\PYZsh{\char`\#}
\def\PYZpc{\char`\%}
\def\PYZdl{\char`\$}
\def\PYZhy{\char`\-}
\def\PYZsq{\char`\'}
\def\PYZdq{\char`\"}
\def\PYZti{\char`\~}
% for compatibility with earlier versions
\def\PYZat{@}
\def\PYZlb{[}
\def\PYZrb{]}
\makeatother


    % For linebreaks inside Verbatim environment from package fancyvrb.
    \makeatletter
        \newbox\Wrappedcontinuationbox
        \newbox\Wrappedvisiblespacebox
        \newcommand*\Wrappedvisiblespace {\textcolor{red}{\textvisiblespace}}
        \newcommand*\Wrappedcontinuationsymbol {\textcolor{red}{\llap{\tiny$\m@th\hookrightarrow$}}}
        \newcommand*\Wrappedcontinuationindent {3ex }
        \newcommand*\Wrappedafterbreak {\kern\Wrappedcontinuationindent\copy\Wrappedcontinuationbox}
        % Take advantage of the already applied Pygments mark-up to insert
        % potential linebreaks for TeX processing.
        %        {, <, #, %, $, ' and ": go to next line.
        %        _, }, ^, &, >, - and ~: stay at end of broken line.
        % Use of \textquotesingle for straight quote.
        \newcommand*\Wrappedbreaksatspecials {%
            \def\PYGZus{\discretionary{\char`\_}{\Wrappedafterbreak}{\char`\_}}%
            \def\PYGZob{\discretionary{}{\Wrappedafterbreak\char`\{}{\char`\{}}%
            \def\PYGZcb{\discretionary{\char`\}}{\Wrappedafterbreak}{\char`\}}}%
            \def\PYGZca{\discretionary{\char`\^}{\Wrappedafterbreak}{\char`\^}}%
            \def\PYGZam{\discretionary{\char`\&}{\Wrappedafterbreak}{\char`\&}}%
            \def\PYGZlt{\discretionary{}{\Wrappedafterbreak\char`\<}{\char`\<}}%
            \def\PYGZgt{\discretionary{\char`\>}{\Wrappedafterbreak}{\char`\>}}%
            \def\PYGZsh{\discretionary{}{\Wrappedafterbreak\char`\#}{\char`\#}}%
            \def\PYGZpc{\discretionary{}{\Wrappedafterbreak\char`\%}{\char`\%}}%
            \def\PYGZdl{\discretionary{}{\Wrappedafterbreak\char`\$}{\char`\$}}%
            \def\PYGZhy{\discretionary{\char`\-}{\Wrappedafterbreak}{\char`\-}}%
            \def\PYGZsq{\discretionary{}{\Wrappedafterbreak\textquotesingle}{\textquotesingle}}%
            \def\PYGZdq{\discretionary{}{\Wrappedafterbreak\char`\"}{\char`\"}}%
            \def\PYGZti{\discretionary{\char`\~}{\Wrappedafterbreak}{\char`\~}}%
        }
        % Some characters . , ; ? ! / are not pygmentized.
        % This macro makes them "active" and they will insert potential linebreaks
        \newcommand*\Wrappedbreaksatpunct {%
            \lccode`\~`\.\lowercase{\def~}{\discretionary{\hbox{\char`\.}}{\Wrappedafterbreak}{\hbox{\char`\.}}}%
            \lccode`\~`\,\lowercase{\def~}{\discretionary{\hbox{\char`\,}}{\Wrappedafterbreak}{\hbox{\char`\,}}}%
            \lccode`\~`\;\lowercase{\def~}{\discretionary{\hbox{\char`\;}}{\Wrappedafterbreak}{\hbox{\char`\;}}}%
            \lccode`\~`\:\lowercase{\def~}{\discretionary{\hbox{\char`\:}}{\Wrappedafterbreak}{\hbox{\char`\:}}}%
            \lccode`\~`\?\lowercase{\def~}{\discretionary{\hbox{\char`\?}}{\Wrappedafterbreak}{\hbox{\char`\?}}}%
            \lccode`\~`\!\lowercase{\def~}{\discretionary{\hbox{\char`\!}}{\Wrappedafterbreak}{\hbox{\char`\!}}}%
            \lccode`\~`\/\lowercase{\def~}{\discretionary{\hbox{\char`\/}}{\Wrappedafterbreak}{\hbox{\char`\/}}}%
            \catcode`\.\active
            \catcode`\,\active
            \catcode`\;\active
            \catcode`\:\active
            \catcode`\?\active
            \catcode`\!\active
            \catcode`\/\active
            \lccode`\~`\~
        }
    \makeatother

    \let\OriginalVerbatim=\Verbatim
    \makeatletter
    \renewcommand{\Verbatim}[1][1]{%
        %\parskip\z@skip
        \sbox\Wrappedcontinuationbox {\Wrappedcontinuationsymbol}%
        \sbox\Wrappedvisiblespacebox {\FV@SetupFont\Wrappedvisiblespace}%
        \def\FancyVerbFormatLine ##1{\hsize\linewidth
            \vtop{\raggedright\hyphenpenalty\z@\exhyphenpenalty\z@
                \doublehyphendemerits\z@\finalhyphendemerits\z@
                \strut ##1\strut}%
        }%
        % If the linebreak is at a space, the latter will be displayed as visible
        % space at end of first line, and a continuation symbol starts next line.
        % Stretch/shrink are however usually zero for typewriter font.
        \def\FV@Space {%
            \nobreak\hskip\z@ plus\fontdimen3\font minus\fontdimen4\font
            \discretionary{\copy\Wrappedvisiblespacebox}{\Wrappedafterbreak}
            {\kern\fontdimen2\font}%
        }%

        % Allow breaks at special characters using \PYG... macros.
        \Wrappedbreaksatspecials
        % Breaks at punctuation characters . , ; ? ! and / need catcode=\active
        \OriginalVerbatim[#1,codes*=\Wrappedbreaksatpunct]%
    }
    \makeatother

    % Exact colors from NB
    \definecolor{incolor}{HTML}{303F9F}
    \definecolor{outcolor}{HTML}{D84315}
    \definecolor{cellborder}{HTML}{CFCFCF}
    \definecolor{cellbackground}{HTML}{F7F7F7}

    % prompt
    \makeatletter
    \newcommand{\boxspacing}{\kern\kvtcb@left@rule\kern\kvtcb@boxsep}
    \makeatother
    \newcommand{\prompt}[4]{
        {\ttfamily\llap{{\color{#2}[#3]:\hspace{3pt}#4}}\vspace{-\baselineskip}}
    }
    

    
    % Prevent overflowing lines due to hard-to-break entities
    \sloppy
    % Setup hyperref package
    \hypersetup{
      breaklinks=true,  % so long urls are correctly broken across lines
      colorlinks=true,
      urlcolor=urlcolor,
      linkcolor=linkcolor,
      citecolor=citecolor,
      }
    % Slightly bigger margins than the latex defaults
    
    \geometry{verbose,tmargin=1in,bmargin=1in,lmargin=1in,rmargin=1in}
    
    

\begin{document}
    
    \maketitle
    
    

    
    \hypertarget{physics-informed-regression-of-aircraft-performance-from-minimal-flight-data}{%
\section{Physics-Informed Regression of Aircraft Performance from
Minimal Flight
Data}\label{physics-informed-regression-of-aircraft-performance-from-minimal-flight-data}}

by Peter Sharpe

    \hypertarget{introduction}{%
\subsection{Introduction}\label{introduction}}

One of the primary goals of an initial flight test of a new aircraft is
to experimentally determine the aircraft's aerodynamic and propulsive
performance characteristics.

The desired output of this process typically includes: - The aircraft's
aerodynamic polar, which gives the relationship between the aircraft's
lift and drag coefficients. - The aircraft's power curve, which gives
the required power to maintain level flight as a function of airspeed. -
The aircraft's propulsive efficiency curve, which yields the overall
propulsive efficiency, typically as a function of throttle setting
and/or propeller advance ratio, if relevant.

All three of these results represent sweeps through the aircraft's
flight envelope, varying the aircraft's airspeed, pitch trim setting,
and throttle setting, and measuring the resulting lift, drag, and power
required.

    \hypertarget{traditional-flight-test-measurement-methods}{%
\subsubsection{Traditional Flight Test Measurement
Methods}\label{traditional-flight-test-measurement-methods}}

Typically, these performance outputs are obtained by performing an
extensive campaign of careful, controlled flight experiments at
quasi-steady flight conditions.

    \hypertarget{aerodynamic-polar}{%
\paragraph{Aerodynamic Polar}\label{aerodynamic-polar}}

For example, the aerodynamic polar is commonly measured by performing a
series of long, steady power-off glides at different airspeeds. An
implicit assumption here is that pitch trim is adjusted to maintain
these airspeeds. Windmilling drag is also estimated and calibrated out,
although this introduces significant uncertainty. The glide ratio (or
equivalently, \(L/D\)) is then measured, yielding

\[(L/D) = \frac{h(t_1) - h(t_2)}{V \cdot (t_2 - t_1)}\]

where \(h(t)\) represents the altitude at time \(t\), and \(V\)
represents the airspeed. The drag is then computed as \(D = W / (L/D)\),
and \(C_L\) and \(C_D\) can be nondimensionalized from here.

    \hypertarget{power-curve}{%
\paragraph{Power Curve}\label{power-curve}}

Similarly, the power curve is often measured by flying in steady level
flight for an extended period at a given airspeed, adjusting pitch trim
as needed. The throttle setting is required, and this is then repeated
for various airspeeds.

    \hypertarget{propulsive-efficiency}{%
\paragraph{Propulsive Efficiency}\label{propulsive-efficiency}}

The propulsive efficiency can be roughly estimated based on these
experiments as well. One possible procedure is as follows:

\begin{enumerate}
\def\labelenumi{\arabic{enumi}.}
\tightlist
\item
  The input power to the propulsion system \(P_{\rm in}\) is measured at
  a given throttle setting. In a liquid-fueled airplane, this can be
  computing using the fuel flow rate at a given throttle setting (read
  off the panel via a fuel flow meter) and the specific energy of the
  fuel. For an electric airplane, one can measure the battery current
  and voltage.
\item
  We observe the steady-state climb/sink rate of the aircraft at this
  throttle setting and we compare this to the power-off sink rate. The
  difference between these two sink rates represents the air power done
  by the propulsion system. The air power can be computed as:
\end{enumerate}

\[P_{\rm air}=T \cdot V = m g \left( \frac{dh}{dt}\Big |_\text{power on} - \frac{dh}{dt}\Big |_\text{power off} \right)\]

\begin{enumerate}
\def\labelenumi{\arabic{enumi}.}
\setcounter{enumi}{2}
\tightlist
\item
  Then, we can compute the propulsive efficiency as the ratio of the
  input power to the air power:
\end{enumerate}

\[\eta = \frac{P_{\rm air}}{P_{\rm in}}\]

    \hypertarget{limitations-of-traditional-methods}{%
\subsubsection{Limitations of Traditional
Methods}\label{limitations-of-traditional-methods}}

Traditional methods for measuring aerodynamic and propulsive performance
have several limitations. By far the biggest limitation is that they
require vast amounts of flight time to obtain a sufficient number of
data points to characterize the aircraft's performance. This is both
expensive and time-consuming.

Furthermore, the aircraft must be flown at precisely-controlled
conditions. This can be frustrating and tedious for the pilot, and it
can also be dangerous if the aircraft is flown for an extended duration
at conditions that are close to the edge of the flight envelope (e.g.,
behind the power curve on a not-yet-characterized experimental
airplane).

Traditional methods also make no direct estimate of sensor noise (and
hence, uncertainty) - data is collected and averaged until the
experimenter is satisfied that the data is ``good enough''. This is a
subjective process, and it is difficult to quantify the uncertainty in
the resulting performance estimates.

Finally, traditional methods are not well-suited to estimating the
performance of an aircraft that is not in steady flight. For example,
data recorded during an aircraft's climb or descent to flight test
altitude is typically discarded, which is wasteful; ideally, every
single second of data should be used to refine our estimate of the
aircraft's performance.

    \hypertarget{inference-based-flight-testing-methods}{%
\subsubsection{Inference-Based Flight Testing
Methods}\label{inference-based-flight-testing-methods}}

By contrast, we propose a (new?) method of estimating these performance
characteristics which relies on a fundamentally different perspective
about flight testing. Traditionally, flight testing is viewed as a
process of \emph{pure measurement}, based only on analyzing the data
itself. By contrast, we view the aircraft flight test as a
\emph{data-generating process}, from which we can \emph{infer}
performance relationships. This might sound like a small difference, but
it allows us to recognize that we have more information than just the
data itself:

\begin{itemize}
\tightlist
\item
  We know that the true state of the aircraft must follow Newtonian
  physics. This allows us to perform relatively robust correction for
  unsteady effects.
\item
  We have some priors (e.g., informed guesses) about what the aircraft's
  aerodynamic and propulsive performance characteristics should be.
  Incorporation of physics-based performance models embeds our physical
  understanding of the aircraft into the estimation process, improving
  accuracy.
\end{itemize}

Finally, we can also make advances at the level of the sensor data by
rigorously estimating the sensor noise \emph{using the data itself},
which allows cleaner data inputs to this entire model. The following
figure illustrates our proposed approach:

    \begin{tcolorbox}[breakable, size=fbox, boxrule=1pt, pad at break*=1mm,colback=cellbackground, colframe=cellborder]
\prompt{In}{incolor}{1}{\boxspacing}
\begin{Verbatim}[commandchars=\\\{\}]
\PY{k+kn}{import} \PY{n+nn}{graphviz}

\PY{n}{gv\PYZus{}settings} \PY{o}{=} \PY{n+nb}{dict}\PY{p}{(}
    \PY{n}{graph\PYZus{}attr}\PY{o}{=}\PY{n+nb}{dict}\PY{p}{(}
        \PY{n}{rankdir}\PY{o}{=}\PY{l+s+s1}{\PYZsq{}}\PY{l+s+s1}{TB}\PY{l+s+s1}{\PYZsq{}}\PY{p}{,}
        \PY{n}{nodesep}\PY{o}{=}\PY{l+s+s1}{\PYZsq{}}\PY{l+s+s1}{0.2}\PY{l+s+s1}{\PYZsq{}}\PY{p}{,}
        \PY{n}{bgcolor}\PY{o}{=}\PY{l+s+s1}{\PYZsq{}}\PY{l+s+s1}{\PYZsh{}FFFFFF}\PY{l+s+s1}{\PYZsq{}}\PY{p}{,}
        \PY{n}{splines}\PY{o}{=}\PY{l+s+s1}{\PYZsq{}}\PY{l+s+s1}{ortho}\PY{l+s+s1}{\PYZsq{}}\PY{p}{,}
    \PY{p}{)}\PY{p}{,}
    \PY{n}{node\PYZus{}attr}\PY{o}{=}\PY{n+nb}{dict}\PY{p}{(}
        \PY{n}{shape}\PY{o}{=}\PY{l+s+s1}{\PYZsq{}}\PY{l+s+s1}{box}\PY{l+s+s1}{\PYZsq{}}\PY{p}{,}
        \PY{n}{style}\PY{o}{=}\PY{l+s+s1}{\PYZsq{}}\PY{l+s+s1}{rounded,filled}\PY{l+s+s1}{\PYZsq{}}\PY{p}{,}
        \PY{n}{fillcolor}\PY{o}{=}\PY{l+s+s1}{\PYZsq{}}\PY{l+s+s1}{\PYZsh{}E9F1F7}\PY{l+s+s1}{\PYZsq{}}\PY{p}{,}
        \PY{n}{fontname}\PY{o}{=}\PY{l+s+s1}{\PYZsq{}}\PY{l+s+s1}{Helvetica}\PY{l+s+s1}{\PYZsq{}}\PY{p}{,}
        \PY{n}{fontsize}\PY{o}{=}\PY{l+s+s1}{\PYZsq{}}\PY{l+s+s1}{12}\PY{l+s+s1}{\PYZsq{}}\PY{p}{,}
        \PY{n}{color}\PY{o}{=}\PY{l+s+s1}{\PYZsq{}}\PY{l+s+s1}{\PYZsh{}333333}\PY{l+s+s1}{\PYZsq{}}\PY{p}{,}
    \PY{p}{)}\PY{p}{,}
    \PY{n}{edge\PYZus{}attr}\PY{o}{=}\PY{n+nb}{dict}\PY{p}{(}
        \PY{n}{arrowhead}\PY{o}{=}\PY{l+s+s1}{\PYZsq{}}\PY{l+s+s1}{vee}\PY{l+s+s1}{\PYZsq{}}\PY{p}{,}
        \PY{n}{arrowtail}\PY{o}{=}\PY{l+s+s1}{\PYZsq{}}\PY{l+s+s1}{none}\PY{l+s+s1}{\PYZsq{}}\PY{p}{,}
        \PY{n}{fontname}\PY{o}{=}\PY{l+s+s1}{\PYZsq{}}\PY{l+s+s1}{Helvetica}\PY{l+s+s1}{\PYZsq{}}\PY{p}{,}
        \PY{n}{fontsize}\PY{o}{=}\PY{l+s+s1}{\PYZsq{}}\PY{l+s+s1}{10}\PY{l+s+s1}{\PYZsq{}}\PY{p}{,}
        \PY{n}{color}\PY{o}{=}\PY{l+s+s1}{\PYZsq{}}\PY{l+s+s1}{\PYZsh{}555555}\PY{l+s+s1}{\PYZsq{}}\PY{p}{,}
    \PY{p}{)}\PY{p}{,}
\PY{p}{)}

\PY{n}{g} \PY{o}{=} \PY{n}{graphviz}\PY{o}{.}\PY{n}{Digraph}\PY{p}{(}\PY{o}{*}\PY{o}{*}\PY{n}{gv\PYZus{}settings}\PY{p}{)}

\PY{n}{title\PYZus{}style} \PY{o}{=} \PY{n+nb}{dict}\PY{p}{(}\PY{n}{fontname}\PY{o}{=}\PY{l+s+s1}{\PYZsq{}}\PY{l+s+s1}{Helvetica\PYZhy{}Bold}\PY{l+s+s1}{\PYZsq{}}\PY{p}{,} \PY{n}{fontsize}\PY{o}{=}\PY{l+s+s1}{\PYZsq{}}\PY{l+s+s1}{16}\PY{l+s+s1}{\PYZsq{}}\PY{p}{,} \PY{n}{shape}\PY{o}{=}\PY{l+s+s1}{\PYZsq{}}\PY{l+s+s1}{plaintext}\PY{l+s+s1}{\PYZsq{}}\PY{p}{)}
\PY{n}{hidden\PYZus{}style} \PY{o}{=} \PY{n+nb}{dict}\PY{p}{(}\PY{n}{fillcolor}\PY{o}{=}\PY{l+s+s1}{\PYZsq{}}\PY{l+s+s1}{white}\PY{l+s+s1}{\PYZsq{}}\PY{p}{)}
\PY{n}{model\PYZus{}style} \PY{o}{=} \PY{n+nb}{dict}\PY{p}{(}\PY{n}{fillcolor}\PY{o}{=}\PY{l+s+s1}{\PYZsq{}}\PY{l+s+s1}{\PYZsh{}F7F1E9}\PY{l+s+s1}{\PYZsq{}}\PY{p}{)}

\PY{c+c1}{\PYZsh{} Create a node for the title}
\PY{n}{g}\PY{o}{.}\PY{n}{attr}\PY{p}{(}
    \PY{n}{label}\PY{o}{=}\PY{l+s+s2}{\PYZdq{}}\PY{l+s+s2}{Inference\PYZhy{}Based Flight Data Reconstruction}\PY{l+s+s2}{\PYZdq{}}\PY{p}{,}
    \PY{n}{labelloc}\PY{o}{=}\PY{l+s+s2}{\PYZdq{}}\PY{l+s+s2}{t}\PY{l+s+s2}{\PYZdq{}}\PY{p}{,}
    \PY{n}{fontname}\PY{o}{=}\PY{l+s+s1}{\PYZsq{}}\PY{l+s+s1}{Helvetica\PYZhy{}Bold}\PY{l+s+s1}{\PYZsq{}}\PY{p}{,} \PY{n}{fontsize}\PY{o}{=}\PY{l+s+s1}{\PYZsq{}}\PY{l+s+s1}{18}\PY{l+s+s1}{\PYZsq{}}\PY{p}{,} \PY{n}{shape}\PY{o}{=}\PY{l+s+s1}{\PYZsq{}}\PY{l+s+s1}{plaintext}\PY{l+s+s1}{\PYZsq{}}
\PY{p}{)}

\PY{n}{g}\PY{o}{.}\PY{n}{node}\PY{p}{(}\PY{l+s+s2}{\PYZdq{}}\PY{l+s+s2}{true}\PY{l+s+s2}{\PYZdq{}}\PY{p}{,} \PY{n}{label}\PY{o}{=}\PY{l+s+s2}{\PYZdq{}}\PY{l+s+s2}{True data}\PY{l+s+s2}{\PYZdq{}}\PY{p}{,} \PY{o}{*}\PY{o}{*}\PY{n}{hidden\PYZus{}style}\PY{p}{)}
\PY{n}{g}\PY{o}{.}\PY{n}{node}\PY{p}{(}\PY{l+s+s2}{\PYZdq{}}\PY{l+s+s2}{sensor}\PY{l+s+s2}{\PYZdq{}}\PY{p}{,} \PY{n}{label}\PY{o}{=}\PY{l+s+s2}{\PYZdq{}}\PY{l+s+s2}{Sensor data}\PY{l+s+s2}{\PYZdq{}}\PY{p}{)}
\PY{n}{g}\PY{o}{.}\PY{n}{node}\PY{p}{(}\PY{l+s+s2}{\PYZdq{}}\PY{l+s+s2}{noise}\PY{l+s+s2}{\PYZdq{}}\PY{p}{,} \PY{n}{label}\PY{o}{=}\PY{l+s+s2}{\PYZdq{}}\PY{l+s+s2}{True noise}\PY{l+s+s2}{\PYZdq{}}\PY{p}{,} \PY{o}{*}\PY{o}{*}\PY{n}{hidden\PYZus{}style}\PY{p}{)}
\PY{n}{g}\PY{o}{.}\PY{n}{node}\PY{p}{(}\PY{l+s+s2}{\PYZdq{}}\PY{l+s+s2}{estimate}\PY{l+s+s2}{\PYZdq{}}\PY{p}{,} \PY{n}{label}\PY{o}{=}\PY{l+s+s2}{\PYZdq{}}\PY{l+s+s2}{Estimate of}\PY{l+s+se}{\PYZbs{}n}\PY{l+s+s2}{true data}\PY{l+s+s2}{\PYZdq{}}\PY{p}{)}
\PY{n}{g}\PY{o}{.}\PY{n}{node}\PY{p}{(}\PY{l+s+s2}{\PYZdq{}}\PY{l+s+s2}{noise\PYZus{}estimate}\PY{l+s+s2}{\PYZdq{}}\PY{p}{,} \PY{n}{label}\PY{o}{=}\PY{l+s+s2}{\PYZdq{}}\PY{l+s+s2}{(Explicit) Estimate of noise,}\PY{l+s+se}{\PYZbs{}n}\PY{l+s+s2}{computed from sensor data}\PY{l+s+s2}{\PYZdq{}}\PY{p}{)}

\PY{n}{g}\PY{o}{.}\PY{n}{edge}\PY{p}{(}\PY{l+s+s2}{\PYZdq{}}\PY{l+s+s2}{true}\PY{l+s+s2}{\PYZdq{}}\PY{p}{,} \PY{l+s+s2}{\PYZdq{}}\PY{l+s+s2}{sensor}\PY{l+s+s2}{\PYZdq{}}\PY{p}{,} \PY{n}{style}\PY{o}{=}\PY{l+s+s1}{\PYZsq{}}\PY{l+s+s1}{dashed}\PY{l+s+s1}{\PYZsq{}}\PY{p}{)}
\PY{n}{g}\PY{o}{.}\PY{n}{edge}\PY{p}{(}\PY{l+s+s2}{\PYZdq{}}\PY{l+s+s2}{noise}\PY{l+s+s2}{\PYZdq{}}\PY{p}{,} \PY{l+s+s2}{\PYZdq{}}\PY{l+s+s2}{sensor}\PY{l+s+s2}{\PYZdq{}}\PY{p}{,} \PY{n}{style}\PY{o}{=}\PY{l+s+s1}{\PYZsq{}}\PY{l+s+s1}{dashed}\PY{l+s+s1}{\PYZsq{}}\PY{p}{)}

\PY{n}{g}\PY{o}{.}\PY{n}{edge}\PY{p}{(}\PY{l+s+s2}{\PYZdq{}}\PY{l+s+s2}{sensor}\PY{l+s+s2}{\PYZdq{}}\PY{p}{,} \PY{l+s+s2}{\PYZdq{}}\PY{l+s+s2}{estimate}\PY{l+s+s2}{\PYZdq{}}\PY{p}{)}
\PY{n}{g}\PY{o}{.}\PY{n}{edge}\PY{p}{(}\PY{l+s+s2}{\PYZdq{}}\PY{l+s+s2}{sensor}\PY{l+s+s2}{\PYZdq{}}\PY{p}{,} \PY{l+s+s2}{\PYZdq{}}\PY{l+s+s2}{noise\PYZus{}estimate}\PY{l+s+s2}{\PYZdq{}}\PY{p}{)}

\PY{n}{g}\PY{o}{.}\PY{n}{node}\PY{p}{(}\PY{l+s+s2}{\PYZdq{}}\PY{l+s+s2}{model\PYZus{}unsolved}\PY{l+s+s2}{\PYZdq{}}\PY{p}{,} \PY{n}{label}\PY{o}{=}\PY{l+s+s2}{\PYZdq{}}\PY{l+s+s2}{Aerodynamic / Propulsive model}\PY{l+s+se}{\PYZbs{}n}\PY{l+s+s2}{with unknown coefficients}\PY{l+s+s2}{\PYZdq{}}\PY{p}{,} \PY{o}{*}\PY{o}{*}\PY{n}{model\PYZus{}style}\PY{p}{)}
\PY{n}{g}\PY{o}{.}\PY{n}{node}\PY{p}{(}\PY{l+s+s2}{\PYZdq{}}\PY{l+s+s2}{newton}\PY{l+s+s2}{\PYZdq{}}\PY{p}{,} \PY{n}{label}\PY{o}{=}\PY{l+s+s2}{\PYZdq{}}\PY{l+s+s2}{Newtonian dynamics}\PY{l+s+se}{\PYZbs{}n}\PY{l+s+s2}{for unsteady corrections}\PY{l+s+s2}{\PYZdq{}}\PY{p}{,} \PY{o}{*}\PY{o}{*}\PY{n}{model\PYZus{}style}\PY{p}{)}

\PY{n}{g}\PY{o}{.}\PY{n}{node}\PY{p}{(}\PY{l+s+s2}{\PYZdq{}}\PY{l+s+s2}{energy\PYZus{}balance}\PY{l+s+s2}{\PYZdq{}}\PY{p}{,} \PY{n}{label}\PY{o}{=}\PY{l+s+s2}{\PYZdq{}}\PY{l+s+s2}{Energy\PYZhy{}balance reconstruction}\PY{l+s+se}{\PYZbs{}n}\PY{l+s+s2}{(compute energy residuals)}\PY{l+s+s2}{\PYZdq{}}\PY{p}{)}

\PY{n}{g}\PY{o}{.}\PY{n}{node}\PY{p}{(}\PY{l+s+s2}{\PYZdq{}}\PY{l+s+s2}{opti}\PY{l+s+s2}{\PYZdq{}}\PY{p}{,} \PY{n}{label}\PY{o}{=}\PY{l+s+s2}{\PYZdq{}}\PY{l+s+s2}{Optimization}\PY{l+s+se}{\PYZbs{}n}\PY{l+s+s2}{(minimize residuals)}\PY{l+s+s2}{\PYZdq{}}\PY{p}{,} \PY{o}{*}\PY{o}{*}\PY{n}{model\PYZus{}style}\PY{p}{)}

\PY{n}{g}\PY{o}{.}\PY{n}{node}\PY{p}{(}\PY{l+s+s2}{\PYZdq{}}\PY{l+s+s2}{model\PYZus{}solved}\PY{l+s+s2}{\PYZdq{}}\PY{p}{,} \PY{n}{label}\PY{o}{=}\PY{l+s+s2}{\PYZdq{}}\PY{l+s+s2}{Aerodynamic / Propulsive model}\PY{l+s+se}{\PYZbs{}n}\PY{l+s+s2}{with known coefficients}\PY{l+s+s2}{\PYZdq{}}\PY{p}{,} \PY{o}{*}\PY{o}{*}\PY{n}{model\PYZus{}style}\PY{p}{)}

\PY{n}{g}\PY{o}{.}\PY{n}{edge}\PY{p}{(}\PY{l+s+s2}{\PYZdq{}}\PY{l+s+s2}{newton}\PY{l+s+s2}{\PYZdq{}}\PY{p}{,} \PY{l+s+s2}{\PYZdq{}}\PY{l+s+s2}{energy\PYZus{}balance}\PY{l+s+s2}{\PYZdq{}}\PY{p}{)}
\PY{n}{g}\PY{o}{.}\PY{n}{edge}\PY{p}{(}\PY{l+s+s2}{\PYZdq{}}\PY{l+s+s2}{estimate}\PY{l+s+s2}{\PYZdq{}}\PY{p}{,} \PY{l+s+s2}{\PYZdq{}}\PY{l+s+s2}{energy\PYZus{}balance}\PY{l+s+s2}{\PYZdq{}}\PY{p}{)}
\PY{n}{g}\PY{o}{.}\PY{n}{edge}\PY{p}{(}\PY{l+s+s2}{\PYZdq{}}\PY{l+s+s2}{model\PYZus{}unsolved}\PY{l+s+s2}{\PYZdq{}}\PY{p}{,} \PY{l+s+s2}{\PYZdq{}}\PY{l+s+s2}{energy\PYZus{}balance}\PY{l+s+s2}{\PYZdq{}}\PY{p}{)}

\PY{n}{g}\PY{o}{.}\PY{n}{edge}\PY{p}{(}\PY{l+s+s2}{\PYZdq{}}\PY{l+s+s2}{energy\PYZus{}balance}\PY{l+s+s2}{\PYZdq{}}\PY{p}{,} \PY{l+s+s2}{\PYZdq{}}\PY{l+s+s2}{opti}\PY{l+s+s2}{\PYZdq{}}\PY{p}{)}

\PY{n}{g}\PY{o}{.}\PY{n}{edge}\PY{p}{(}\PY{l+s+s2}{\PYZdq{}}\PY{l+s+s2}{opti}\PY{l+s+s2}{\PYZdq{}}\PY{p}{,} \PY{l+s+s2}{\PYZdq{}}\PY{l+s+s2}{model\PYZus{}solved}\PY{l+s+s2}{\PYZdq{}}\PY{p}{)}

\PY{n}{g}
\end{Verbatim}
\end{tcolorbox}
 
            
\prompt{Out}{outcolor}{1}{}
    
    \begin{center}
    \adjustimage{max size={0.9\linewidth}{0.9\paperheight}}{output_8_0.pdf}
    \end{center}
    { \hspace*{\fill} \\}
    

    A fair point of hesitancy here might be the fact that applying these
corrections to recover useful insights from noisy, limited, unsteady
data takes a significant amount of algorithmic development and
mathematical complexity. However, algorithms are cheap, and more
importantly, \emph{repeatable} - so, once a workflow is established, it
can easily be applied to a large number of flight test campaigns. By
contrast, flight-test hours and improved instrumentation is
extraordinarily expensive - so, it is worth investing in the development
of algorithms that can extract the maximum amount of information from
the data that we do collect.

    \hypertarget{a-minimal-flight-test-dataset}{%
\subsection{A Minimal Flight Test
Dataset}\label{a-minimal-flight-test-dataset}}

    To illustrate our proposed flight test procedure, this manuscript will
use example data from a flight test conducted during MIT 16.821: Flight
Vehicle Development, a senior-level aircraft design/build/fly capstone
course at MIT AeroAstro. The flown aircraft, named \emph{Solar Surfer},
is a remote-controlled solar-electric seaplane design with a 14-foot
wingspan. The as-flown all-up mass is 9.4 kg.

    

    General specifications of the aircraft can be found in the early design
drawing that is reproduced below. While this drawing does not reflect
as-built weights or various planform adjustments that were made during
preliminary design and construction, the overall configuration and
performance numbers are representative of the aircraft that was flown.

    

    The aircraft was flown in the vicinity of the Charles river basin near
Cambridge, MA. Five tests were conducted on the morning of May 3, 2023,
of which two were airborne flight tests. The final airborne flight test
lasted approximately 260 seconds (4.3 minutes), beginning and ending
with a successful water takeoff and landing.

A simple racetrack-like pattern was flown, as shown in the figure below.
Winds were calm at roughly 1.5 m/s from the south.

\begin{verbatim}
<img src="./assets/flight3-circuit.png" alt="path" style="max-width:60%;">
\end{verbatim}

The following relevant sensors were on-board the aircraft: * A 9-DOF IMU
(accelerometer, gyroscope, magnetometer), logging at 2 Hz * A
pitot-static probe, logging at 5 Hz * A barometric altimeter, logging at
5 Hz * A GPS, logging at approximately 2 Hz, with a few delayed or
missed samples * A battery voltage and current monitor, logging at 5 Hz

    Below is an example view of the raw data from the sensors during the
flight:

    \begin{tcolorbox}[breakable, size=fbox, boxrule=1pt, pad at break*=1mm,colback=cellbackground, colframe=cellborder]
\prompt{In}{incolor}{2}{\boxspacing}
\begin{Verbatim}[commandchars=\\\{\}]
\PY{c+c1}{\PYZsh{} Set up general imports and locate data sources}

\PY{k+kn}{import} \PY{n+nn}{aerosandbox}\PY{n+nn}{.}\PY{n+nn}{numpy} \PY{k}{as} \PY{n+nn}{np}
\PY{k+kn}{import} \PY{n+nn}{matplotlib}\PY{n+nn}{.}\PY{n+nn}{pyplot} \PY{k}{as} \PY{n+nn}{plt}
\PY{k+kn}{import} \PY{n+nn}{aerosandbox}\PY{n+nn}{.}\PY{n+nn}{tools}\PY{n+nn}{.}\PY{n+nn}{pretty\PYZus{}plots} \PY{k}{as} \PY{n+nn}{p}
\PY{k+kn}{import} \PY{n+nn}{pandas} \PY{k}{as} \PY{n+nn}{pd}
\PY{k+kn}{from} \PY{n+nn}{scipy} \PY{k+kn}{import} \PY{n}{interpolate}

\PY{n}{timestamp\PYZus{}0} \PY{o}{=} \PY{l+m+mi}{539330432}

\PY{n}{raw\PYZus{}time\PYZus{}takeoff} \PY{o}{=} \PY{l+m+mi}{577}
\PY{n}{raw\PYZus{}time\PYZus{}landing} \PY{o}{=} \PY{l+m+mi}{840}

\PY{n}{t\PYZus{}max} \PY{o}{=} \PY{n}{raw\PYZus{}time\PYZus{}landing} \PY{o}{\PYZhy{}} \PY{n}{raw\PYZus{}time\PYZus{}takeoff}

\PY{n}{data\PYZus{}sources} \PY{o}{=} \PY{p}{\PYZob{}}
    \PY{l+s+s2}{\PYZdq{}}\PY{l+s+s2}{airspeed}\PY{l+s+s2}{\PYZdq{}}  \PY{p}{:} \PY{p}{(}\PY{l+s+s2}{\PYZdq{}}\PY{l+s+s2}{./data/flight3\PYZus{}airspeed\PYZus{}validated\PYZus{}0.csv}\PY{l+s+s2}{\PYZdq{}}\PY{p}{,} \PY{l+s+s2}{\PYZdq{}}\PY{l+s+s2}{calibrated\PYZus{}airspeed\PYZus{}m\PYZus{}s}\PY{l+s+s2}{\PYZdq{}}\PY{p}{)}\PY{p}{,}
    \PY{l+s+s2}{\PYZdq{}}\PY{l+s+s2}{barometer}\PY{l+s+s2}{\PYZdq{}} \PY{p}{:} \PY{p}{(}\PY{l+s+s2}{\PYZdq{}}\PY{l+s+s2}{./data/flight3\PYZus{}sensor\PYZus{}baro\PYZus{}0.csv}\PY{l+s+s2}{\PYZdq{}}\PY{p}{,} \PY{l+s+s2}{\PYZdq{}}\PY{l+s+s2}{pressure}\PY{l+s+s2}{\PYZdq{}}\PY{p}{)}\PY{p}{,}
    \PY{l+s+s2}{\PYZdq{}}\PY{l+s+s2}{baro\PYZus{}alt}\PY{l+s+s2}{\PYZdq{}}  \PY{p}{:} \PY{p}{(}\PY{l+s+s2}{\PYZdq{}}\PY{l+s+s2}{./data/flight3\PYZus{}vehicle\PYZus{}air\PYZus{}data\PYZus{}0.csv}\PY{l+s+s2}{\PYZdq{}}\PY{p}{,} \PY{l+s+s2}{\PYZdq{}}\PY{l+s+s2}{baro\PYZus{}alt\PYZus{}meter}\PY{l+s+s2}{\PYZdq{}}\PY{p}{)}\PY{p}{,}
    \PY{l+s+s2}{\PYZdq{}}\PY{l+s+s2}{gps\PYZus{}alt\PYZus{}mm}\PY{l+s+s2}{\PYZdq{}}\PY{p}{:} \PY{p}{(}\PY{l+s+s2}{\PYZdq{}}\PY{l+s+s2}{./data/flight3\PYZus{}vehicle\PYZus{}gps\PYZus{}position\PYZus{}0.csv}\PY{l+s+s2}{\PYZdq{}}\PY{p}{,} \PY{l+s+s2}{\PYZdq{}}\PY{l+s+s2}{alt}\PY{l+s+s2}{\PYZdq{}}\PY{p}{)}\PY{p}{,}
    \PY{l+s+s2}{\PYZdq{}}\PY{l+s+s2}{voltage}\PY{l+s+s2}{\PYZdq{}}   \PY{p}{:} \PY{p}{(}\PY{l+s+s2}{\PYZdq{}}\PY{l+s+s2}{./data/flight3\PYZus{}battery\PYZus{}status\PYZus{}1.csv}\PY{l+s+s2}{\PYZdq{}}\PY{p}{,} \PY{l+s+s2}{\PYZdq{}}\PY{l+s+s2}{voltage\PYZus{}v}\PY{l+s+s2}{\PYZdq{}}\PY{p}{)}\PY{p}{,}
    \PY{l+s+s2}{\PYZdq{}}\PY{l+s+s2}{current}\PY{l+s+s2}{\PYZdq{}}   \PY{p}{:} \PY{p}{(}\PY{l+s+s2}{\PYZdq{}}\PY{l+s+s2}{./data/flight3\PYZus{}battery\PYZus{}status\PYZus{}1.csv}\PY{l+s+s2}{\PYZdq{}}\PY{p}{,} \PY{l+s+s2}{\PYZdq{}}\PY{l+s+s2}{current\PYZus{}a}\PY{l+s+s2}{\PYZdq{}}\PY{p}{)}\PY{p}{,}
\PY{p}{\PYZcb{}}
\end{Verbatim}
\end{tcolorbox}

    \begin{tcolorbox}[breakable, size=fbox, boxrule=1pt, pad at break*=1mm,colback=cellbackground, colframe=cellborder]
\prompt{In}{incolor}{3}{\boxspacing}
\begin{Verbatim}[commandchars=\\\{\}]
\PY{n}{fig}\PY{p}{,} \PY{n}{ax} \PY{o}{=} \PY{n}{plt}\PY{o}{.}\PY{n}{subplots}\PY{p}{(}\PY{n}{figsize}\PY{o}{=}\PY{p}{(}\PY{l+m+mi}{7}\PY{p}{,} \PY{l+m+mf}{6.4}\PY{p}{)}\PY{p}{)}

\PY{n}{speed\PYZus{}color} \PY{o}{=} \PY{l+s+s2}{\PYZdq{}}\PY{l+s+s2}{darkgreen}\PY{l+s+s2}{\PYZdq{}}
\PY{n}{alt\PYZus{}color} \PY{o}{=} \PY{l+s+s2}{\PYZdq{}}\PY{l+s+s2}{navy}\PY{l+s+s2}{\PYZdq{}}
\PY{n}{voltage\PYZus{}color} \PY{o}{=} \PY{l+s+s2}{\PYZdq{}}\PY{l+s+s2}{red}\PY{l+s+s2}{\PYZdq{}}
\PY{n}{current\PYZus{}color} \PY{o}{=} \PY{l+s+s2}{\PYZdq{}}\PY{l+s+s2}{darkgoldenrod}\PY{l+s+s2}{\PYZdq{}}


\PY{k}{def} \PY{n+nf}{simple\PYZus{}read}\PY{p}{(}\PY{n}{name}\PY{p}{)}\PY{p}{:}
    \PY{n}{source} \PY{o}{=} \PY{n}{data\PYZus{}sources}\PY{p}{[}\PY{n}{name}\PY{p}{]}\PY{p}{[}\PY{l+m+mi}{0}\PY{p}{]}
    \PY{n}{colname} \PY{o}{=} \PY{n}{data\PYZus{}sources}\PY{p}{[}\PY{n}{name}\PY{p}{]}\PY{p}{[}\PY{l+m+mi}{1}\PY{p}{]}

    \PY{n}{df} \PY{o}{=} \PY{n}{pd}\PY{o}{.}\PY{n}{read\PYZus{}csv}\PY{p}{(}\PY{n}{source}\PY{p}{)}

    \PY{n}{raw\PYZus{}time} \PY{o}{=} \PY{p}{(}\PY{n}{df}\PY{p}{[}\PY{l+s+s2}{\PYZdq{}}\PY{l+s+s2}{timestamp}\PY{l+s+s2}{\PYZdq{}}\PY{p}{]}\PY{o}{.}\PY{n}{values} \PY{o}{\PYZhy{}} \PY{n}{timestamp\PYZus{}0}\PY{p}{)} \PY{o}{/} \PY{l+m+mf}{1e6}
    \PY{n}{data} \PY{o}{=} \PY{n}{df}\PY{p}{[}\PY{n}{colname}\PY{p}{]}\PY{o}{.}\PY{n}{values}

    \PY{n}{mask} \PY{o}{=} \PY{p}{(}\PY{n}{raw\PYZus{}time} \PY{o}{\PYZgt{}} \PY{n}{raw\PYZus{}time\PYZus{}takeoff}\PY{p}{)} \PY{o}{\PYZam{}} \PY{p}{(}\PY{n}{raw\PYZus{}time} \PY{o}{\PYZlt{}} \PY{n}{raw\PYZus{}time\PYZus{}landing}\PY{p}{)}

    \PY{n}{time} \PY{o}{=} \PY{n}{raw\PYZus{}time}\PY{p}{[}\PY{n}{mask}\PY{p}{]} \PY{o}{\PYZhy{}} \PY{n}{raw\PYZus{}time\PYZus{}takeoff}
    \PY{n}{data} \PY{o}{=} \PY{n}{data}\PY{p}{[}\PY{n}{mask}\PY{p}{]}

    \PY{k}{return} \PY{n}{time}\PY{p}{,} \PY{n}{data}


\PY{n}{plt}\PY{o}{.}\PY{n}{plot}\PY{p}{(}
    \PY{o}{*}\PY{n}{simple\PYZus{}read}\PY{p}{(}\PY{l+s+s2}{\PYZdq{}}\PY{l+s+s2}{airspeed}\PY{l+s+s2}{\PYZdq{}}\PY{p}{)}\PY{p}{,}
    \PY{n}{color}\PY{o}{=}\PY{n}{speed\PYZus{}color}\PY{p}{,} \PY{n}{alpha}\PY{o}{=}\PY{l+m+mf}{0.5}\PY{p}{,}
    \PY{n}{label}\PY{o}{=}\PY{l+s+s2}{\PYZdq{}}\PY{l+s+s2}{Airspeed [m/s]}\PY{l+s+s2}{\PYZdq{}}
\PY{p}{)}
\PY{n}{baro\PYZus{}data} \PY{o}{=} \PY{n}{simple\PYZus{}read}\PY{p}{(}\PY{l+s+s2}{\PYZdq{}}\PY{l+s+s2}{baro\PYZus{}alt}\PY{l+s+s2}{\PYZdq{}}\PY{p}{)}
\PY{n}{baro\PYZus{}altitude\PYZus{}at\PYZus{}takeoff} \PY{o}{=} \PY{n}{baro\PYZus{}data}\PY{p}{[}\PY{l+m+mi}{1}\PY{p}{]}\PY{p}{[}\PY{l+m+mi}{0}\PY{p}{]}
\PY{n}{plt}\PY{o}{.}\PY{n}{plot}\PY{p}{(}
    \PY{n}{baro\PYZus{}data}\PY{p}{[}\PY{l+m+mi}{0}\PY{p}{]}\PY{p}{,} \PY{n}{baro\PYZus{}data}\PY{p}{[}\PY{l+m+mi}{1}\PY{p}{]} \PY{o}{\PYZhy{}} \PY{n}{baro\PYZus{}altitude\PYZus{}at\PYZus{}takeoff}\PY{p}{,}
    \PY{n}{color}\PY{o}{=}\PY{n}{alt\PYZus{}color}\PY{p}{,} \PY{n}{alpha}\PY{o}{=}\PY{l+m+mf}{0.5}\PY{p}{,}
    \PY{n}{label}\PY{o}{=}\PY{l+s+s2}{\PYZdq{}}\PY{l+s+s2}{Altitude AGL (from barometer) [m]}\PY{l+s+s2}{\PYZdq{}}
\PY{p}{)}
\PY{n}{plt}\PY{o}{.}\PY{n}{plot}\PY{p}{(}
    \PY{o}{*}\PY{n}{simple\PYZus{}read}\PY{p}{(}\PY{l+s+s2}{\PYZdq{}}\PY{l+s+s2}{voltage}\PY{l+s+s2}{\PYZdq{}}\PY{p}{)}\PY{p}{,}
    \PY{n}{color}\PY{o}{=}\PY{n}{voltage\PYZus{}color}\PY{p}{,} \PY{n}{alpha}\PY{o}{=}\PY{l+m+mf}{0.5}\PY{p}{,}
    \PY{n}{label}\PY{o}{=}\PY{l+s+s2}{\PYZdq{}}\PY{l+s+s2}{Battery Voltage [V]}\PY{l+s+s2}{\PYZdq{}}
\PY{p}{)}
\PY{n}{plt}\PY{o}{.}\PY{n}{plot}\PY{p}{(}
    \PY{o}{*}\PY{n}{simple\PYZus{}read}\PY{p}{(}\PY{l+s+s2}{\PYZdq{}}\PY{l+s+s2}{current}\PY{l+s+s2}{\PYZdq{}}\PY{p}{)}\PY{p}{,}
    \PY{n}{color}\PY{o}{=}\PY{n}{current\PYZus{}color}\PY{p}{,} \PY{n}{alpha}\PY{o}{=}\PY{l+m+mf}{0.5}\PY{p}{,}
    \PY{n}{label}\PY{o}{=}\PY{l+s+s2}{\PYZdq{}}\PY{l+s+s2}{Battery Current [A]}\PY{l+s+s2}{\PYZdq{}}
\PY{p}{)}

\PY{n}{plt}\PY{o}{.}\PY{n}{xlabel}\PY{p}{(}\PY{l+s+s2}{\PYZdq{}}\PY{l+s+s2}{Time after Takeoff [seconds]}\PY{l+s+s2}{\PYZdq{}}\PY{p}{)}
\PY{n}{plt}\PY{o}{.}\PY{n}{ylabel}\PY{p}{(}\PY{l+s+s2}{\PYZdq{}}\PY{l+s+s2}{Value (see legend)}\PY{l+s+s2}{\PYZdq{}}\PY{p}{)}
\PY{n}{plt}\PY{o}{.}\PY{n}{title}\PY{p}{(}\PY{l+s+s2}{\PYZdq{}}\PY{l+s+s2}{Raw Sensor Data}\PY{l+s+s2}{\PYZdq{}}\PY{p}{)}
\PY{n}{p}\PY{o}{.}\PY{n}{set\PYZus{}ticks}\PY{p}{(}\PY{l+m+mi}{20}\PY{p}{,} \PY{l+m+mi}{5}\PY{p}{,} \PY{l+m+mi}{5}\PY{p}{,} \PY{l+m+mi}{1}\PY{p}{)}
\PY{n}{plt}\PY{o}{.}\PY{n}{xlim}\PY{p}{(}\PY{o}{\PYZhy{}}\PY{l+m+mi}{1}\PY{p}{,} \PY{n}{t\PYZus{}max} \PY{o}{+} \PY{l+m+mi}{1}\PY{p}{)}
\PY{n}{plt}\PY{o}{.}\PY{n}{ylim}\PY{p}{(}\PY{n}{bottom}\PY{o}{=}\PY{o}{\PYZhy{}}\PY{l+m+mi}{2}\PY{p}{)}
\PY{n}{p}\PY{o}{.}\PY{n}{show\PYZus{}plot}\PY{p}{(}
    \PY{n}{dpi}\PY{o}{=}\PY{l+m+mi}{300}
\PY{p}{)}
\end{Verbatim}
\end{tcolorbox}

    \begin{center}
    \adjustimage{max size={0.9\linewidth}{0.9\paperheight}}{output_18_0.png}
    \end{center}
    { \hspace*{\fill} \\}
    
    Overall, the data is reasonable, and we can clearly identify takeoff and
landing by the strong increase in barometric altimeter noise near
\(t=0\ \rm sec\) and \(t=263\ \rm sec\). We can also identify several
flight phases: a high-speed pass that was conducted near
\(t=150\ \rm sec\), and gliding periods near \(t=40\ \rm sec\),
\(t=210\ \rm sec\), and \(t=250\ \rm sec\).

There are two primary problems with the data we have. Firstly, all of
the data sources, and in particular the airspeed sensor, yield noisy
measurements. The unsteady corrections that we will later implement
require taking the derivative of this data; this is a problem, because
taking the derivative of noisy data tends to amplify the noise further.

Secondly, the total amount of data is extremely limited, both in
duration and sample rate. For example, there are only
\textasciitilde1,200 data points \textbf{in total} for each of the
plotted sensor traces, and this is before eliminating noise. After
eliminating noise (which effectively acts as a low-pass filter on our
data, as noise tends to be relatively high-frequency), the effective
amount of information we have from the sensors is extremely low. While
1,200 data points might be sufficient to characterize the aircraft
flight performance at a single operating point (i.e., a single power
setting and airspeed), it is typically a vastly insufficient quantity of
data to characterize any substantial portion of the flight envelope.

Consider that even the most sophisticated techniques for system
identification via statistical inference require the system to be
observable in some form - no amount of math can recover a signal that is
simply not present in the data. This is difficult, because after noise
removal, the data really no more than perhaps a half-dozen ``system
excitations'' from which we can infer aircraft performance. As a more
intuitive analogy, consider that even the world's best test pilot can't
accurately characterize the handling qualities of the aircraft without
observing at least a few flight maneuvers.

    For these reasons, this dataset makes a good example case for what kinds
of flight data reconstruction are possible with very limited, noisy data
using statistical inference techniques and corrections from unsteady
flight physics.

    \hypertarget{before-picture}{%
\subsubsection{``Before'' Picture}\label{before-picture}}

    To give us an idea of what we're working with, let's

    \begin{tcolorbox}[breakable, size=fbox, boxrule=1pt, pad at break*=1mm,colback=cellbackground, colframe=cellborder]
\prompt{In}{incolor}{4}{\boxspacing}
\begin{Verbatim}[commandchars=\\\{\}]
\PY{n}{\PYZus{}}\PY{p}{,} \PY{n}{airspeed} \PY{o}{=} \PY{n}{simple\PYZus{}read}\PY{p}{(}\PY{l+s+s2}{\PYZdq{}}\PY{l+s+s2}{airspeed}\PY{l+s+s2}{\PYZdq{}}\PY{p}{)}
\PY{n}{\PYZus{}}\PY{p}{,} \PY{n}{baro\PYZus{}alt} \PY{o}{=} \PY{n}{simple\PYZus{}read}\PY{p}{(}\PY{l+s+s2}{\PYZdq{}}\PY{l+s+s2}{baro\PYZus{}alt}\PY{l+s+s2}{\PYZdq{}}\PY{p}{)}
\PY{n}{\PYZus{}}\PY{p}{,} \PY{n}{voltage} \PY{o}{=} \PY{n}{simple\PYZus{}read}\PY{p}{(}\PY{l+s+s2}{\PYZdq{}}\PY{l+s+s2}{voltage}\PY{l+s+s2}{\PYZdq{}}\PY{p}{)}
\PY{n}{\PYZus{}}\PY{p}{,} \PY{n}{current} \PY{o}{=} \PY{n}{simple\PYZus{}read}\PY{p}{(}\PY{l+s+s2}{\PYZdq{}}\PY{l+s+s2}{current}\PY{l+s+s2}{\PYZdq{}}\PY{p}{)}

\PY{k+kn}{from} \PY{n+nn}{aerosandbox}\PY{n+nn}{.}\PY{n+nn}{tools}\PY{n+nn}{.}\PY{n+nn}{statistics} \PY{k+kn}{import} \PY{n}{time\PYZus{}series\PYZus{}uncertainty\PYZus{}quantification} \PY{k}{as} \PY{n}{tsuq}
\end{Verbatim}
\end{tcolorbox}

    \begin{tcolorbox}[breakable, size=fbox, boxrule=1pt, pad at break*=1mm,colback=cellbackground, colframe=cellborder]
\prompt{In}{incolor}{19}{\boxspacing}
\begin{Verbatim}[commandchars=\\\{\}]
\PY{n}{fig}\PY{p}{,} \PY{n}{ax} \PY{o}{=} \PY{n}{plt}\PY{o}{.}\PY{n}{subplots}\PY{p}{(}\PY{p}{)}

\PY{n}{np}\PY{o}{.}\PY{n}{savez}\PY{p}{(}\PY{l+s+s2}{\PYZdq{}}\PY{l+s+s2}{raw\PYZus{}data.npz}\PY{l+s+s2}{\PYZdq{}}\PY{p}{,} \PY{n}{airspeed}\PY{o}{=}\PY{n}{airspeed}\PY{p}{,} \PY{n}{baro\PYZus{}alt}\PY{o}{=}\PY{n}{baro\PYZus{}alt}\PY{p}{,} \PY{n}{voltage}\PY{o}{=}\PY{n}{voltage}\PY{p}{,} \PY{n}{current}\PY{o}{=}\PY{n}{current}\PY{p}{)}

\PY{n}{p}\PY{o}{.}\PY{n}{plot\PYZus{}with\PYZus{}bootstrapped\PYZus{}uncertainty}\PY{p}{(}
    \PY{n}{airspeed}\PY{p}{,} \PY{p}{(}\PY{n}{voltage}\PY{o}{*}\PY{n}{current}\PY{p}{)}\PY{p}{,}
    \PY{n}{x\PYZus{}stdev} \PY{o}{=} \PY{k+kc}{None}\PY{p}{,}
    \PY{n}{y\PYZus{}stdev} \PY{o}{=} \PY{k+kc}{None}\PY{p}{,}
    \PY{n}{ci}\PY{o}{=}\PY{p}{[}\PY{l+m+mf}{0.5}\PY{p}{,} \PY{l+m+mf}{0.75}\PY{p}{,} \PY{l+m+mf}{0.95}\PY{p}{]}\PY{p}{,}
    \PY{n}{color}\PY{o}{=}\PY{l+s+s2}{\PYZdq{}}\PY{l+s+s2}{red}\PY{l+s+s2}{\PYZdq{}}\PY{p}{,}
    \PY{n}{n\PYZus{}bootstraps}\PY{o}{=}\PY{l+m+mi}{100}\PY{p}{,}
\PY{p}{)}
\PY{n}{plt}\PY{o}{.}\PY{n}{ylim}\PY{p}{(}\PY{o}{\PYZhy{}}\PY{l+m+mi}{10}\PY{p}{,} \PY{l+m+mi}{800}\PY{p}{)}
\PY{n}{p}\PY{o}{.}\PY{n}{show\PYZus{}plot}\PY{p}{(}
    \PY{n}{xlabel}\PY{o}{=}\PY{l+s+s2}{\PYZdq{}}\PY{l+s+s2}{Airspeed [m/s]}\PY{l+s+s2}{\PYZdq{}}\PY{p}{,}
    \PY{n}{ylabel}\PY{o}{=}\PY{l+s+s2}{\PYZdq{}}\PY{l+s+s2}{Power [W]}\PY{l+s+s2}{\PYZdq{}}\PY{p}{,}
    \PY{n}{title}\PY{o}{=}\PY{l+s+s2}{\PYZdq{}}\PY{l+s+s2}{Raw Data}\PY{l+s+s2}{\PYZdq{}}\PY{p}{,}
    \PY{n}{dpi}\PY{o}{=}\PY{l+m+mi}{300}
\PY{p}{)}
\end{Verbatim}
\end{tcolorbox}

    \begin{Verbatim}[commandchars=\\\{\}]
Estimated x standard deviation: 0.26614729326103154
Estimated y standard deviation: 6.563188911834445
    \end{Verbatim}

    \begin{Verbatim}[commandchars=\\\{\}]
Bootstrapping: 100\%|██████████| 100/100 [00:00<00:00, 238.89 samples/s]
C:\textbackslash{}Users\textbackslash{}peter\textbackslash{}miniconda3\textbackslash{}lib\textbackslash{}site-packages\textbackslash{}numpy\textbackslash{}lib\textbackslash{}nanfunctions.py:1577:
RuntimeWarning: All-NaN slice encountered
  result = np.apply\_along\_axis(\_nanquantile\_1d, axis, a, q,
    \end{Verbatim}

    \begin{center}
    \adjustimage{max size={0.9\linewidth}{0.9\paperheight}}{output_24_2.png}
    \end{center}
    { \hspace*{\fill} \\}
    
    \hypertarget{sensor-data-pre-processing}{%
\subsection{Sensor Data
Pre-Processing}\label{sensor-data-pre-processing}}

    The first contribution here is to develop a (new?) means of
pre-processing the sensor data. Traditionally, noise is removed by
simple averaging; after this process is complete, any uncertainty in the
data is typically not considered, and the average is regarded as the
ground truth. This works fine for traditional steady analysis where
there is a wealth of data that we can average over.

However, in cases where data is limited, we're strongly encouraged to
find a way to extract some amount of information from unsteady data.
This is difficult, because it forces us to directly deal with the sensor
noise (and embed priors about the bias-variance tradeoff of data),
rather than hand-waving it away by averaging.

As an illustrative example, imagine that we wish to reconstruct a
\emph{truth} estimate of the airspeed from the example dataset. This
truth value is not directly observable, although data from an associated
airspeed sensor is. However, this sensor data has noise; hence, any
attempt to reconstruct the truth value must adopt a strategy to
separating the noise from the data.

However, there are (literally) an infinite number of possible strategies
that one can use to do this noise removal, depending on our embedded
assumptions about how sensor noise is entering our data-generating
process. For example, which of the following three possible curves best
estimates the true underlying state of the vehicle?

    \begin{tcolorbox}[breakable, size=fbox, boxrule=1pt, pad at break*=1mm,colback=cellbackground, colframe=cellborder]
\prompt{In}{incolor}{6}{\boxspacing}
\begin{Verbatim}[commandchars=\\\{\}]
\PY{n}{t}\PY{p}{,} \PY{n}{a} \PY{o}{=} \PY{n}{simple\PYZus{}read}\PY{p}{(}\PY{l+s+s2}{\PYZdq{}}\PY{l+s+s2}{airspeed}\PY{l+s+s2}{\PYZdq{}}\PY{p}{)}

\PY{n}{assumed\PYZus{}stdev} \PY{o}{=} \PY{p}{[}
    \PY{l+m+mi}{0}\PY{p}{,}
    \PY{n}{np}\PY{o}{.}\PY{n}{std}\PY{p}{(}\PY{n}{np}\PY{o}{.}\PY{n}{diff}\PY{p}{(}\PY{n}{np}\PY{o}{.}\PY{n}{diff}\PY{p}{(}\PY{n}{a}\PY{p}{)}\PY{p}{)}\PY{p}{)} \PY{o}{/} \PY{n}{np}\PY{o}{.}\PY{n}{sqrt}\PY{p}{(}\PY{l+m+mi}{6}\PY{p}{)}\PY{p}{,}
    \PY{l+m+mf}{0.8}
\PY{p}{]}
\PY{n}{colors} \PY{o}{=} \PY{p}{[}
    \PY{l+s+s2}{\PYZdq{}}\PY{l+s+s2}{tomato}\PY{l+s+s2}{\PYZdq{}}\PY{p}{,}
    \PY{l+s+s2}{\PYZdq{}}\PY{l+s+s2}{teal}\PY{l+s+s2}{\PYZdq{}}\PY{p}{,}
    \PY{l+s+s2}{\PYZdq{}}\PY{l+s+s2}{darkviolet}\PY{l+s+s2}{\PYZdq{}}
\PY{p}{]}
\PY{n}{names} \PY{o}{=} \PY{p}{[}
    \PY{l+s+s2}{\PYZdq{}}\PY{l+s+s2}{Overfitted}\PY{l+s+s2}{\PYZdq{}}\PY{p}{,}
    \PY{l+s+s2}{\PYZdq{}}\PY{l+s+s2}{Our Method}\PY{l+s+s2}{\PYZdq{}}\PY{p}{,}
    \PY{l+s+s2}{\PYZdq{}}\PY{l+s+s2}{Underfitted}\PY{l+s+s2}{\PYZdq{}}
\PY{p}{]}

\PY{n}{estimators} \PY{o}{=} \PY{p}{[}
    \PY{n}{interpolate}\PY{o}{.}\PY{n}{UnivariateSpline}\PY{p}{(}
        \PY{n}{x}\PY{o}{=}\PY{n}{t}\PY{p}{,}
        \PY{n}{y}\PY{o}{=}\PY{n}{a}\PY{p}{,}
        \PY{n}{k}\PY{o}{=}\PY{l+m+mi}{5}\PY{p}{,}
        \PY{n}{w}\PY{o}{=}\PY{l+m+mi}{1} \PY{o}{/} \PY{n}{s} \PY{o}{*} \PY{n}{np}\PY{o}{.}\PY{n}{ones\PYZus{}like}\PY{p}{(}\PY{n}{a}\PY{p}{)} \PY{k}{if} \PY{n}{s} \PY{o}{\PYZgt{}} \PY{l+m+mi}{0} \PY{k}{else} \PY{k+kc}{None}\PY{p}{,}
        \PY{n}{s}\PY{o}{=}\PY{n+nb}{len}\PY{p}{(}\PY{n}{a}\PY{p}{)} \PY{k}{if} \PY{n}{s} \PY{o}{\PYZgt{}} \PY{l+m+mi}{0} \PY{k}{else} \PY{l+m+mi}{0}\PY{p}{,}
        \PY{n}{check\PYZus{}finite}\PY{o}{=}\PY{k+kc}{True}\PY{p}{,}
        \PY{n}{ext}\PY{o}{=}\PY{l+s+s1}{\PYZsq{}}\PY{l+s+s1}{raise}\PY{l+s+s1}{\PYZsq{}}
    \PY{p}{)}
    \PY{k}{for} \PY{n}{s} \PY{o+ow}{in} \PY{n}{assumed\PYZus{}stdev}
\PY{p}{]}

\PY{n}{fig}\PY{p}{,} \PY{n}{ax} \PY{o}{=} \PY{n}{plt}\PY{o}{.}\PY{n}{subplots}\PY{p}{(}
    \PY{l+m+mi}{1}\PY{p}{,} \PY{n+nb}{len}\PY{p}{(}\PY{n}{assumed\PYZus{}stdev}\PY{p}{)}\PY{p}{,} \PY{n}{figsize}\PY{o}{=}\PY{p}{(}\PY{l+m+mi}{7}\PY{p}{,} \PY{l+m+mf}{6.4}\PY{p}{)}\PY{p}{,}
    \PY{n}{sharey}\PY{o}{=}\PY{k+kc}{True}
\PY{p}{)}

\PY{k}{for} \PY{n}{i} \PY{o+ow}{in} \PY{n+nb}{range}\PY{p}{(}\PY{n+nb}{len}\PY{p}{(}\PY{n}{assumed\PYZus{}stdev}\PY{p}{)}\PY{p}{)}\PY{p}{:}
    \PY{n}{plt}\PY{o}{.}\PY{n}{sca}\PY{p}{(}\PY{n}{ax}\PY{p}{[}\PY{n}{i}\PY{p}{]}\PY{p}{)}
    \PY{n}{plt}\PY{o}{.}\PY{n}{plot}\PY{p}{(}
        \PY{n}{t}\PY{p}{,} \PY{n}{a}\PY{p}{,}
        \PY{l+s+s2}{\PYZdq{}}\PY{l+s+s2}{.k}\PY{l+s+s2}{\PYZdq{}}\PY{p}{,}
        \PY{n}{label}\PY{o}{=}\PY{l+s+s2}{\PYZdq{}}\PY{l+s+s2}{Raw Data}\PY{l+s+s2}{\PYZdq{}}\PY{p}{,}
        \PY{n}{alpha}\PY{o}{=}\PY{l+m+mf}{0.5}\PY{p}{,}
        \PY{n}{markersize}\PY{o}{=}\PY{l+m+mi}{6}\PY{p}{,}
        \PY{n}{markeredgewidth}\PY{o}{=}\PY{l+m+mi}{0}
    \PY{p}{)}

    \PY{n}{n} \PY{o}{=} \PY{n}{names}\PY{p}{[}\PY{n}{i}\PY{p}{]}
    \PY{n}{t\PYZus{}plot} \PY{o}{=} \PY{n}{np}\PY{o}{.}\PY{n}{linspace}\PY{p}{(}\PY{n}{t}\PY{p}{[}\PY{l+m+mi}{0}\PY{p}{]}\PY{p}{,} \PY{n}{t}\PY{p}{[}\PY{o}{\PYZhy{}}\PY{l+m+mi}{1}\PY{p}{]}\PY{p}{,} \PY{l+m+mi}{5000}\PY{p}{)}
    \PY{n}{plt}\PY{o}{.}\PY{n}{plot}\PY{p}{(}
        \PY{n}{t\PYZus{}plot}\PY{p}{,}
        \PY{n}{estimators}\PY{p}{[}\PY{n}{i}\PY{p}{]}\PY{p}{(}\PY{n}{t\PYZus{}plot}\PY{p}{)}\PY{p}{,}
        \PY{n}{label}\PY{o}{=}\PY{l+s+s2}{\PYZdq{}}\PY{l+s+s2}{Estimated Truth}\PY{l+s+s2}{\PYZdq{}}\PY{p}{,}
        \PY{n}{alpha}\PY{o}{=}\PY{l+m+mf}{0.9}\PY{p}{,}
        \PY{n}{color}\PY{o}{=}\PY{n}{colors}\PY{p}{[}\PY{n}{i}\PY{p}{]}
    \PY{p}{)}

    \PY{n}{xlim} \PY{o}{=} \PY{p}{(}\PY{l+m+mi}{60}\PY{p}{,} \PY{l+m+mi}{90}\PY{p}{)}
    \PY{n}{plt}\PY{o}{.}\PY{n}{xlim}\PY{p}{(}\PY{o}{*}\PY{n}{xlim}\PY{p}{)}
    \PY{n}{plt}\PY{o}{.}\PY{n}{ylim}\PY{p}{(}\PY{l+m+mi}{6}\PY{p}{,} \PY{l+m+mi}{13}\PY{p}{)}
    \PY{n}{plt}\PY{o}{.}\PY{n}{title}\PY{p}{(}\PY{n}{n}\PY{p}{)}
    \PY{n}{plt}\PY{o}{.}\PY{n}{xlabel}\PY{p}{(}\PY{l+s+s2}{\PYZdq{}}\PY{l+s+s2}{Time after Takeoff [sec]}\PY{l+s+s2}{\PYZdq{}}\PY{p}{)}
    \PY{k}{if} \PY{n}{i} \PY{o}{==} \PY{l+m+mi}{0}\PY{p}{:}
        \PY{n}{plt}\PY{o}{.}\PY{n}{ylabel}\PY{p}{(}\PY{l+s+s2}{\PYZdq{}}\PY{l+s+s2}{Airspeed [m/s]}\PY{l+s+s2}{\PYZdq{}}\PY{p}{)}
    \PY{n}{p}\PY{o}{.}\PY{n}{set\PYZus{}ticks}\PY{p}{(}\PY{l+m+mi}{10}\PY{p}{,} \PY{l+m+mi}{2}\PY{p}{,} \PY{l+m+mi}{1}\PY{p}{,} \PY{l+m+mf}{0.2}\PY{p}{)}
    \PY{k}{if} \PY{n}{i} \PY{o}{==} \PY{l+m+mi}{1}\PY{p}{:}
        \PY{n}{plt}\PY{o}{.}\PY{n}{legend}\PY{p}{(}\PY{p}{)}

\PY{n}{plt}\PY{o}{.}\PY{n}{suptitle}\PY{p}{(}\PY{l+s+sa}{f}\PY{l+s+s2}{\PYZdq{}}\PY{l+s+s2}{One Dataset, Many Possible Interpretations:}\PY{l+s+se}{\PYZbs{}n}\PY{l+s+s2}{Airspeed data, zoomed to \PYZdl{}t }\PY{l+s+s2}{\PYZbs{}}\PY{l+s+s2}{in }\PY{l+s+si}{\PYZob{}}\PY{n}{xlim}\PY{l+s+si}{\PYZcb{}}\PY{l+s+s2}{\PYZdl{}}\PY{l+s+s2}{\PYZdq{}}\PY{p}{)}

\PY{n}{p}\PY{o}{.}\PY{n}{show\PYZus{}plot}\PY{p}{(}
    \PY{n}{legend}\PY{o}{=}\PY{k+kc}{False}\PY{p}{,}
    \PY{n}{dpi}\PY{o}{=}\PY{l+m+mi}{300}\PY{p}{,}
\PY{p}{)}
\end{Verbatim}
\end{tcolorbox}

    \begin{center}
    \adjustimage{max size={0.9\linewidth}{0.9\paperheight}}{output_27_0.png}
    \end{center}
    { \hspace*{\fill} \\}
    
    Ultimately, all three charts represent different \emph{interpretations
of the ground truth based on the data}. If we didn't have any more
information about the problem, all three could be equally justifiable
given the right contexts - but, in fact, we do have more information,
since we know this data originates from a physical system.

The chart on the left is clearly overfitted (i.e., tracks the data too
closely) based on engineering intuition. Essentially, we are assuming
that each sample is the true data, and that each sample has no noise;
hence, the ``truth'' curve exhibits extremely high variance (strong
wiggles). There's no \emph{mathematical} reason why this interpretation
can't be correct (it's theoretically possible that the sensor is giving
perfect information), but of course it is \emph{physically} implausible
for our system. If this interpretation were true, it would imply that
the vehicle is achieving velocity-vector-aligned-accelerations on the
order of:

    \begin{tcolorbox}[breakable, size=fbox, boxrule=1pt, pad at break*=1mm,colback=cellbackground, colframe=cellborder]
\prompt{In}{incolor}{7}{\boxspacing}
\begin{Verbatim}[commandchars=\\\{\}]
\PY{n+nb}{print}\PY{p}{(}\PY{l+s+s2}{\PYZdq{}}\PY{l+s+s2}{Peak instantaneous x\PYZhy{}acceleration:}\PY{l+s+s2}{\PYZdq{}}\PY{p}{,}
      \PY{l+s+sa}{f}\PY{l+s+s2}{\PYZdq{}}\PY{l+s+si}{\PYZob{}}\PY{n}{np}\PY{o}{.}\PY{n}{abs}\PY{p}{(}\PY{n}{estimators}\PY{p}{[}\PY{l+m+mi}{0}\PY{p}{]}\PY{o}{.}\PY{n}{derivative}\PY{p}{(}\PY{p}{)}\PY{p}{(}\PY{n}{t}\PY{p}{)}\PY{p}{)}\PY{o}{.}\PY{n}{max}\PY{p}{(}\PY{p}{)}\PY{+w}{ }\PY{o}{/}\PY{+w}{ }\PY{l+m+mf}{9.81}\PY{l+s+si}{:}\PY{l+s+s2}{.2f}\PY{l+s+si}{\PYZcb{}}\PY{l+s+s2}{ G}\PY{l+s+s2}{\PYZdq{}}\PY{p}{)}
\end{Verbatim}
\end{tcolorbox}

    \begin{Verbatim}[commandchars=\\\{\}]
Peak instantaneous x-acceleration: 2.42 G
    \end{Verbatim}

    which is obviously not physically possible.

    The chart on the right is clearly underfitted (i.e., tracks the data too
slowly), as there are large regions where the estimator has a consistent
bias (e.g., consistently too high or low) with respect to the underlying
data. It might seem like there are no contexts where such an
interpretation would be reasonable, but that is not necessarily the
case. For example, if the sensor noise at each sample was not
independent, but rather correlated with noise from the previous samples,
we might not be able to rule out this interpretation of reality.

    Of course, a quick glance at the three charts above would tell us ``the
middle one looks about right'' - but why? What is it about the middle
chart that makes it look more reasonable than the other two?

    \hypertarget{optimal-sensor-data-reconstruction-using-data-driven-noise-estimates}{%
\subsubsection{Optimal Sensor Data Reconstruction using Data-Driven
Noise
Estimates}\label{optimal-sensor-data-reconstruction-using-data-driven-noise-estimates}}

The key difference is that the middle chart embeds certain intuitive
assumptions that we ``just know'' about our data into its
reconstruction. Specifically, the middle chart is an optimal
reconstruction of the data assuming that: - The noise in each sample is
independent (i.e., uncorrelated with previous samples) and
normally-distributed - The noise is unbiased (i.e., there are no
systematic errors in the data, only random ones) - The noise is
homoscedastic (i.e., the standard deviation of the noise is constant
across the entire dataset) - The sample rate is significantly higher
than the underlying dynamics of the system that we're aiming to recover
(i.e., the system is ``slow'' relative to the sample rate).

Under these assumptions above, the probabilistic properties of the noise
reduce to a single parameter: the variance (that is, the square of the
standard deviation, \(\sigma^2\)) of the noise. Thus, by constructing an
estimator for this statistic, we can recover the probability
distribution of the noise and hence an optimal estimator of the data.

    \hypertarget{motivation-for-finding-the-variance-of-the-noise}{%
\paragraph{Motivation for Finding the Variance of the
Noise}\label{motivation-for-finding-the-variance-of-the-noise}}

Because we're just one parameter away from having a probabalistic model
of the sensor noise, we're strongly motivated to estimate this noise
variance. Once we estimate the variance, we can begin to make optimal
choices about the bias-variance tradeoff \emph{with a rigorous
definition of what optimal means} here.

There is a second factor that motivates us to estimate the variance of
the noise. It turns out that computing an optimal smoothing spline for a
time-series dataset is a well-studied problem, but only if we know the
variance of the noise. So, if we can obtain an estimate of the variance
of the noise, we've done most of the work required to compute an optimal
data reconstructor.

In an ideal world, we would estimate this variance of the noise by
taking a large number of samples while the vehicle is at some fixed,
steady, known condition. For example, to estimate the airspeed data
noise, we might place the aircraft in a wind tunnel at some constant
speed. If we ran that experiment for an extended duration, we would know
that the true underlying data was constant, and hence any observed
variance in the samples would be due to the variance of the sensor
noise.

If we want to reconstruct data from unsteady measurement, however, we
need to find a way to estimate the variance of the noise using the
unsteady data itself, which is much more tricky.

    \hypertarget{initial-approach-and-a-first-order-data-based-noise-estimator}{%
\paragraph{Initial Approach and a First-Order Data-Based Noise
Estimator}\label{initial-approach-and-a-first-order-data-based-noise-estimator}}

One way to do this is by assuming that the data is ``slow'' relative to
the sample rate. If this is true, then subsequent samples will
effectively have the same underlying truth value, but with noise drawn
independently from the same underlying distribution.

Quantitatively, if our true data is \(x(t)\), and the noise of the
sample \(n(t)\) is drawn from \(\mathcal{N}(0, \sigma^2)\), then our
observed sensor data \(s(t)\) will be:

\[s(t_1) = x(t_1) + n(t_1)\]

\[s(t_2) = x(t_2) + n(t_2)\]

If \(t_1 \approx t_2\) (as would be the case if we're looking at
subsequent samples), then \(x(t_1) \approx x(t_2)\), and hence:

\[s(t_2) - s(t_1) \approx n(t_2) - n(t_1)\]

    In other words, the difference of two subsequent observed samples will
be approximately equal to the difference of two independent draws from
the same noise distribution. This means that we can estimate the
variance of the noise by looking at the variance of the differences
between subsequent samples.

From the properties of the difference of two independent random normal
variables, we can then say:

\[n(t_2) - n(t_1) \sim \mathcal{N}(0, 2\sigma^2)\]

where \(\sigma^2\) is the variance of the sensor noise. So, the
following is then approximately true:

\[s(t_2) - s(t_1) \sim \mathcal{N}(0, 2\sigma^2)\]

    Therefore, by taking the mean difference across all of the adjacent
pairs of sample data, we can reconstruct the variance of the noise as:

\[\sigma^2 = \frac{1}{2 \cdot (N-1)} \sum_{i=1}^{N-1} \Big( s(t_{i+1}) - s(t_i) \Big)^2\]

where \(N\) is the number of samples in the dataset. Note that this
calculation essentially uses a population standard deviation of the
differences of subsequent sensor samples, then corrects it by a
denominator to get the variance of the noise.

    \hypertarget{numerical-demonstration-of-the-first-order-noise-estimator}{%
\paragraph{Numerical Demonstration of the First-Order Noise
Estimator}\label{numerical-demonstration-of-the-first-order-noise-estimator}}

We can demonstrate that this works in practice and is convergent to the
correct answer by constructing a synthetic dataset with known noise
properties, and then reconstructing the noise variance using the method
above. The dataset we use is a simple sinusoid with added noise. We then
attempt to recover the noise variance using the method above, and
compare it to the true noise variance.

    In other words, our process looks like:

    \begin{tcolorbox}[breakable, size=fbox, boxrule=1pt, pad at break*=1mm,colback=cellbackground, colframe=cellborder]
\prompt{In}{incolor}{8}{\boxspacing}
\begin{Verbatim}[commandchars=\\\{\}]
\PY{n}{g} \PY{o}{=} \PY{n}{graphviz}\PY{o}{.}\PY{n}{Digraph}\PY{p}{(}\PY{o}{*}\PY{o}{*}\PY{n}{gv\PYZus{}settings}\PY{p}{)}
\PY{n}{var\PYZus{}style} \PY{o}{=} \PY{n+nb}{dict}\PY{p}{(}\PY{n}{fillcolor}\PY{o}{=}\PY{l+s+s1}{\PYZsq{}}\PY{l+s+s1}{\PYZsh{}F1F7E9}\PY{l+s+s1}{\PYZsq{}}\PY{p}{)}

\PY{n}{g}\PY{o}{.}\PY{n}{attr}\PY{p}{(}
    \PY{n}{label}\PY{o}{=}\PY{l+s+s2}{\PYZdq{}}\PY{l+s+s2}{Procedure for Numerical Demonstration}\PY{l+s+se}{\PYZbs{}n}\PY{l+s+s2}{of Estimator Performance}\PY{l+s+s2}{\PYZdq{}}\PY{p}{,}
    \PY{n}{labelloc}\PY{o}{=}\PY{l+s+s2}{\PYZdq{}}\PY{l+s+s2}{t}\PY{l+s+s2}{\PYZdq{}}\PY{p}{,}
    \PY{n}{fontname}\PY{o}{=}\PY{l+s+s1}{\PYZsq{}}\PY{l+s+s1}{Helvetica\PYZhy{}Bold}\PY{l+s+s1}{\PYZsq{}}\PY{p}{,} \PY{n}{fontsize}\PY{o}{=}\PY{l+s+s1}{\PYZsq{}}\PY{l+s+s1}{18}\PY{l+s+s1}{\PYZsq{}}\PY{p}{,} \PY{n}{shape}\PY{o}{=}\PY{l+s+s1}{\PYZsq{}}\PY{l+s+s1}{plaintext}\PY{l+s+s1}{\PYZsq{}}
\PY{p}{)}

\PY{n}{g}\PY{o}{.}\PY{n}{node}\PY{p}{(}\PY{l+s+s2}{\PYZdq{}}\PY{l+s+s2}{func}\PY{l+s+s2}{\PYZdq{}}\PY{p}{,} \PY{n}{label}\PY{o}{=}\PY{l+s+s2}{\PYZdq{}}\PY{l+s+s2}{Underlying }\PY{l+s+se}{\PYZbs{}\PYZdq{}}\PY{l+s+s2}{truth}\PY{l+s+se}{\PYZbs{}\PYZdq{}}\PY{l+s+s2}{ function}\PY{l+s+se}{\PYZbs{}n}\PY{l+s+s2}{(sinusoid with a known signal frequency)}\PY{l+s+s2}{\PYZdq{}}\PY{p}{)}
\PY{n}{g}\PY{o}{.}\PY{n}{node}\PY{p}{(}\PY{l+s+s2}{\PYZdq{}}\PY{l+s+s2}{draw}\PY{l+s+s2}{\PYZdq{}}\PY{p}{,} \PY{n}{label}\PY{o}{=}\PY{l+s+s2}{\PYZdq{}}\PY{l+s+s2}{Take discrete samples}\PY{l+s+se}{\PYZbs{}n}\PY{l+s+s2}{at sample frequency}\PY{l+s+s2}{\PYZdq{}}\PY{p}{)}
\PY{n}{g}\PY{o}{.}\PY{n}{edge}\PY{p}{(}\PY{l+s+s2}{\PYZdq{}}\PY{l+s+s2}{func}\PY{l+s+s2}{\PYZdq{}}\PY{p}{,} \PY{l+s+s2}{\PYZdq{}}\PY{l+s+s2}{draw}\PY{l+s+s2}{\PYZdq{}}\PY{p}{)}

\PY{n}{g}\PY{o}{.}\PY{n}{node}\PY{p}{(}\PY{l+s+s2}{\PYZdq{}}\PY{l+s+s2}{tvar}\PY{l+s+s2}{\PYZdq{}}\PY{p}{,} \PY{n}{label}\PY{o}{=}\PY{l+s+s2}{\PYZdq{}}\PY{l+s+s2}{True variance}\PY{l+s+s2}{\PYZdq{}}\PY{p}{,} \PY{o}{*}\PY{o}{*}\PY{n}{var\PYZus{}style}\PY{p}{)}
\PY{n}{g}\PY{o}{.}\PY{n}{node}\PY{p}{(}\PY{l+s+s2}{\PYZdq{}}\PY{l+s+s2}{noise}\PY{l+s+s2}{\PYZdq{}}\PY{p}{,} \PY{n}{label}\PY{o}{=}\PY{l+s+s2}{\PYZdq{}}\PY{l+s+s2}{Add noise}\PY{l+s+se}{\PYZbs{}n}\PY{l+s+s2}{(normal dist., independent)}\PY{l+s+s2}{\PYZdq{}}\PY{p}{)}
\PY{n}{g}\PY{o}{.}\PY{n}{edge}\PY{p}{(}\PY{l+s+s2}{\PYZdq{}}\PY{l+s+s2}{tvar}\PY{l+s+s2}{\PYZdq{}}\PY{p}{,} \PY{l+s+s2}{\PYZdq{}}\PY{l+s+s2}{noise}\PY{l+s+s2}{\PYZdq{}}\PY{p}{)}

\PY{n}{g}\PY{o}{.}\PY{n}{edge}\PY{p}{(}\PY{l+s+s2}{\PYZdq{}}\PY{l+s+s2}{draw}\PY{l+s+s2}{\PYZdq{}}\PY{p}{,} \PY{l+s+s2}{\PYZdq{}}\PY{l+s+s2}{noise}\PY{l+s+s2}{\PYZdq{}}\PY{p}{)}

\PY{n}{g}\PY{o}{.}\PY{n}{node}\PY{p}{(}\PY{l+s+s2}{\PYZdq{}}\PY{l+s+s2}{syn}\PY{l+s+s2}{\PYZdq{}}\PY{p}{,} \PY{n}{label}\PY{o}{=}\PY{l+s+s2}{\PYZdq{}}\PY{l+s+s2}{Synthetic Dataset}\PY{l+s+s2}{\PYZdq{}}\PY{p}{,} \PY{n}{fillcolor}\PY{o}{=}\PY{l+s+s1}{\PYZsq{}}\PY{l+s+s1}{\PYZsh{}F7E9F1}\PY{l+s+s1}{\PYZsq{}}\PY{p}{)}
\PY{n}{g}\PY{o}{.}\PY{n}{edge}\PY{p}{(}\PY{l+s+s2}{\PYZdq{}}\PY{l+s+s2}{noise}\PY{l+s+s2}{\PYZdq{}}\PY{p}{,} \PY{l+s+s2}{\PYZdq{}}\PY{l+s+s2}{syn}\PY{l+s+s2}{\PYZdq{}}\PY{p}{)}

\PY{n}{g}\PY{o}{.}\PY{n}{node}\PY{p}{(}\PY{l+s+s2}{\PYZdq{}}\PY{l+s+s2}{est}\PY{l+s+s2}{\PYZdq{}}\PY{p}{,} \PY{n}{label}\PY{o}{=}\PY{l+s+s2}{\PYZdq{}}\PY{l+s+s2}{Apply the first\PYZhy{}order estimator}\PY{l+s+s2}{\PYZdq{}}\PY{p}{)}
\PY{n}{g}\PY{o}{.}\PY{n}{node}\PY{p}{(}\PY{l+s+s2}{\PYZdq{}}\PY{l+s+s2}{evar}\PY{l+s+s2}{\PYZdq{}}\PY{p}{,} \PY{n}{label}\PY{o}{=}\PY{l+s+s2}{\PYZdq{}}\PY{l+s+s2}{Estimated variance}\PY{l+s+s2}{\PYZdq{}}\PY{p}{,} \PY{o}{*}\PY{o}{*}\PY{n}{var\PYZus{}style}\PY{p}{)}

\PY{n}{g}\PY{o}{.}\PY{n}{edge}\PY{p}{(}\PY{l+s+s2}{\PYZdq{}}\PY{l+s+s2}{syn}\PY{l+s+s2}{\PYZdq{}}\PY{p}{,} \PY{l+s+s2}{\PYZdq{}}\PY{l+s+s2}{est}\PY{l+s+s2}{\PYZdq{}}\PY{p}{)}
\PY{n}{g}\PY{o}{.}\PY{n}{edge}\PY{p}{(}\PY{l+s+s2}{\PYZdq{}}\PY{l+s+s2}{est}\PY{l+s+s2}{\PYZdq{}}\PY{p}{,} \PY{l+s+s2}{\PYZdq{}}\PY{l+s+s2}{evar}\PY{l+s+s2}{\PYZdq{}}\PY{p}{)}

\PY{n}{g}\PY{o}{.}\PY{n}{node}\PY{p}{(}\PY{l+s+s2}{\PYZdq{}}\PY{l+s+s2}{c}\PY{l+s+s2}{\PYZdq{}}\PY{p}{,} \PY{n}{label}\PY{o}{=}\PY{l+s+s2}{\PYZdq{}}\PY{l+s+s2}{Compare to}\PY{l+s+se}{\PYZbs{}n}\PY{l+s+s2}{assess performance}\PY{l+s+s2}{\PYZdq{}}\PY{p}{,} \PY{n}{shape}\PY{o}{=}\PY{l+s+s1}{\PYZsq{}}\PY{l+s+s1}{plaintext}\PY{l+s+s1}{\PYZsq{}}\PY{p}{,} \PY{n}{fillcolor}\PY{o}{=}\PY{l+s+s2}{\PYZdq{}}\PY{l+s+s2}{white}\PY{l+s+s2}{\PYZdq{}}\PY{p}{)}

\PY{n}{dashed\PYZus{}line} \PY{o}{=} \PY{n+nb}{dict}\PY{p}{(}
    \PY{n}{style}\PY{o}{=}\PY{l+s+s1}{\PYZsq{}}\PY{l+s+s1}{dashed}\PY{l+s+s1}{\PYZsq{}}\PY{p}{,}
    \PY{n}{color}\PY{o}{=}\PY{l+s+s1}{\PYZsq{}}\PY{l+s+s1}{gray}\PY{l+s+s1}{\PYZsq{}}\PY{p}{,}
    \PY{n}{arrowhead}\PY{o}{=}\PY{l+s+s1}{\PYZsq{}}\PY{l+s+s1}{none}\PY{l+s+s1}{\PYZsq{}}\PY{p}{,}
    \PY{n}{arrowtail}\PY{o}{=}\PY{l+s+s1}{\PYZsq{}}\PY{l+s+s1}{none}\PY{l+s+s1}{\PYZsq{}}\PY{p}{,}
    \PY{n+nb}{dir}\PY{o}{=}\PY{l+s+s1}{\PYZsq{}}\PY{l+s+s1}{both}\PY{l+s+s1}{\PYZsq{}}\PY{p}{,}
    \PY{n}{penwidth}\PY{o}{=}\PY{l+s+s1}{\PYZsq{}}\PY{l+s+s1}{1}\PY{l+s+s1}{\PYZsq{}}\PY{p}{,}
\PY{p}{)}
\PY{n}{g}\PY{o}{.}\PY{n}{edge}\PY{p}{(}\PY{l+s+s2}{\PYZdq{}}\PY{l+s+s2}{tvar}\PY{l+s+s2}{\PYZdq{}}\PY{p}{,} \PY{l+s+s2}{\PYZdq{}}\PY{l+s+s2}{c}\PY{l+s+s2}{\PYZdq{}}\PY{p}{,} \PY{o}{*}\PY{o}{*}\PY{n}{dashed\PYZus{}line}\PY{p}{)}
\PY{n}{g}\PY{o}{.}\PY{n}{edge}\PY{p}{(}\PY{l+s+s2}{\PYZdq{}}\PY{l+s+s2}{evar}\PY{l+s+s2}{\PYZdq{}}\PY{p}{,} \PY{l+s+s2}{\PYZdq{}}\PY{l+s+s2}{c}\PY{l+s+s2}{\PYZdq{}}\PY{p}{,} \PY{o}{*}\PY{o}{*}\PY{n}{dashed\PYZus{}line}\PY{p}{)}

\PY{n}{g}
\end{Verbatim}
\end{tcolorbox}
 
            
\prompt{Out}{outcolor}{8}{}
    
    \begin{center}
    \adjustimage{max size={0.9\linewidth}{0.9\paperheight}}{output_40_0.pdf}
    \end{center}
    { \hspace*{\fill} \\}
    

    In code:

    \begin{tcolorbox}[breakable, size=fbox, boxrule=1pt, pad at break*=1mm,colback=cellbackground, colframe=cellborder]
\prompt{In}{incolor}{9}{\boxspacing}
\begin{Verbatim}[commandchars=\\\{\}]
\PY{n}{np}\PY{o}{.}\PY{n}{random}\PY{o}{.}\PY{n}{seed}\PY{p}{(}\PY{l+m+mi}{0}\PY{p}{)}  \PY{c+c1}{\PYZsh{} Sets a pseudo\PYZhy{}random seed for reproducibility}


\PY{k}{def} \PY{n+nf}{synthesize\PYZus{}data\PYZus{}and\PYZus{}reconstruct\PYZus{}noise\PYZus{}variance\PYZus{}1}\PY{p}{(}\PY{n}{freq\PYZus{}signal}\PY{p}{,} \PY{n}{freq\PYZus{}sample}\PY{p}{)}\PY{p}{:}
\PY{+w}{    }\PY{l+s+sd}{\PYZdq{}\PYZdq{}\PYZdq{}}
\PY{l+s+sd}{    Generates a synthetic dataset with known noise properties, and then reconstructs the noise variance using the}
\PY{l+s+sd}{    first\PYZhy{}order method described above.}

\PY{l+s+sd}{    The signal is assumed to be a sinusoid with some frequency, and the noise is assumed to be independent,}
\PY{l+s+sd}{    homoscedastic, and normally\PYZhy{}distributed.}

\PY{l+s+sd}{    Args:}
\PY{l+s+sd}{        freq\PYZus{}signal: Frequency of the underlying true signal [Hz]}
\PY{l+s+sd}{        freq\PYZus{}sample: Data sample rate from the sensor [Hz]}

\PY{l+s+sd}{    Returns: None. Prints the true noise standard deviation and the reconstructed (estimated) noise standard}
\PY{l+s+sd}{    deviation, as well as various input parameters.}
\PY{l+s+sd}{    \PYZdq{}\PYZdq{}\PYZdq{}}
    \PY{c+c1}{\PYZsh{}\PYZsh{}\PYZsh{}\PYZsh{}\PYZsh{} Synthetic Data Generation \PYZsh{}\PYZsh{}\PYZsh{}\PYZsh{}\PYZsh{}}
    \PY{n}{t} \PY{o}{=} \PY{n}{np}\PY{o}{.}\PY{n}{arange}\PY{p}{(}\PY{l+m+mi}{0}\PY{p}{,} \PY{l+m+mi}{5}\PY{p}{,} \PY{l+m+mi}{1} \PY{o}{/} \PY{n}{freq\PYZus{}sample}\PY{p}{)}

    \PY{n}{true\PYZus{}data} \PY{o}{=} \PY{n}{np}\PY{o}{.}\PY{n}{sin}\PY{p}{(}\PY{l+m+mi}{2} \PY{o}{*} \PY{n}{np}\PY{o}{.}\PY{n}{pi} \PY{o}{*} \PY{n}{freq\PYZus{}signal} \PY{o}{*} \PY{n}{t}\PY{p}{)}

    \PY{n}{true\PYZus{}noise\PYZus{}standard\PYZus{}deviation} \PY{o}{=} \PY{l+m+mf}{0.1}
    \PY{n}{noise} \PY{o}{=} \PY{n}{np}\PY{o}{.}\PY{n}{random}\PY{o}{.}\PY{n}{normal}\PY{p}{(}\PY{n}{size}\PY{o}{=}\PY{n+nb}{len}\PY{p}{(}\PY{n}{true\PYZus{}data}\PY{p}{)}\PY{p}{)} \PY{o}{*} \PY{n}{true\PYZus{}noise\PYZus{}standard\PYZus{}deviation}

    \PY{n}{sensor\PYZus{}data} \PY{o}{=} \PY{n}{true\PYZus{}data} \PY{o}{+} \PY{n}{noise}

    \PY{c+c1}{\PYZsh{}\PYZsh{}\PYZsh{}\PYZsh{}\PYZsh{} Noise Variance Reconstruction \PYZsh{}\PYZsh{}\PYZsh{}\PYZsh{}\PYZsh{}}
    \PY{n}{estimated\PYZus{}noise\PYZus{}standard\PYZus{}deviation} \PY{o}{=} \PY{p}{(}
            \PY{n}{np}\PY{o}{.}\PY{n}{std}\PY{p}{(}  \PY{c+c1}{\PYZsh{} Standard deviation}
                \PY{n}{np}\PY{o}{.}\PY{n}{diff}\PY{p}{(}\PY{n}{sensor\PYZus{}data}\PY{p}{)}  \PY{c+c1}{\PYZsh{} of the differences between subsequent samples}
            \PY{p}{)} \PY{o}{/} \PY{n}{np}\PY{o}{.}\PY{n}{sqrt}\PY{p}{(}\PY{l+m+mi}{2}\PY{p}{)}
    \PY{p}{)}

    \PY{n+nb}{print}\PY{p}{(}\PY{l+s+sa}{f}\PY{l+s+s2}{\PYZdq{}}\PY{l+s+s2}{First\PYZhy{}Order Data\PYZhy{}Driven Noise Estimator}\PY{l+s+s2}{\PYZdq{}}\PY{p}{)}
    \PY{n+nb}{print}\PY{p}{(}\PY{l+s+sa}{f}\PY{l+s+s2}{\PYZdq{}}\PY{l+s+s2}{Sample frequency / Underlying Signal Frequency = }\PY{l+s+si}{\PYZob{}}\PY{l+m+mi}{1}\PY{+w}{ }\PY{o}{/}\PY{+w}{ }\PY{n}{np}\PY{o}{.}\PY{n}{diff}\PY{p}{(}\PY{n}{t}\PY{p}{)}\PY{p}{[}\PY{l+m+mi}{0}\PY{p}{]}\PY{l+s+si}{\PYZcb{}}\PY{l+s+s2}{\PYZdq{}}\PY{p}{)}
    \PY{n+nb}{print}\PY{p}{(}\PY{l+s+s2}{\PYZdq{}}\PY{l+s+s2}{\PYZhy{}}\PY{l+s+s2}{\PYZdq{}} \PY{o}{*} \PY{l+m+mi}{50}\PY{p}{)}
    \PY{n+nb}{print}\PY{p}{(}\PY{l+s+sa}{f}\PY{l+s+s2}{\PYZdq{}}\PY{l+s+s2}{True noise standard deviation: }\PY{l+s+si}{\PYZob{}}\PY{n}{true\PYZus{}noise\PYZus{}standard\PYZus{}deviation}\PY{l+s+si}{:}\PY{l+s+s2}{.3f}\PY{l+s+si}{\PYZcb{}}\PY{l+s+s2}{\PYZdq{}}\PY{p}{)}
    \PY{n+nb}{print}\PY{p}{(}\PY{l+s+sa}{f}\PY{l+s+s2}{\PYZdq{}}\PY{l+s+s2}{Estimated noise standard deviation: }\PY{l+s+si}{\PYZob{}}\PY{n}{estimated\PYZus{}noise\PYZus{}standard\PYZus{}deviation}\PY{l+s+si}{:}\PY{l+s+s2}{.3f}\PY{l+s+si}{\PYZcb{}}\PY{l+s+s2}{\PYZdq{}}\PY{p}{)}
\end{Verbatim}
\end{tcolorbox}

    \begin{tcolorbox}[breakable, size=fbox, boxrule=1pt, pad at break*=1mm,colback=cellbackground, colframe=cellborder]
\prompt{In}{incolor}{10}{\boxspacing}
\begin{Verbatim}[commandchars=\\\{\}]
\PY{n}{synthesize\PYZus{}data\PYZus{}and\PYZus{}reconstruct\PYZus{}noise\PYZus{}variance\PYZus{}1}\PY{p}{(}
    \PY{n}{freq\PYZus{}signal}\PY{o}{=}\PY{l+m+mi}{1}\PY{p}{,}
    \PY{n}{freq\PYZus{}sample}\PY{o}{=}\PY{l+m+mi}{1000}
\PY{p}{)}
\end{Verbatim}
\end{tcolorbox}

    \begin{Verbatim}[commandchars=\\\{\}]
First-Order Data-Driven Noise Estimator
Sample frequency / Underlying Signal Frequency = 1000.0
--------------------------------------------------
True noise standard deviation: 0.100
Estimated noise standard deviation: 0.098
    \end{Verbatim}

    As we can see, noise standard deviation that we estimate using this
method is very close to the true value. We will also show later that
this estimator is convergent to the true value as the number of samples
increases.

However, this estimator is only first-order convergent with respect to
the ratio between the sampling frequency (\texttt{freq\_sample}) and the
underlying dynamics of the system (\texttt{freq\_signal}). In other
words, if the system is ``fast'' relative to the sample rate, then the
estimator will be inaccurate. We can demonstrate this by repeating the
above experiment, but with a higher-frequency true signal:

    \begin{tcolorbox}[breakable, size=fbox, boxrule=1pt, pad at break*=1mm,colback=cellbackground, colframe=cellborder]
\prompt{In}{incolor}{11}{\boxspacing}
\begin{Verbatim}[commandchars=\\\{\}]
\PY{n}{synthesize\PYZus{}data\PYZus{}and\PYZus{}reconstruct\PYZus{}noise\PYZus{}variance\PYZus{}1}\PY{p}{(}
    \PY{n}{freq\PYZus{}signal}\PY{o}{=}\PY{l+m+mi}{100}\PY{p}{,}
    \PY{n}{freq\PYZus{}sample}\PY{o}{=}\PY{l+m+mi}{1000}
\PY{p}{)}
\end{Verbatim}
\end{tcolorbox}

    \begin{Verbatim}[commandchars=\\\{\}]
First-Order Data-Driven Noise Estimator
Sample frequency / Underlying Signal Frequency = 1000.0
--------------------------------------------------
True noise standard deviation: 0.100
Estimated noise standard deviation: 0.324
    \end{Verbatim}

    Notably, the discrepancy between the true and estimated noise standard
deviations are much larger, so our first-order estimator is not working
as well.

    \hypertarget{a-second-order-estimator}{%
\paragraph{A Second-Order Estimator}\label{a-second-order-estimator}}

To do a better job of estimating the noise variance even in cases where
\(f_{\rm sample} \gg f_{\rm signal}\) doesn't strictly hold, we can use
a second-order estimator. With this extension, we use a three-point
numerical stencil.

Derivation of this estimator follows similar principles to the
first-order estimator above, with the key result as follows:

\[\sigma^2 = \frac{1}{6 \cdot (N-2)} \sum_{i=1}^{N-2} \Big( s(t_{i+2}) - 2 \cdot s(t_{i+1}) + s(t_i) \Big)^2\]

where \(N\) is the number of samples in the dataset.

Notably, this effectively implements a first-order, second-degree
finite-difference of the underlying data - a discrete derivative. It's
instructive to consider the theoretical underpinning of using a discrete
derivative operator here. If we assume that the underlying true signal
is relatively smooth, then as we take more discrete derivatives of the
observed data, the signal component will asymptote to zero (assuming
\(f_{\rm sample} \geq f_{\rm signal}\)). Stated equivalently, the
frequency spectrum of the true signal is assumed to have some cutoff
frequency (analogous to \(f_{\rm signal}\)), above which the power
spectral density gradually goes to zero. We don't need to specify this
cutoff frequency or know it precisely \emph{a priori}; the remarks
presented here merely require that one exists and is somewhere below
\(f_{\rm sample}\).

In contrast, the noise is assumed to be independent, which means that
the noise component of the observed data will \emph{not} go to zero as
we take successive derivatives. Stated equivalently, the frequency
spectrum of the noise is \emph{white}, with uniform power across all
frequencies. Therefore, \textbf{repeated application of the discrete
derivative operator acts as a way to spectrally-separate the noise from
the signal}. This is a key insight that we will use later in the paper.

In code, this second-order estimator would look like the following:

    \begin{tcolorbox}[breakable, size=fbox, boxrule=1pt, pad at break*=1mm,colback=cellbackground, colframe=cellborder]
\prompt{In}{incolor}{12}{\boxspacing}
\begin{Verbatim}[commandchars=\\\{\}]
\PY{k}{def} \PY{n+nf}{synthesize\PYZus{}data\PYZus{}and\PYZus{}reconstruct\PYZus{}noise\PYZus{}variance\PYZus{}2}\PY{p}{(}\PY{n}{freq\PYZus{}signal}\PY{p}{,} \PY{n}{freq\PYZus{}sample}\PY{p}{)}\PY{p}{:}
\PY{+w}{    }\PY{l+s+sd}{\PYZdq{}\PYZdq{}\PYZdq{}}
\PY{l+s+sd}{    Generates a synthetic dataset with known noise properties, and then reconstructs the noise variance using the}
\PY{l+s+sd}{    second\PYZhy{}order method described above.}

\PY{l+s+sd}{    The signal is assumed to be a sinusoid with some frequency, and the noise is assumed to be independent,}
\PY{l+s+sd}{    homoscedastic, and normally\PYZhy{}distributed.}

\PY{l+s+sd}{    Args:}
\PY{l+s+sd}{        freq\PYZus{}signal: Frequency of the underlying true signal [Hz]}
\PY{l+s+sd}{        freq\PYZus{}sample: Data sample rate from the sensor [Hz]}

\PY{l+s+sd}{    Returns: None. Prints the true noise standard deviation and the reconstructed (estimated) noise standard}
\PY{l+s+sd}{    deviation, as well as various input parameters.}
\PY{l+s+sd}{    \PYZdq{}\PYZdq{}\PYZdq{}}
    \PY{c+c1}{\PYZsh{}\PYZsh{}\PYZsh{}\PYZsh{}\PYZsh{} Synthetic Data Generation \PYZsh{}\PYZsh{}\PYZsh{}\PYZsh{}\PYZsh{}}
    \PY{n}{t} \PY{o}{=} \PY{n}{np}\PY{o}{.}\PY{n}{arange}\PY{p}{(}\PY{l+m+mi}{0}\PY{p}{,} \PY{l+m+mi}{5}\PY{p}{,} \PY{l+m+mi}{1} \PY{o}{/} \PY{n}{freq\PYZus{}sample}\PY{p}{)}

    \PY{n}{true\PYZus{}data} \PY{o}{=} \PY{n}{np}\PY{o}{.}\PY{n}{sin}\PY{p}{(}\PY{l+m+mi}{2} \PY{o}{*} \PY{n}{np}\PY{o}{.}\PY{n}{pi} \PY{o}{*} \PY{n}{freq\PYZus{}signal} \PY{o}{*} \PY{n}{t}\PY{p}{)}

    \PY{n}{true\PYZus{}noise\PYZus{}standard\PYZus{}deviation} \PY{o}{=} \PY{l+m+mf}{0.1}
    \PY{n}{noise} \PY{o}{=} \PY{n}{np}\PY{o}{.}\PY{n}{random}\PY{o}{.}\PY{n}{normal}\PY{p}{(}\PY{n}{size}\PY{o}{=}\PY{n+nb}{len}\PY{p}{(}\PY{n}{true\PYZus{}data}\PY{p}{)}\PY{p}{)} \PY{o}{*} \PY{n}{true\PYZus{}noise\PYZus{}standard\PYZus{}deviation}

    \PY{n}{sensor\PYZus{}data} \PY{o}{=} \PY{n}{true\PYZus{}data} \PY{o}{+} \PY{n}{noise}

    \PY{c+c1}{\PYZsh{}\PYZsh{}\PYZsh{}\PYZsh{}\PYZsh{} Noise Variance Reconstruction \PYZsh{}\PYZsh{}\PYZsh{}\PYZsh{}\PYZsh{}}
    \PY{n}{estimated\PYZus{}noise\PYZus{}standard\PYZus{}deviation} \PY{o}{=} \PY{p}{(}
            \PY{n}{np}\PY{o}{.}\PY{n}{std}\PY{p}{(}  \PY{c+c1}{\PYZsh{} Standard deviation}
                \PY{n}{np}\PY{o}{.}\PY{n}{diff}\PY{p}{(}\PY{n}{np}\PY{o}{.}\PY{n}{diff}\PY{p}{(}\PY{n}{sensor\PYZus{}data}\PY{p}{)}\PY{p}{)}  \PY{c+c1}{\PYZsh{} of the differences between subsequent samples}
            \PY{p}{)} \PY{o}{/} \PY{n}{np}\PY{o}{.}\PY{n}{sqrt}\PY{p}{(}\PY{l+m+mi}{6}\PY{p}{)}
    \PY{p}{)}

    \PY{n+nb}{print}\PY{p}{(}\PY{l+s+sa}{f}\PY{l+s+s2}{\PYZdq{}}\PY{l+s+s2}{Second\PYZhy{}Order Data\PYZhy{}Driven Noise Estimator}\PY{l+s+s2}{\PYZdq{}}\PY{p}{)}
    \PY{n+nb}{print}\PY{p}{(}\PY{l+s+sa}{f}\PY{l+s+s2}{\PYZdq{}}\PY{l+s+s2}{Sample frequency / Underlying Signal Frequency = }\PY{l+s+si}{\PYZob{}}\PY{l+m+mi}{1}\PY{+w}{ }\PY{o}{/}\PY{+w}{ }\PY{n}{np}\PY{o}{.}\PY{n}{diff}\PY{p}{(}\PY{n}{t}\PY{p}{)}\PY{p}{[}\PY{l+m+mi}{0}\PY{p}{]}\PY{l+s+si}{\PYZcb{}}\PY{l+s+s2}{\PYZdq{}}\PY{p}{)}
    \PY{n+nb}{print}\PY{p}{(}\PY{l+s+s2}{\PYZdq{}}\PY{l+s+s2}{\PYZhy{}}\PY{l+s+s2}{\PYZdq{}} \PY{o}{*} \PY{l+m+mi}{50}\PY{p}{)}
    \PY{n+nb}{print}\PY{p}{(}\PY{l+s+sa}{f}\PY{l+s+s2}{\PYZdq{}}\PY{l+s+s2}{True noise standard deviation: }\PY{l+s+si}{\PYZob{}}\PY{n}{true\PYZus{}noise\PYZus{}standard\PYZus{}deviation}\PY{l+s+si}{:}\PY{l+s+s2}{.3f}\PY{l+s+si}{\PYZcb{}}\PY{l+s+s2}{\PYZdq{}}\PY{p}{)}
    \PY{n+nb}{print}\PY{p}{(}\PY{l+s+sa}{f}\PY{l+s+s2}{\PYZdq{}}\PY{l+s+s2}{Estimated noise standard deviation: }\PY{l+s+si}{\PYZob{}}\PY{n}{estimated\PYZus{}noise\PYZus{}standard\PYZus{}deviation}\PY{l+s+si}{:}\PY{l+s+s2}{.3f}\PY{l+s+si}{\PYZcb{}}\PY{l+s+s2}{\PYZdq{}}\PY{p}{)}
\end{Verbatim}
\end{tcolorbox}

    This second-order estimator is still convergent when we have a wealth of
samples (i.e., the system is ``slow'' relative to the sample rate):

    \begin{tcolorbox}[breakable, size=fbox, boxrule=1pt, pad at break*=1mm,colback=cellbackground, colframe=cellborder]
\prompt{In}{incolor}{13}{\boxspacing}
\begin{Verbatim}[commandchars=\\\{\}]
\PY{n}{synthesize\PYZus{}data\PYZus{}and\PYZus{}reconstruct\PYZus{}noise\PYZus{}variance\PYZus{}2}\PY{p}{(}
    \PY{n}{freq\PYZus{}signal}\PY{o}{=}\PY{l+m+mi}{1}\PY{p}{,}
    \PY{n}{freq\PYZus{}sample}\PY{o}{=}\PY{l+m+mi}{1000}
\PY{p}{)}
\end{Verbatim}
\end{tcolorbox}

    \begin{Verbatim}[commandchars=\\\{\}]
Second-Order Data-Driven Noise Estimator
Sample frequency / Underlying Signal Frequency = 1000.0
--------------------------------------------------
True noise standard deviation: 0.100
Estimated noise standard deviation: 0.099
    \end{Verbatim}

    But, in comparison, its performance drop-off for ``fast'' systems is
much less severe than it is for the first-order estimator:

    \begin{tcolorbox}[breakable, size=fbox, boxrule=1pt, pad at break*=1mm,colback=cellbackground, colframe=cellborder]
\prompt{In}{incolor}{14}{\boxspacing}
\begin{Verbatim}[commandchars=\\\{\}]
\PY{n}{synthesize\PYZus{}data\PYZus{}and\PYZus{}reconstruct\PYZus{}noise\PYZus{}variance\PYZus{}2}\PY{p}{(}
    \PY{n}{freq\PYZus{}signal}\PY{o}{=}\PY{l+m+mi}{100}\PY{p}{,}
    \PY{n}{freq\PYZus{}sample}\PY{o}{=}\PY{l+m+mi}{1000}
\PY{p}{)}
\end{Verbatim}
\end{tcolorbox}

    \begin{Verbatim}[commandchars=\\\{\}]
Second-Order Data-Driven Noise Estimator
Sample frequency / Underlying Signal Frequency = 1000.0
--------------------------------------------------
True noise standard deviation: 0.100
Estimated noise standard deviation: 0.148
    \end{Verbatim}

    \hypertarget{arbitrary-order-estimators}{%
\paragraph{Arbitrary-Order
Estimators}\label{arbitrary-order-estimators}}

Clearly, the second-order estimator improves performance compared to the
first-order one. How far can we push this?

We can generalize the logic to an \(d\)-th order estimator by observing
that the denominator (2 for the first-order estimator, 6 for the
second-order estimator) is the sum of the squares of the first-order,
\(d\)-th-degree, uniform-grid finite difference coefficients (or,
perhaps more intuitively, the elements of the \(d\)-th row of Pascal's
triangle). Then, we use the combinatorial trick that:

\[{d \choose 0}^2 + {d \choose 1}^2 + \dots + {d \choose d}^2 = {2 d \choose d} = \frac{(2d)!}{(d!)^2}\]

to derive the following \(d\)-th order estimator of the noise variance:

\[\sigma^2 =
\frac{1}{{2 d \choose d} \cdot (N-d)}
\cdot \sum_{i=1}^{N-d} \left[
{d \choose 0} s(t_i)
- {d \choose 1} s(t_{i+1})
+ {d \choose 2} s(t_{i+2})
- {d \choose 3} s(t_{i+3})
+ \dots
\pm {d \choose d} s(t_{i+d})
\right]^2\]

which, in code, looks like:

    \begin{tcolorbox}[breakable, size=fbox, boxrule=1pt, pad at break*=1mm,colback=cellbackground, colframe=cellborder]
\prompt{In}{incolor}{15}{\boxspacing}
\begin{Verbatim}[commandchars=\\\{\}]
\PY{k}{def} \PY{n+nf}{synthesize\PYZus{}data\PYZus{}and\PYZus{}reconstruct\PYZus{}noise\PYZus{}variance}\PY{p}{(}\PY{n}{freq\PYZus{}signal}\PY{p}{,} \PY{n}{freq\PYZus{}sample}\PY{p}{,} \PY{n}{order}\PY{o}{=}\PY{l+m+mi}{1}\PY{p}{)}\PY{p}{:}
\PY{+w}{    }\PY{l+s+sd}{\PYZdq{}\PYZdq{}\PYZdq{}}
\PY{l+s+sd}{    Generates a synthetic dataset with known noise properties, and then reconstructs the noise variance using the}
\PY{l+s+sd}{    specified\PYZhy{}order method described above.}

\PY{l+s+sd}{    The signal is assumed to be a sinusoid with some frequency, and the noise is assumed to be independent,}
\PY{l+s+sd}{    homoscedastic, and normally\PYZhy{}distributed.}

\PY{l+s+sd}{    Args:}
\PY{l+s+sd}{        freq\PYZus{}signal: Frequency of the underlying true signal [Hz]}
\PY{l+s+sd}{        freq\PYZus{}sample: Data sample rate from the sensor [Hz]}
\PY{l+s+sd}{        order: Order of the estimator to use [int]}

\PY{l+s+sd}{    Returns: The true noise standard deviation and the reconstructed (estimated) noise standard deviation.}
\PY{l+s+sd}{    \PYZdq{}\PYZdq{}\PYZdq{}}
    \PY{c+c1}{\PYZsh{}\PYZsh{}\PYZsh{}\PYZsh{}\PYZsh{} Synthetic Data Generation \PYZsh{}\PYZsh{}\PYZsh{}\PYZsh{}\PYZsh{}}
    \PY{n}{t} \PY{o}{=} \PY{n}{np}\PY{o}{.}\PY{n}{arange}\PY{p}{(}\PY{l+m+mi}{0}\PY{p}{,} \PY{l+m+mi}{5}\PY{p}{,} \PY{l+m+mi}{1} \PY{o}{/} \PY{n}{freq\PYZus{}sample}\PY{p}{)}

    \PY{n}{true\PYZus{}data} \PY{o}{=} \PY{n}{np}\PY{o}{.}\PY{n}{sin}\PY{p}{(}\PY{l+m+mi}{2} \PY{o}{*} \PY{n}{np}\PY{o}{.}\PY{n}{pi} \PY{o}{*} \PY{n}{freq\PYZus{}signal} \PY{o}{*} \PY{n}{t}\PY{p}{)}

    \PY{n}{true\PYZus{}noise\PYZus{}standard\PYZus{}deviation} \PY{o}{=} \PY{l+m+mf}{0.1}
    \PY{n}{noise} \PY{o}{=} \PY{n}{np}\PY{o}{.}\PY{n}{random}\PY{o}{.}\PY{n}{normal}\PY{p}{(}\PY{n}{size}\PY{o}{=}\PY{n+nb}{len}\PY{p}{(}\PY{n}{true\PYZus{}data}\PY{p}{)}\PY{p}{)} \PY{o}{*} \PY{n}{true\PYZus{}noise\PYZus{}standard\PYZus{}deviation}

    \PY{n}{sensor\PYZus{}data} \PY{o}{=} \PY{n}{true\PYZus{}data} \PY{o}{+} \PY{n}{noise}

    \PY{c+c1}{\PYZsh{}\PYZsh{}\PYZsh{}\PYZsh{}\PYZsh{} Noise Variance Reconstruction \PYZsh{}\PYZsh{}\PYZsh{}\PYZsh{}\PYZsh{}}
    \PY{c+c1}{\PYZsh{}\PYZsh{}\PYZsh{} Note: there are a few differences between the equation above and this, to prevent overflow}
    \PY{n}{diff\PYZus{}data} \PY{o}{=} \PY{n}{sensor\PYZus{}data}

    \PY{k+kn}{from} \PY{n+nn}{scipy}\PY{n+nn}{.}\PY{n+nn}{special} \PY{k+kn}{import} \PY{n}{binom}
    \PY{n}{denominator} \PY{o}{=} \PY{n}{np}\PY{o}{.}\PY{n}{sqrt}\PY{p}{(}\PY{n}{binom}\PY{p}{(}\PY{l+m+mi}{2} \PY{o}{*} \PY{n}{order}\PY{p}{,} \PY{n}{order}\PY{p}{)}\PY{p}{)}

    \PY{k}{for} \PY{n}{\PYZus{}} \PY{o+ow}{in} \PY{n+nb}{range}\PY{p}{(}\PY{n}{order}\PY{p}{)}\PY{p}{:}
        \PY{n}{diff\PYZus{}data} \PY{o}{=} \PY{n}{np}\PY{o}{.}\PY{n}{diff}\PY{p}{(}\PY{n}{diff\PYZus{}data}\PY{p}{)} \PY{o}{/} \PY{n}{denominator} \PY{o}{*}\PY{o}{*} \PY{p}{(}\PY{l+m+mi}{1} \PY{o}{/} \PY{n}{order}\PY{p}{)}

    \PY{n}{estimated\PYZus{}noise\PYZus{}standard\PYZus{}deviation} \PY{o}{=} \PY{n}{np}\PY{o}{.}\PY{n}{std}\PY{p}{(}\PY{n}{diff\PYZus{}data}\PY{p}{)}

    \PY{k}{return} \PY{n}{true\PYZus{}noise\PYZus{}standard\PYZus{}deviation}\PY{p}{,} \PY{n}{estimated\PYZus{}noise\PYZus{}standard\PYZus{}deviation}
\end{Verbatim}
\end{tcolorbox}

    While we have shown that the second-order estimator has improved
robustness to low sample rates, it is not immediately clear that this
will hold for arbitrary orders - as the order increases, we might expect
to see some form of numerical instability as we take successively
higher-order derivatives of noisy data.

To test this, we can plot the performance of noise estimators of various
orders:

    \begin{tcolorbox}[breakable, size=fbox, boxrule=1pt, pad at break*=1mm,colback=cellbackground, colframe=cellborder]
\prompt{In}{incolor}{16}{\boxspacing}
\begin{Verbatim}[commandchars=\\\{\}]
\PY{n}{fig}\PY{p}{,} \PY{n}{ax} \PY{o}{=} \PY{n}{plt}\PY{o}{.}\PY{n}{subplots}\PY{p}{(}\PY{p}{)}

\PY{n}{freq\PYZus{}sample} \PY{o}{=} \PY{l+m+mi}{1000}
\PY{n}{freq\PYZus{}signals} \PY{o}{=} \PY{n}{freq\PYZus{}sample} \PY{o}{/} \PY{n}{np}\PY{o}{.}\PY{n}{logspace}\PY{p}{(}\PY{l+m+mi}{0}\PY{p}{,} \PY{l+m+mi}{3}\PY{p}{,} \PY{l+m+mi}{500}\PY{p}{)}
\PY{n}{orders} \PY{o}{=} \PY{p}{[}\PY{l+m+mi}{1}\PY{p}{,} \PY{l+m+mi}{2}\PY{p}{,} \PY{l+m+mi}{3}\PY{p}{,} \PY{l+m+mi}{4}\PY{p}{,} \PY{l+m+mi}{8}\PY{p}{,} \PY{l+m+mi}{16}\PY{p}{,} \PY{l+m+mi}{64}\PY{p}{,} \PY{l+m+mi}{512}\PY{p}{]}

\PY{n}{colors} \PY{o}{=} \PY{n}{p}\PY{o}{.}\PY{n}{sns}\PY{o}{.}\PY{n}{color\PYZus{}palette}\PY{p}{(}\PY{l+s+s1}{\PYZsq{}}\PY{l+s+s1}{rainbow}\PY{l+s+s1}{\PYZsq{}}\PY{p}{,} \PY{n}{n\PYZus{}colors}\PY{o}{=}\PY{n+nb}{len}\PY{p}{(}\PY{n}{orders}\PY{p}{)}\PY{p}{)}

\PY{k}{for} \PY{n}{i}\PY{p}{,} \PY{n}{order} \PY{o+ow}{in} \PY{n+nb}{enumerate}\PY{p}{(}\PY{n}{orders}\PY{p}{)}\PY{p}{:}
    \PY{n}{true}\PY{p}{,} \PY{n}{estimated} \PY{o}{=} \PY{n}{np}\PY{o}{.}\PY{n}{vectorize}\PY{p}{(}\PY{n}{synthesize\PYZus{}data\PYZus{}and\PYZus{}reconstruct\PYZus{}noise\PYZus{}variance}\PY{p}{)}\PY{p}{(}
        \PY{n}{freq\PYZus{}signal}\PY{o}{=}\PY{n}{freq\PYZus{}signals}\PY{p}{,}
        \PY{n}{freq\PYZus{}sample}\PY{o}{=}\PY{n}{freq\PYZus{}sample}\PY{p}{,}
        \PY{n}{order}\PY{o}{=}\PY{n}{order}\PY{p}{,}
    \PY{p}{)}
    \PY{n}{ratio} \PY{o}{=} \PY{n}{freq\PYZus{}sample} \PY{o}{/} \PY{n}{freq\PYZus{}signals}
    \PY{n}{error} \PY{o}{=} \PY{n}{np}\PY{o}{.}\PY{n}{abs}\PY{p}{(}\PY{n}{true} \PY{o}{\PYZhy{}} \PY{n}{estimated}\PY{p}{)} \PY{o}{/} \PY{n}{true}

    \PY{n}{index\PYZus{}noise\PYZus{}floor} \PY{o}{=} \PY{n}{np}\PY{o}{.}\PY{n}{argwhere}\PY{p}{(}
        \PY{p}{(}\PY{n}{np}\PY{o}{.}\PY{n}{arange}\PY{p}{(}\PY{n+nb}{len}\PY{p}{(}\PY{n}{error}\PY{p}{)}\PY{p}{)} \PY{o}{\PYZgt{}} \PY{n}{np}\PY{o}{.}\PY{n}{argmax}\PY{p}{(}\PY{n}{error}\PY{p}{)}\PY{p}{)} \PY{o}{\PYZam{}} \PY{p}{(}\PY{n}{error} \PY{o}{\PYZlt{}} \PY{l+m+mi}{2} \PY{o}{*} \PY{n}{np}\PY{o}{.}\PY{n}{median}\PY{p}{(}\PY{n}{error}\PY{p}{[}\PY{o}{\PYZhy{}}\PY{l+m+mi}{50}\PY{p}{:}\PY{p}{]}\PY{p}{)}\PY{p}{)}
    \PY{p}{)}\PY{p}{[}\PY{l+m+mi}{0}\PY{p}{]}\PY{p}{[}\PY{l+m+mi}{0}\PY{p}{]}

    \PY{n}{c} \PY{o}{=} \PY{n}{p}\PY{o}{.}\PY{n}{adjust\PYZus{}lightness}\PY{p}{(}\PY{n}{colors}\PY{p}{[}\PY{n}{i}\PY{p}{]}\PY{p}{,} \PY{l+m+mf}{0.6}\PY{p}{)}

    \PY{c+c1}{\PYZsh{} print(order)}
    \PY{c+c1}{\PYZsh{} print(np.mean(estimated[index\PYZus{}noise\PYZus{}floor+1:]), np.std(estimated[index\PYZus{}noise\PYZus{}floor+1:]))}

    \PY{n}{ax}\PY{o}{.}\PY{n}{loglog}\PY{p}{(}
        \PY{n}{ratio}\PY{p}{[}\PY{p}{:}\PY{n}{index\PYZus{}noise\PYZus{}floor} \PY{o}{+} \PY{l+m+mi}{1}\PY{p}{]}\PY{p}{,}
        \PY{n}{error}\PY{p}{[}\PY{p}{:}\PY{n}{index\PYZus{}noise\PYZus{}floor} \PY{o}{+} \PY{l+m+mi}{1}\PY{p}{]}\PY{p}{,}
        \PY{n}{label}\PY{o}{=}\PY{l+s+sa}{f}\PY{l+s+s2}{\PYZdq{}}\PY{l+s+si}{\PYZob{}}\PY{n}{order}\PY{l+s+si}{\PYZcb{}}\PY{l+s+s2}{\PYZdq{}}\PY{p}{,}
        \PY{n}{alpha}\PY{o}{=}\PY{l+m+mf}{0.8}\PY{p}{,} \PY{n}{color}\PY{o}{=}\PY{n}{c}\PY{p}{,} \PY{n}{linewidth}\PY{o}{=}\PY{l+m+mi}{1}\PY{p}{,}
        \PY{n}{zorder}\PY{o}{=}\PY{l+m+mi}{4} \PY{o}{+} \PY{n+nb}{len}\PY{p}{(}\PY{n}{orders}\PY{p}{)} \PY{o}{\PYZhy{}} \PY{n}{i}\PY{p}{,}
    \PY{p}{)}
    \PY{n}{ax}\PY{o}{.}\PY{n}{loglog}\PY{p}{(}
        \PY{n}{ratio}\PY{p}{[}\PY{n}{index\PYZus{}noise\PYZus{}floor}\PY{p}{:}\PY{p}{]}\PY{p}{,}
        \PY{n}{error}\PY{p}{[}\PY{n}{index\PYZus{}noise\PYZus{}floor}\PY{p}{:}\PY{p}{]}\PY{p}{,}
        \PY{n}{alpha}\PY{o}{=}\PY{l+m+mf}{0.2}\PY{p}{,} \PY{n}{color}\PY{o}{=}\PY{n}{c}\PY{p}{,} \PY{n}{linewidth}\PY{o}{=}\PY{l+m+mi}{1}\PY{p}{,}
        \PY{n}{zorder}\PY{o}{=}\PY{l+m+mi}{3}\PY{p}{,}
    \PY{p}{)}

\PY{n}{plt}\PY{o}{.}\PY{n}{xlim}\PY{p}{(}\PY{n}{left}\PY{o}{=}\PY{l+m+mi}{2}\PY{p}{,} \PY{n}{right}\PY{o}{=}\PY{p}{(}\PY{n}{freq\PYZus{}sample} \PY{o}{/} \PY{n}{freq\PYZus{}signals}\PY{p}{)}\PY{p}{[}\PY{o}{\PYZhy{}}\PY{l+m+mi}{1}\PY{p}{]}\PY{p}{)}
\PY{n}{plt}\PY{o}{.}\PY{n}{ylim}\PY{p}{(}\PY{n}{bottom}\PY{o}{=}\PY{l+m+mi}{1} \PY{o}{/} \PY{n}{freq\PYZus{}sample} \PY{o}{/} \PY{l+m+mi}{10}\PY{p}{)}

\PY{n}{plt}\PY{o}{.}\PY{n}{annotate}\PY{p}{(}
    \PY{n}{text}\PY{o}{=}\PY{l+s+s2}{\PYZdq{}}\PY{l+s+s2}{Nyquist Freq.,}\PY{l+s+se}{\PYZbs{}n}\PY{l+s+s2}{\PYZdl{}}\PY{l+s+s2}{\PYZbs{}}\PY{l+s+s2}{mathrm}\PY{l+s+si}{\PYZob{}Ratio\PYZcb{}}\PY{l+s+s2}{=2\PYZdl{}}\PY{l+s+s2}{\PYZdq{}}\PY{p}{,}
    \PY{n}{xy}\PY{o}{=}\PY{p}{(}\PY{l+m+mi}{0}\PY{p}{,} \PY{l+m+mi}{0}\PY{p}{)}\PY{p}{,}
    \PY{n}{xytext}\PY{o}{=}\PY{p}{(}\PY{l+m+mi}{0}\PY{p}{,} \PY{o}{\PYZhy{}}\PY{l+m+mi}{20}\PY{p}{)}\PY{p}{,}
    \PY{n}{xycoords}\PY{o}{=}\PY{l+s+s2}{\PYZdq{}}\PY{l+s+s2}{axes fraction}\PY{l+s+s2}{\PYZdq{}}\PY{p}{,}
    \PY{n}{textcoords}\PY{o}{=}\PY{l+s+s2}{\PYZdq{}}\PY{l+s+s2}{offset points}\PY{l+s+s2}{\PYZdq{}}\PY{p}{,}
    \PY{n}{ha}\PY{o}{=}\PY{l+s+s2}{\PYZdq{}}\PY{l+s+s2}{center}\PY{l+s+s2}{\PYZdq{}}\PY{p}{,}
    \PY{n}{va}\PY{o}{=}\PY{l+s+s2}{\PYZdq{}}\PY{l+s+s2}{top}\PY{l+s+s2}{\PYZdq{}}\PY{p}{,}
    \PY{n}{alpha}\PY{o}{=}\PY{l+m+mf}{0.6}\PY{p}{,}
    \PY{n}{fontsize}\PY{o}{=}\PY{l+m+mi}{8}\PY{p}{,}
    \PY{n}{arrowprops}\PY{o}{=}\PY{n+nb}{dict}\PY{p}{(}
        \PY{n}{arrowstyle}\PY{o}{=}\PY{l+s+s2}{\PYZdq{}}\PY{l+s+s2}{simple}\PY{l+s+s2}{\PYZdq{}}\PY{p}{,}
        \PY{c+c1}{\PYZsh{} connectionstyle=\PYZdq{}arc3,rad=0.2\PYZdq{},}
        \PY{n}{linewidth}\PY{o}{=}\PY{l+m+mi}{0}\PY{p}{,}
        \PY{n}{alpha}\PY{o}{=}\PY{l+m+mf}{0.5}\PY{p}{,}
        \PY{n}{facecolor}\PY{o}{=}\PY{l+s+s2}{\PYZdq{}}\PY{l+s+s2}{k}\PY{l+s+s2}{\PYZdq{}}\PY{p}{,}
    \PY{p}{)}\PY{p}{,}
\PY{p}{)}
\PY{n}{plt}\PY{o}{.}\PY{n}{annotate}\PY{p}{(}
    \PY{n}{text}\PY{o}{=}\PY{l+s+s2}{\PYZdq{}}\PY{l+s+s2}{For readability, lines are faded once they hit the theoretical}\PY{l+s+se}{\PYZbs{}n}\PY{l+s+s2}{minimum\PYZhy{}possible estimator error, roughly \PYZdl{}}\PY{l+s+se}{\PYZbs{}\PYZbs{}}\PY{l+s+s2}{epsilon }\PY{l+s+se}{\PYZbs{}\PYZbs{}}\PY{l+s+s2}{approx N\PYZca{}}\PY{l+s+s2}{\PYZob{}}\PY{l+s+s2}{\PYZhy{}0.5\PYZcb{}\PYZdl{}}\PY{l+s+s2}{\PYZdq{}}\PY{p}{,}
    \PY{n}{xy}\PY{o}{=}\PY{p}{(}\PY{l+m+mf}{0.02}\PY{p}{,} \PY{l+m+mf}{0.02}\PY{p}{)}\PY{p}{,}
    \PY{n}{xycoords}\PY{o}{=}\PY{l+s+s2}{\PYZdq{}}\PY{l+s+s2}{axes fraction}\PY{l+s+s2}{\PYZdq{}}\PY{p}{,}
    \PY{n}{ha}\PY{o}{=}\PY{l+s+s2}{\PYZdq{}}\PY{l+s+s2}{left}\PY{l+s+s2}{\PYZdq{}}\PY{p}{,}
    \PY{n}{va}\PY{o}{=}\PY{l+s+s2}{\PYZdq{}}\PY{l+s+s2}{bottom}\PY{l+s+s2}{\PYZdq{}}\PY{p}{,}
    \PY{n}{fontsize}\PY{o}{=}\PY{l+m+mi}{9}
\PY{p}{)}

\PY{n}{plt}\PY{o}{.}\PY{n}{legend}\PY{p}{(}
    \PY{n}{title}\PY{o}{=}\PY{l+s+s2}{\PYZdq{}}\PY{l+s+s2}{Estimator Order}\PY{l+s+s2}{\PYZdq{}}\PY{p}{,}
    \PY{n}{ncols}\PY{o}{=}\PY{l+m+mi}{2}\PY{p}{,}
\PY{p}{)}
\PY{n}{p}\PY{o}{.}\PY{n}{show\PYZus{}plot}\PY{p}{(}
    \PY{n}{title}\PY{o}{=}\PY{l+s+s2}{\PYZdq{}}\PY{l+s+s2}{Performance of Higher\PYZhy{}Order Data\PYZhy{}Driven Noise Estimators}\PY{l+s+s2}{\PYZdq{}}\PY{p}{,}
    \PY{n}{xlabel}\PY{o}{=}\PY{l+s+sa}{r}\PY{l+s+s2}{\PYZdq{}}\PY{l+s+s2}{Ratio of \PYZdl{}}\PY{l+s+s2}{\PYZbs{}}\PY{l+s+s2}{frac}\PY{l+s+s2}{\PYZob{}}\PY{l+s+s2}{\PYZbs{}}\PY{l+s+s2}{mathrm}\PY{l+s+s2}{\PYZob{}}\PY{l+s+s2}{Sample}\PY{l+s+s2}{\PYZbs{}}\PY{l+s+s2}{ Frequency\PYZcb{}\PYZcb{}}\PY{l+s+s2}{\PYZob{}}\PY{l+s+s2}{\PYZbs{}}\PY{l+s+s2}{mathrm}\PY{l+s+s2}{\PYZob{}}\PY{l+s+s2}{Underlying}\PY{l+s+s2}{\PYZbs{}}\PY{l+s+s2}{ Signal}\PY{l+s+s2}{\PYZbs{}}\PY{l+s+s2}{ Frequency\PYZcb{}\PYZcb{}\PYZdl{}}\PY{l+s+s2}{\PYZdq{}}\PY{p}{,}
    \PY{n}{ylabel}\PY{o}{=}\PY{l+s+sa}{r}\PY{l+s+s2}{\PYZdq{}}\PY{l+s+s2}{Relative Error of Estimator \PYZdl{}}\PY{l+s+s2}{\PYZbs{}}\PY{l+s+s2}{epsilon = }\PY{l+s+s2}{\PYZbs{}}\PY{l+s+s2}{left| }\PY{l+s+s2}{\PYZbs{}}\PY{l+s+s2}{frac}\PY{l+s+s2}{\PYZob{}}\PY{l+s+s2}{ }\PY{l+s+s2}{\PYZbs{}}\PY{l+s+s2}{sigma\PYZus{}}\PY{l+s+s2}{\PYZob{}}\PY{l+s+s2}{\PYZbs{}}\PY{l+s+s2}{mathrm}\PY{l+s+si}{\PYZob{}estimated\PYZcb{}}\PY{l+s+s2}{\PYZcb{} \PYZhy{} }\PY{l+s+s2}{\PYZbs{}}\PY{l+s+s2}{sigma\PYZus{}}\PY{l+s+s2}{\PYZob{}}\PY{l+s+s2}{\PYZbs{}}\PY{l+s+s2}{mathrm}\PY{l+s+si}{\PYZob{}true\PYZcb{}}\PY{l+s+s2}{\PYZcb{} \PYZcb{}}\PY{l+s+s2}{\PYZob{}}\PY{l+s+s2}{\PYZbs{}}\PY{l+s+s2}{sigma\PYZus{}}\PY{l+s+s2}{\PYZob{}}\PY{l+s+s2}{\PYZbs{}}\PY{l+s+s2}{mathrm}\PY{l+s+si}{\PYZob{}true\PYZcb{}}\PY{l+s+s2}{\PYZcb{}\PYZcb{}}\PY{l+s+s2}{\PYZbs{}}\PY{l+s+s2}{right|\PYZdl{}}\PY{l+s+s2}{\PYZdq{}}\PY{p}{,}
    \PY{n}{legend}\PY{o}{=}\PY{k+kc}{False}\PY{p}{,}
    \PY{n}{dpi}\PY{o}{=}\PY{l+m+mi}{300}\PY{p}{,}
\PY{p}{)}
\end{Verbatim}
\end{tcolorbox}

    \begin{center}
    \adjustimage{max size={0.9\linewidth}{0.9\paperheight}}{output_56_0.png}
    \end{center}
    { \hspace*{\fill} \\}
    
    The figure above shows something quite remarkable: we can use
extremely-high-order estimators (even up to the 512th order, i.e.,
taking the 512th-derivative of raw, noisy data via finite differences),
and we see not only that it's numerically stable, but also that it
accurately estimates the true noise variance even when the underlying
signal is almost at the Nyquist frequency.

Interestingly, except for a miniscule region near the Nyquist frequency,
estimator performance seems to monotonically improve with increasing
estimator order. Since practical data reconstruction would likely have
the \(f_{\rm sample} / f_{\rm signal}\) ratio be much larger than 2,
this suggests that we can use extremely-high-order estimators to
estimate the noise variance with very high accuracy.

    \hypertarget{bootstrapped-uncertainty-quantification-using-noise-estimators}{%
\subsubsection{Bootstrapped Uncertainty Quantification using Noise
Estimators}\label{bootstrapped-uncertainty-quantification-using-noise-estimators}}

    An added benefit of rigorously estimating the noise variance is that we
can use this to improve the uncertainty quantification of our air
vehicle state estimation with minimal extra effort.

For example, consider uncertainty quantification through a resampling
boostrap.

    \begin{tcolorbox}[breakable, size=fbox, boxrule=1pt, pad at break*=1mm,colback=cellbackground, colframe=cellborder]
\prompt{In}{incolor}{17}{\boxspacing}
\begin{Verbatim}[commandchars=\\\{\}]
\PY{k}{def} \PY{n+nf}{read\PYZus{}and\PYZus{}estimate\PYZus{}noise}\PY{p}{(}\PY{n}{name}\PY{p}{,} \PY{n}{estimator\PYZus{}order}\PY{o}{=}\PY{l+m+mi}{100}\PY{p}{)}\PY{p}{:}
    \PY{n}{source} \PY{o}{=} \PY{n}{data\PYZus{}sources}\PY{p}{[}\PY{n}{name}\PY{p}{]}\PY{p}{[}\PY{l+m+mi}{0}\PY{p}{]}
    \PY{n}{colname} \PY{o}{=} \PY{n}{data\PYZus{}sources}\PY{p}{[}\PY{n}{name}\PY{p}{]}\PY{p}{[}\PY{l+m+mi}{1}\PY{p}{]}

    \PY{n}{df} \PY{o}{=} \PY{n}{pd}\PY{o}{.}\PY{n}{read\PYZus{}csv}\PY{p}{(}\PY{n}{source}\PY{p}{)}

    \PY{n}{raw\PYZus{}time} \PY{o}{=} \PY{p}{(}\PY{n}{df}\PY{p}{[}\PY{l+s+s2}{\PYZdq{}}\PY{l+s+s2}{timestamp}\PY{l+s+s2}{\PYZdq{}}\PY{p}{]}\PY{o}{.}\PY{n}{values}\PY{o}{.}\PY{n}{astype}\PY{p}{(}\PY{n+nb}{float}\PY{p}{)} \PY{o}{\PYZhy{}} \PY{n}{timestamp\PYZus{}0}\PY{p}{)} \PY{o}{/} \PY{l+m+mf}{1e6}
    \PY{n}{data} \PY{o}{=} \PY{n}{df}\PY{p}{[}\PY{n}{colname}\PY{p}{]}\PY{o}{.}\PY{n}{values}\PY{o}{.}\PY{n}{astype}\PY{p}{(}\PY{n+nb}{float}\PY{p}{)}

    \PY{n}{mask} \PY{o}{=} \PY{p}{(}\PY{n}{raw\PYZus{}time} \PY{o}{\PYZgt{}} \PY{n}{raw\PYZus{}time\PYZus{}takeoff}\PY{p}{)} \PY{o}{\PYZam{}} \PY{p}{(}\PY{n}{raw\PYZus{}time} \PY{o}{\PYZlt{}} \PY{n}{raw\PYZus{}time\PYZus{}landing}\PY{p}{)}

    \PY{n}{time} \PY{o}{=} \PY{n}{raw\PYZus{}time}\PY{p}{[}\PY{n}{mask}\PY{p}{]} \PY{o}{\PYZhy{}} \PY{n}{raw\PYZus{}time\PYZus{}takeoff}
    \PY{n}{data} \PY{o}{=} \PY{n}{data}\PY{p}{[}\PY{n}{mask}\PY{p}{]}

    \PY{n}{diff\PYZus{}data} \PY{o}{=} \PY{n}{data}

    \PY{k}{for} \PY{n}{\PYZus{}} \PY{o+ow}{in} \PY{n+nb}{range}\PY{p}{(}\PY{n}{estimator\PYZus{}order}\PY{p}{)}\PY{p}{:}
        \PY{n}{diff\PYZus{}data} \PY{o}{=} \PY{n}{np}\PY{o}{.}\PY{n}{diff}\PY{p}{(}\PY{n}{diff\PYZus{}data}\PY{p}{)}

    \PY{k+kn}{from} \PY{n+nn}{scipy}\PY{n+nn}{.}\PY{n+nn}{special} \PY{k+kn}{import} \PY{n}{binom}

    \PY{n}{estimated\PYZus{}noise\PYZus{}standard\PYZus{}deviation} \PY{o}{=} \PY{p}{(}
        \PY{n}{np}\PY{o}{.}\PY{n}{std}\PY{p}{(}  \PY{c+c1}{\PYZsh{} Standard deviation}
            \PY{n}{diff\PYZus{}data} \PY{o}{/} \PY{n}{np}\PY{o}{.}\PY{n}{sqrt}\PY{p}{(}\PY{n}{binom}\PY{p}{(}\PY{l+m+mi}{2} \PY{o}{*} \PY{n}{estimator\PYZus{}order}\PY{p}{,} \PY{n}{estimator\PYZus{}order}\PY{p}{)}\PY{p}{)}
        \PY{p}{)}
    \PY{p}{)}

    \PY{k}{return} \PY{n}{time}\PY{p}{,} \PY{n}{data}\PY{p}{,} \PY{n}{estimated\PYZus{}noise\PYZus{}standard\PYZus{}deviation}
\end{Verbatim}
\end{tcolorbox}

    \begin{tcolorbox}[breakable, size=fbox, boxrule=1pt, pad at break*=1mm,colback=cellbackground, colframe=cellborder]
\prompt{In}{incolor}{18}{\boxspacing}
\begin{Verbatim}[commandchars=\\\{\}]
\PY{n}{x}\PY{p}{,} \PY{n}{y}\PY{p}{,} \PY{n}{y\PYZus{}stdev} \PY{o}{=} \PY{n}{read\PYZus{}and\PYZus{}estimate\PYZus{}noise}\PY{p}{(}\PY{l+s+s2}{\PYZdq{}}\PY{l+s+s2}{airspeed}\PY{l+s+s2}{\PYZdq{}}\PY{p}{,} \PY{n}{estimator\PYZus{}order}\PY{o}{=}\PY{l+m+mi}{100}\PY{p}{)}

\PY{n}{fig}\PY{p}{,} \PY{n}{ax} \PY{o}{=} \PY{n}{plt}\PY{o}{.}\PY{n}{subplots}\PY{p}{(}\PY{n}{figsize}\PY{o}{=}\PY{p}{(}\PY{l+m+mi}{7}\PY{p}{,} \PY{l+m+mf}{6.4}\PY{p}{)}\PY{p}{)}

\PY{n}{color}\PY{o}{=}\PY{l+s+s2}{\PYZdq{}}\PY{l+s+s2}{teal}\PY{l+s+s2}{\PYZdq{}}

\PY{n}{x\PYZus{}fit}\PY{p}{,} \PY{n}{y\PYZus{}bootstrap\PYZus{}fits} \PY{o}{=} \PY{n}{p}\PY{o}{.}\PY{n}{plot\PYZus{}with\PYZus{}bootstrapped\PYZus{}uncertainty}\PY{p}{(}
    \PY{n}{x}\PY{p}{,}
    \PY{n}{y}\PY{p}{,}
    \PY{n}{y\PYZus{}stdev} \PY{o}{=} \PY{n}{y\PYZus{}stdev}\PY{p}{,}
    \PY{n}{color}\PY{o}{=}\PY{n}{color}\PY{p}{,}
    \PY{n}{n\PYZus{}fit\PYZus{}points}\PY{o}{=}\PY{l+m+mi}{5000}\PY{p}{,}
    \PY{n}{draw\PYZus{}data}\PY{o}{=}\PY{k+kc}{False}\PY{p}{,}
    \PY{n}{draw\PYZus{}ci}\PY{o}{=}\PY{k+kc}{False}\PY{p}{,}
    \PY{n}{label\PYZus{}line}\PY{o}{=}\PY{l+s+s2}{\PYZdq{}}\PY{l+s+s2}{Best Estimate}\PY{l+s+s2}{\PYZdq{}}\PY{p}{,}
\PY{p}{)}
\PY{n}{ci\PYZus{}ranges} \PY{o}{=} \PY{p}{(}\PY{l+m+mf}{0.50}\PY{p}{,} \PY{l+m+mf}{0.95}\PY{p}{)}
\PY{k}{for} \PY{n}{ci} \PY{o+ow}{in} \PY{n}{ci\PYZus{}ranges}\PY{p}{:}
    \PY{n}{plt}\PY{o}{.}\PY{n}{fill\PYZus{}between}\PY{p}{(}
        \PY{n}{x\PYZus{}fit}\PY{p}{,}
        \PY{n}{np}\PY{o}{.}\PY{n}{nanquantile}\PY{p}{(}\PY{n}{y\PYZus{}bootstrap\PYZus{}fits}\PY{p}{,} \PY{n}{q}\PY{o}{=}\PY{p}{(}\PY{l+m+mi}{1} \PY{o}{\PYZhy{}} \PY{n}{ci}\PY{p}{)} \PY{o}{/} \PY{l+m+mi}{2}\PY{p}{,} \PY{n}{axis}\PY{o}{=}\PY{l+m+mi}{0}\PY{p}{)}\PY{p}{,}  \PY{c+c1}{\PYZsh{} Use the equal\PYZhy{}tails method}
        \PY{n}{np}\PY{o}{.}\PY{n}{nanquantile}\PY{p}{(}\PY{n}{y\PYZus{}bootstrap\PYZus{}fits}\PY{p}{,} \PY{n}{q}\PY{o}{=}\PY{l+m+mi}{1} \PY{o}{\PYZhy{}} \PY{p}{(}\PY{l+m+mi}{1} \PY{o}{\PYZhy{}} \PY{n}{ci}\PY{p}{)} \PY{o}{/} \PY{l+m+mi}{2}\PY{p}{,} \PY{n}{axis}\PY{o}{=}\PY{l+m+mi}{0}\PY{p}{)}\PY{p}{,}
        \PY{n}{color}\PY{o}{=}\PY{n}{color}\PY{p}{,}
        \PY{n}{label}\PY{o}{=}\PY{n}{ci}\PY{p}{,}
        \PY{n}{alpha}\PY{o}{=}\PY{l+m+mf}{0.25}\PY{p}{,}
        \PY{n}{linewidth}\PY{o}{=}\PY{l+m+mi}{0}
    \PY{p}{)}
\PY{n}{plt}\PY{o}{.}\PY{n}{plot}\PY{p}{(}
    \PY{n}{x}\PY{p}{,} \PY{n}{y}\PY{p}{,}
    \PY{l+s+s2}{\PYZdq{}}\PY{l+s+s2}{.k}\PY{l+s+s2}{\PYZdq{}}\PY{p}{,}
    \PY{n}{label}\PY{o}{=}\PY{l+s+s2}{\PYZdq{}}\PY{l+s+s2}{Raw Data}\PY{l+s+s2}{\PYZdq{}}\PY{p}{,}
    \PY{n}{alpha}\PY{o}{=}\PY{l+m+mf}{0.5}\PY{p}{,}
    \PY{n}{markersize}\PY{o}{=}\PY{l+m+mi}{6}\PY{p}{,}
    \PY{n}{markeredgewidth}\PY{o}{=}\PY{l+m+mi}{0}\PY{p}{,}
    \PY{n}{zorder}\PY{o}{=}\PY{l+m+mf}{1.9}
\PY{p}{)}

\PY{n}{xlim} \PY{o}{=} \PY{p}{(}\PY{l+m+mi}{60}\PY{p}{,} \PY{l+m+mi}{90}\PY{p}{)}
\PY{n}{plt}\PY{o}{.}\PY{n}{xlim}\PY{p}{(}\PY{o}{*}\PY{n}{xlim}\PY{p}{)}
\PY{n}{plt}\PY{o}{.}\PY{n}{ylim}\PY{p}{(}\PY{l+m+mi}{6}\PY{p}{,} \PY{l+m+mi}{13}\PY{p}{)}
\PY{n}{p}\PY{o}{.}\PY{n}{set\PYZus{}ticks}\PY{p}{(}\PY{l+m+mi}{5}\PY{p}{,} \PY{l+m+mi}{1}\PY{p}{,} \PY{l+m+mi}{1}\PY{p}{,} \PY{l+m+mf}{0.2}\PY{p}{)}

\PY{n}{p}\PY{o}{.}\PY{n}{show\PYZus{}plot}\PY{p}{(}
    \PY{l+s+sa}{f}\PY{l+s+s2}{\PYZdq{}}\PY{l+s+s2}{Probabilistic State Reconstruction with Improved Noise Estimates}\PY{l+s+se}{\PYZbs{}n}\PY{l+s+s2}{Airspeed data, zoomed to \PYZdl{}t }\PY{l+s+s2}{\PYZbs{}}\PY{l+s+s2}{in }\PY{l+s+si}{\PYZob{}}\PY{n}{xlim}\PY{l+s+si}{\PYZcb{}}\PY{l+s+s2}{\PYZdl{}}\PY{l+s+s2}{\PYZdq{}}\PY{p}{,}
    \PY{l+s+s2}{\PYZdq{}}\PY{l+s+s2}{Time after Takeoff [sec]}\PY{l+s+s2}{\PYZdq{}}\PY{p}{,}
    \PY{l+s+s2}{\PYZdq{}}\PY{l+s+s2}{Airspeed [m/s]}\PY{l+s+s2}{\PYZdq{}}\PY{p}{,}
    \PY{n}{dpi}\PY{o}{=}\PY{l+m+mi}{300}\PY{p}{,}
\PY{p}{)}
\end{Verbatim}
\end{tcolorbox}

    \begin{Verbatim}[commandchars=\\\{\}, frame=single, framerule=2mm, rulecolor=\color{outerrorbackground}]
\textcolor{ansi-red-intense}{\textbf{---------------------------------------------------------------------------}}
\textcolor{ansi-red-intense}{\textbf{TypeError}}                                 Traceback (most recent call last)
Cell \textcolor{ansi-green-intense}{\textbf{In[18], line 7}}
\textcolor{ansi-green}{      3} fig, ax \def\tcRGB{\textcolor[RGB]}\expandafter\tcRGB\expandafter{\detokenize{98,98,98}}{=} plt\def\tcRGB{\textcolor[RGB]}\expandafter\tcRGB\expandafter{\detokenize{98,98,98}}{.}subplots(figsize\def\tcRGB{\textcolor[RGB]}\expandafter\tcRGB\expandafter{\detokenize{98,98,98}}{=}(\def\tcRGB{\textcolor[RGB]}\expandafter\tcRGB\expandafter{\detokenize{98,98,98}}{7}, \def\tcRGB{\textcolor[RGB]}\expandafter\tcRGB\expandafter{\detokenize{98,98,98}}{6.4}))
\textcolor{ansi-green}{      5} color\def\tcRGB{\textcolor[RGB]}\expandafter\tcRGB\expandafter{\detokenize{98,98,98}}{=}\def\tcRGB{\textcolor[RGB]}\expandafter\tcRGB\expandafter{\detokenize{175,0,0}}{"}\def\tcRGB{\textcolor[RGB]}\expandafter\tcRGB\expandafter{\detokenize{175,0,0}}{teal}\def\tcRGB{\textcolor[RGB]}\expandafter\tcRGB\expandafter{\detokenize{175,0,0}}{"}
\textcolor{ansi-green-intense}{\textbf{----> 7}} x\_fit, y\_bootstrap\_fits \def\tcRGB{\textcolor[RGB]}\expandafter\tcRGB\expandafter{\detokenize{98,98,98}}{=} \setlength{\fboxsep}{0pt}\colorbox{ansi-yellow}{p\strut}\def\tcRGB{\textcolor[RGB]}\expandafter\tcRGB\expandafter{\detokenize{98,98,98}}{\setlength{\fboxsep}{0pt}\colorbox{ansi-yellow}{.\strut}}\setlength{\fboxsep}{0pt}\colorbox{ansi-yellow}{plot\_with\_bootstrapped\_uncertainty\strut}\setlength{\fboxsep}{0pt}\colorbox{ansi-yellow}{(\strut}
\textcolor{ansi-green}{      8} \setlength{\fboxsep}{0pt}\colorbox{ansi-yellow}{    \strut}\setlength{\fboxsep}{0pt}\colorbox{ansi-yellow}{x\strut}\setlength{\fboxsep}{0pt}\colorbox{ansi-yellow}{,\strut}
\textcolor{ansi-green}{      9} \setlength{\fboxsep}{0pt}\colorbox{ansi-yellow}{    \strut}\setlength{\fboxsep}{0pt}\colorbox{ansi-yellow}{y\strut}\setlength{\fboxsep}{0pt}\colorbox{ansi-yellow}{,\strut}
\textcolor{ansi-green}{     10} \setlength{\fboxsep}{0pt}\colorbox{ansi-yellow}{    \strut}\setlength{\fboxsep}{0pt}\colorbox{ansi-yellow}{y\_stdev\strut}\setlength{\fboxsep}{0pt}\colorbox{ansi-yellow}{ \strut}\def\tcRGB{\textcolor[RGB]}\expandafter\tcRGB\expandafter{\detokenize{98,98,98}}{\setlength{\fboxsep}{0pt}\colorbox{ansi-yellow}{=\strut}}\setlength{\fboxsep}{0pt}\colorbox{ansi-yellow}{ \strut}\setlength{\fboxsep}{0pt}\colorbox{ansi-yellow}{y\_stdev\strut}\setlength{\fboxsep}{0pt}\colorbox{ansi-yellow}{,\strut}
\textcolor{ansi-green}{     11} \setlength{\fboxsep}{0pt}\colorbox{ansi-yellow}{    \strut}\setlength{\fboxsep}{0pt}\colorbox{ansi-yellow}{color\strut}\def\tcRGB{\textcolor[RGB]}\expandafter\tcRGB\expandafter{\detokenize{98,98,98}}{\setlength{\fboxsep}{0pt}\colorbox{ansi-yellow}{=\strut}}\setlength{\fboxsep}{0pt}\colorbox{ansi-yellow}{color\strut}\setlength{\fboxsep}{0pt}\colorbox{ansi-yellow}{,\strut}
\textcolor{ansi-green}{     12} \setlength{\fboxsep}{0pt}\colorbox{ansi-yellow}{    \strut}\setlength{\fboxsep}{0pt}\colorbox{ansi-yellow}{n\_fit\_points\strut}\def\tcRGB{\textcolor[RGB]}\expandafter\tcRGB\expandafter{\detokenize{98,98,98}}{\setlength{\fboxsep}{0pt}\colorbox{ansi-yellow}{=\strut}}\def\tcRGB{\textcolor[RGB]}\expandafter\tcRGB\expandafter{\detokenize{98,98,98}}{\setlength{\fboxsep}{0pt}\colorbox{ansi-yellow}{5000\strut}}\setlength{\fboxsep}{0pt}\colorbox{ansi-yellow}{,\strut}
\textcolor{ansi-green}{     13} \setlength{\fboxsep}{0pt}\colorbox{ansi-yellow}{    \strut}\setlength{\fboxsep}{0pt}\colorbox{ansi-yellow}{draw\_data\strut}\def\tcRGB{\textcolor[RGB]}\expandafter\tcRGB\expandafter{\detokenize{98,98,98}}{\setlength{\fboxsep}{0pt}\colorbox{ansi-yellow}{=\strut}}\def\tcRGB{\textcolor[RGB]}\expandafter\tcRGB\expandafter{\detokenize{0,135,0}}{\setlength{\fboxsep}{0pt}\colorbox{ansi-yellow}{\textbf{False}\strut}}\setlength{\fboxsep}{0pt}\colorbox{ansi-yellow}{,\strut}
\textcolor{ansi-green}{     14} \setlength{\fboxsep}{0pt}\colorbox{ansi-yellow}{    \strut}\setlength{\fboxsep}{0pt}\colorbox{ansi-yellow}{draw\_ci\strut}\def\tcRGB{\textcolor[RGB]}\expandafter\tcRGB\expandafter{\detokenize{98,98,98}}{\setlength{\fboxsep}{0pt}\colorbox{ansi-yellow}{=\strut}}\def\tcRGB{\textcolor[RGB]}\expandafter\tcRGB\expandafter{\detokenize{0,135,0}}{\setlength{\fboxsep}{0pt}\colorbox{ansi-yellow}{\textbf{False}\strut}}\setlength{\fboxsep}{0pt}\colorbox{ansi-yellow}{,\strut}
\textcolor{ansi-green}{     15} \setlength{\fboxsep}{0pt}\colorbox{ansi-yellow}{    \strut}\setlength{\fboxsep}{0pt}\colorbox{ansi-yellow}{label\_line\strut}\def\tcRGB{\textcolor[RGB]}\expandafter\tcRGB\expandafter{\detokenize{98,98,98}}{\setlength{\fboxsep}{0pt}\colorbox{ansi-yellow}{=\strut}}\def\tcRGB{\textcolor[RGB]}\expandafter\tcRGB\expandafter{\detokenize{175,0,0}}{\setlength{\fboxsep}{0pt}\colorbox{ansi-yellow}{"\strut}}\def\tcRGB{\textcolor[RGB]}\expandafter\tcRGB\expandafter{\detokenize{175,0,0}}{\setlength{\fboxsep}{0pt}\colorbox{ansi-yellow}{Best Estimate\strut}}\def\tcRGB{\textcolor[RGB]}\expandafter\tcRGB\expandafter{\detokenize{175,0,0}}{\setlength{\fboxsep}{0pt}\colorbox{ansi-yellow}{"\strut}}\setlength{\fboxsep}{0pt}\colorbox{ansi-yellow}{,\strut}
\textcolor{ansi-green}{     16} \setlength{\fboxsep}{0pt}\colorbox{ansi-yellow}{)\strut}
\textcolor{ansi-green}{     17} ci\_ranges \def\tcRGB{\textcolor[RGB]}\expandafter\tcRGB\expandafter{\detokenize{98,98,98}}{=} (\def\tcRGB{\textcolor[RGB]}\expandafter\tcRGB\expandafter{\detokenize{98,98,98}}{0.50}, \def\tcRGB{\textcolor[RGB]}\expandafter\tcRGB\expandafter{\detokenize{98,98,98}}{0.95})
\textcolor{ansi-green}{     18} \def\tcRGB{\textcolor[RGB]}\expandafter\tcRGB\expandafter{\detokenize{0,135,0}}{\textbf{for}} ci \def\tcRGB{\textcolor[RGB]}\expandafter\tcRGB\expandafter{\detokenize{175,0,255}}{\textbf{in}} ci\_ranges:

\textcolor{ansi-red-intense}{\textbf{TypeError}}: plot\_with\_bootstrapped\_uncertainty() got an unexpected keyword argument 'draw\_ci'
    \end{Verbatim}

    \begin{center}
    \adjustimage{max size={0.9\linewidth}{0.9\paperheight}}{output_61_1.png}
    \end{center}
    { \hspace*{\fill} \\}
    
    


    % Add a bibliography block to the postdoc
    
    
    
\end{document}
